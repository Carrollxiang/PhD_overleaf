\documentclass[../main.tex]{subfiles}
\begin{document}

\chapter{里德堡原子阵列中的量子多体疤痕与信息输运}

摘要: 封闭量子多体系统的热化动力学通常遵循本征态热化假说(ETH),初始的局部量子信息会迅速扩散到整个系统而变得不可恢复。然而,近期发现的“量子多体疤痕”(Quantum Many-Body Scars)现象挑战了这一范式,表明在某些强相互作用系统中存在弱各态历经性破缺,允许系统在长时间内保持相干性。 本章首先介绍基于里德堡原子阵列实现的 PXP 模型及其动力学约束机制。我们详细阐述了如何在实验上高保真度地制备与疤痕态子空间重叠极大的反铁磁 $\mathbb{Z}_2$ 初态。为了深入探究该系统中的量子信息动力学,我们引入了 Holevo 信息(Holevo Information)作为探测工具。与传统的布居数测量不同,Holevo 信息能够量化量子通道中可传输信息的上限。实验结果显示,在疤痕态演化过程中,Holevo 信息表现出显著的非马尔可夫(Non-Markovian)特征,即原本耗散到环境中的信息会周期性地回流到局部子系统中。这一发现从量子信息输运的角度揭示了动力学约束对多体系统热化的抑制作用,为后续章节研究更复杂的信息加扰动力学奠定了基础。

\section{引言}
量子信息加扰(Quantum information scrambling)——即初始局部信息在相互作用多体系统中扩散至整个系统的过程\cite{lewis2019dynamics, swingle2018unscrambling}——揭示了量子热化与混沌\cite{deutsch1991quantum,srednicki1994chaos,rigol2008thermalization,sekino2008fast,hosur2016chaos,maldacena2016bound}、黑洞信息悖论\cite{hayden2007black,shenker2014black}以及量子机器学习\cite{Shen2020Information,garcia2022Quantifying}中的基本物理机制。理解加扰动力学对于推进量子模拟、基准测试和精密测量等量子技术的发展至关重要\cite{garttner2017measuring,mi2021information,choi2023preparing,li2023improving}。近期的研究揭示了量子多体系统中丰富的信息动力学现象,从混沌不可积哈密顿量导致的快速加扰\cite{lewis2019dynamics,swingle2018unscrambling},到多体局域化系统(MBL)中对数级缓慢的信息传播\cite{nandkishore2015many,abanin2019colloquium,lukin2019probing,schreiber2015observation,choi2016exploring,rispoli2019quantum,huang2017out,chen2017out,fan2017out,deng2017logarithmic,he2017characterizing,swingle2017slow,banuls2017dynamics}。然而,多体物理的广阔版图,以其多样的相互作用形式和涌现现象,预示着更多未被探索的信息加扰动力学机制。

在此背景下,动力学约束系统(Kinetically Constrained Systems, KCS)中的量子信息动力学成为了一个引人注目的前沿领域,例如里德堡原子阵列\cite{kumar2018sorting,sheng2022defect,graham2022multi,barnes2022assembly,kim2022rydberg,okuno2022high,chen2023continuous,liu2023realization,singh2023mid,eckner2023realizing,ma2023high,zhao2023floquet,bluvstein2024logical,gregory2024second,pause2024supercharged,shaw2024benchmarking}。里德堡原子间的强范德瓦尔斯相互作用导致了里德堡阻塞(Rydberg blockade)机制\cite{lukin2001dipole,urban2009observation_combined},从而限制了局域量子比特的翻转。这一核心物理机制可以近似地由 PXP 哈密顿量描述\cite{lesanovsky2012interacting,turner2018weak}:
\begin{equation}
  \hat{H}_\text{PXP} =
  \sum_i\hat{P}_i\hat\sigma^x_{i+1}\hat{P}_{i+2}.\label{Eq:PXP}
\end{equation}
此处,$\hat{P}_i = (1 - \hat\sigma^z_i)/2$ 是第 $i$ 个格点上自旋向下 $\ket{\downarrow}$ 态的投影算符,$\hat\sigma^{x,y,z}_i$ 是第 $i$ 个自旋的泡利矩阵。对这些系统的既往研究发现了量子多体疤痕(Quantum Many-Body Scars)现象\cite{bernien2017probing,turner2018weak,serbyn2021quantum,bluvstein2021controlling, desaules2024ergodicity}。此外,近期的理论研究\cite{yuan2022quantum}表明,动力学约束多体系统(如 PXP 模型描述的系统)可能蕴藏着尚未被发现的信息加扰机制,为探索超越多体疤痕范畴的独特时空量子信息动力学开辟了新前沿。

图~\ref{Fig:1}a 展示了该动力学约束系统中新颖信息传播动力学的直观图像。在 PXP 模型中,翻转反铁磁 N\'eel 态 $\ket{\mathbb{Z}_2} = \ket{\uparrow\downarrow\uparrow\downarrow\uparrow\cdots}$ 中的中心自旋编码了一比特的信息。这一局域扰动导致邻近自旋的旋转发生迟滞,这种迟滞随即以弹道形式传播,形成线性的光锥状波前。在该光锥内部,自旋恢复其受限旋转,但自旋态的可区分性(distinguishability)表现出周期性的消失与重现,这由其 $\hat\sigma^y$ 期望值的差异所标识(虚线圆圈)。这种行为与典型的混沌量子系统显著不同,在后者中,光锥内的自旋态由于加扰会迅速变得不可区分。这些动力学约束动力学可能导致持续的信息回流(information backflow)和量子混沌的非寻常破坏,暗示了探索独特信息加扰动力学的新途径\cite{yuan2022quantum}。

然而,在多体系统中对量子信息动力学进行实验探索面临着巨大挑战。诸如序时关联函数(OTOCs)和 Holevo 信息等关键探测工具,对态制备、多体演化控制以及原位(in-situ)态测量提出了极高的精度要求\cite{lewis2019dynamics,swingle2018unscrambling}。

本文在动力学约束的里德堡原子阵列中探索了新颖的量子信息动力学。通过开发并集成新的实验技术,本文克服了上述挑战,成功测量了 OTOC 和 Holevo 信息以探测信息加扰。值得注意的是,本文观察到了多体系统中量子信息的时空“塌缩与复苏”(collapse-and-revival)现象,揭示了独特的相干性质:局部编码的信息在扩散至远处格点后,尽管存在强多体相互作用,仍能周期性地保持可访问性。这一行为与典型的混沌系统及多体局域化系统均有显著差异,为量子加扰提供了新的见解,并凸显了里德堡原子阵列在动力学约束天然提供的动态量子保护下进行新型量子信息处理的潜力。

\section{实验模型与态制备}
\label{sec:model_setup}
本文的量子模拟器采用由光镊阵列囚禁的最多 25 个 $^{87}$Rb 原子组成的线性链,如图~\ref{Fig:1}b, c 所示。量子比特态 $\ket{\downarrow}$ 和 $\ket{\uparrow}$ 分别编码在原子基态 $\ket{g} = \ket{5S_{1/2}, F=2, m_F=2}$ 和里德堡态 $\ket{r} = \ket{68D_{5/2}, m_J=5/2}$ 上。系统的哈密顿量由下式给出:
\begin{equation}
\label{Rydberg Hamiltonian}
  \hat{H}_\text{R}=
  \frac{\Omega}{2}\sum_i\hat\sigma^x_i-
  \Delta\sum_i\hat{n}_i+
  \sum_{i<j}V_{ij}\hat{n}_i\hat{n}_j,
\end{equation}
其中 $\Omega$ 是 $\ket{\uparrow}$ 和 $\ket{\downarrow}$ 之间相干驱动的拉比频率,$\Delta$ 是驱动激光相对于基态-里德堡态跃迁的失谐量,$\hat{n}_i = \ket{\uparrow_i}\bra{\uparrow_i}$ 是第 $i$ 个格点上里德堡态的投影算符,$V_{ij} \propto 1/R^6_{ij}$ 表示距离为 $R_{ij}$ 的第 $i$ 和第 $j$ 个格点处里德堡原子间的范德瓦尔斯相互作用。

在满足 $V_{i,i+2} \ll \Omega \ll V_{i,i+1}$ 的参数区间内,里德堡阻塞效应禁止相邻量子比特同时处于 $\ket{\uparrow}$ 态(例如 $\ket{\cdots \uparrow_i \uparrow_{i+1} \cdots}$),使得系统可以被 PXP 模型很好地近似。

为了实现该模型,实验中使用了波长为 780 nm 和 480 nm 的全局激光束,通过双光子拉曼过程驱动基态-里德堡态跃迁。单格点的量子比特操作则通过 795 nm 和 480 nm 的寻址激光束实现(如图~\ref{Fig:1}b 所示)。相关原子能级如图~\ref{Fig:1}c 插图所示,中间态包括 $\ket{e} = \ket{5P_{3/2}, F=3, m_F=3}$ 和 $\ket{e'} = \ket{5P_{1/2}}$。

高保真度的态制备是后续所有实验成功的关键。本文通过结合全局里德堡激发激光与位置选择性的 795 nm 寻址激光来制备 $\ket{\mathbb{Z}_2}$ 态。寻址激光使被寻址原子的基态-里德堡态跃迁发生失谐,从而创建出可激发原子与不可激发原子交替的图案(图~\ref{Fig:2}a)。如图~\ref{Fig:2}b, c 所示,该方法实现了很高的 $\ket{\mathbb{Z}_2}$ 态制备保真度:对于中心 13 个量子比特为 70(1)\%,对于整个 25 量子比特链为 49(3)\%。在修正探测误差后,13 量子比特和 25 量子比特的保真度分别提升至 78(1)\% 和 60(3)\%。在如此大的希尔伯特空间($2^{25}$ 维)中实现如此高的制备保真度,对于精确探测动力学约束的多体态演化及量子信息动力学至关重要。

接下来,我们研究了前文所述的动力学约束自旋动力学,重点关注自旋构型 $\ket{\mathbb{Z}_2}$ 和 $\hat\sigma^x_c \ket{\mathbb{Z}_2}$ 在 PXP 哈密顿量下的演化。对于 $\ket{\mathbb{Z}_2}$ 态(图~\ref{Fig:2}d),所有自旋在 PXP 约束下相干旋转,形成均匀的波前。这种同步演化一直持续,直到边界效应逐渐从外边缘向内改变自旋旋转。

相比之下,对于 $\hat\sigma^x_c \ket{\mathbb{Z}_2}$ 态(图~\ref{Fig:2}f),我们观察到了清晰的线性光锥结构\cite{surace2020lattice,frerot2018multispeed,Chen2019Finite,kuwahara2020strictly}。光锥外的自旋不受翻转的中心自旋影响,表现出与 $\ket{\mathbb{Z}_2}$ 态相似的动力学。而在光锥内部,被翻转的中心自旋导致邻近自旋的演化发生迟滞(retardation),产生了一个独特的演化波前。这一弧形弯曲的波前向外传播,由迟滞的自旋旋转驱动,清晰地展示了 PXP 模型中的受限自旋动力学。实验数据与数值模拟结果(图~\ref{Fig:2}e, g)之间的极佳吻合,证实了我们的里德堡量子模拟器在捕捉动力学约束系统行为方面的有效性,为进一步探索量子信息动力学铺平了道路。

\section{序时关联函数 (OTOC) 动力学}
\label{sec:otoc}

序时关联函数(Out-of-Time-Ordered Correlator, OTOC)是探测多体系统中量子信息加扰的有力工具\cite{lewis2019dynamics,swingle2018unscrambling}。然而,由于巨大的实验挑战,OTOC 测量此前仅在少数几个最先进的物理平台上得到演示\cite{mi2021information,blok2021quantum,braumuller2022probing,zhao2022probing,wang2022information,landsman2019verified,garttner2017measuring,joshi2020quantum,green2022experimental,li2017measuring,wei2018exploring,niknam2020sensitivity,chen2020detecting,pegahan2021energy,li2023improving}。在里德堡原子阵列中,本文首次通过实验利用 OTOC 研究了动力学约束下的信息加扰动力学。OTOC 定义如下,它量化了局域扰动 $\hat{W}$(称为蝴蝶算符)的算符增长如何在格点间传播:
\begin{equation}
\hat{F}_{ij}(t) = \bra{\psi}\hat{W}_i^\dagger(t)\hat{V}_j^\dagger \hat{W}_i(t)\hat{V}_j\ket{\psi}.
\end{equation}
本文的研究聚焦于 ZZ-OTOC,选取局域蝴蝶算符 $\hat{W}_i = \hat\sigma^z_c$ 和测量算符 $\hat{V}_j = \hat\sigma^z_j$,以避免破坏动力学约束的自旋旋转。$\hat{W}_i(t) = e^{i\hat{H}t}\hat{W}_i e^{-i\hat{H}t}$ 表示 $\hat{W}_i$ 的海森堡演化,$\ket{\psi}$ 为初态。对应于中心量子比特上的 $\hat\sigma^z_c$ 门(即 $\pi$ 相移)的局域扰动,在实验上是通过 795 nm 寻址激光的光频移实现的。图~\ref{Fig:3}a 展示了测量 OTOC $\hat{F}_{ij}(t)$ 的五步协议:(1)态制备;(2)正向演化;(3)局域扰动;(4)反向(时间反演)演化;(5)计算基矢测量。

本文采用了两种不同的初态来研究 OTOC 动力学,包括反铁磁 $\mathbb{Z}_2$ 有序的 N\'eel 态 $\ket{\mathbb{Z}_2} = \ket{\uparrow\downarrow\uparrow\downarrow\uparrow\cdots}$(与疤痕本征态子空间有较大重叠\cite{turner2018weak})和平凡直积态 $\ket{\mathbf{0}} = \ket{\downarrow \downarrow\downarrow\downarrow\downarrow\cdots}$(位于普通热化本征态子空间)。为了减小边界和有限尺寸效应,我们特意制备了 25 个量子比特的 $\ket{\mathbb{Z}_2}$(或 $\ket{\mathbf{0}}$)态,并在中心 13 个量子比特上进行 OTOC 实验。这种配置在演化过程中屏蔽了边界效应,能更好地近似体(bulk)PXP 动力学。此外,我们的高 $\ket{\mathbb{Z}_2}$ 态制备保真度对于准确探测 OTOC 动力学至关重要。低制备保真度会引入大量错误态,模糊预期的 OTOC 图案。

在成功制备初态后,我们要解决 OTOC 测量中另一个关键且极具挑战性的步骤——在多体系统中实现 $-\hat{H}$ 下的时间反演演化。虽然反转单粒子哈密顿量相对容易,但反转多体相互作用项(如在吸引和排斥相互作用间切换)在理论和实验上都面临巨大困难。相比于谷歌 Sycamore 量子计算机\cite{mi2021information}利用数字模拟方法实现时间反演,本文的方法利用了 PXP 模型的粒子-空穴对称性(particle-hole symmetry),通过对所有量子比特施加全局 $\hat\sigma^z$ 门来实现逆哈密顿量:$(\prod_i \hat\sigma_i^z)\hat{H}_\text{PXP}(\prod_i \hat\sigma_i^z) = -\hat{H}_\text{PXP}$。在实验上,这是通过使用远失谐微波场对里德堡态诱导 $\pi$ 相移来实现的。为了弥合全里德堡哈密顿量 $\hat{H}_\text{R}$ 与理想 PXP 模型 $\hat{H}_{\text{PXP}}$ 之间的差距,我们进行了详细的基准模拟并优化了实验参数(例如引入小量全局失谐以补偿次近邻相互作用)。图~\ref{Fig:3}c 展示了实验测量的 $\ket{\mathbb{Z}_2}$ 态正向-反向演化动力学。这种新颖的数字-模拟结合(digital-analogue)协议为里德堡原子阵列中相干反转多体演化和探测信息加扰提供了一种强有力的技术手段。

在实现了高保真度态制备和时间反演后,我们执行了完整的 OTOC 测量协议(图~\ref{Fig:3}d, e)。图~\ref{Fig:3}f 和 g 分别展示了 $\ket{\mathbb{Z}_2}$ 和 $\ket{\mathbf{0}}$ 初态下 ZZ-OTOC 的时空演化。测量到的 OTOC 值的整体衰减主要源于演化过程中的原子丢失和退相干。为了区分感兴趣的加扰动力学与这些无关的衰减机制\cite{landsman2019verified},我们额外测量了 IZ-OTOC(蝴蝶算符为单位算符 $I$),并利用类似于前人工作\cite{mi2021information}的误差缓解技术对原始 ZZ-OTOC 数据进行了修正(图~\ref{Fig:3}j, k)。

OTOC 演化动力学揭示了 $\ket{\mathbf{0}}$ 和 $\ket{\mathbb{Z}_2}$ 初态截然不同的行为。对于 $\ket{\mathbf{0}}$ 态,OTOC 在线性光锥内迅速衰减,未见明显的复苏。这种快速的信息加扰符合混沌量子系统中高能初态的理论预期。与此形成鲜明对比的是,$\ket{\mathbb{Z}_2}$ 态表现出较慢的蝴蝶速度(butterfly velocity),并在光锥内显现出显著的量子信息时空**“塌缩与复苏”(collapse-and-revival)**模式。

波函数的塌缩与复苏此前主要在单粒子系统(如 Jaynes-Cummings 模型)中被观察到\cite{eberly1980periodic,rempe1987observation},反映了量子相干性的周期性扩散与重聚。与多体系统中相互作用导致量子信息在指数级大的希尔伯特空间中迅速加扰的传统预期不同,我们的观察揭示了在光锥内部量子相干性的惊人保持和重聚焦。这些不寻常的加扰动力学表明,动力学约束系统允许量子信息的弹道传播和周期性恢复,这与典型的混沌和多体局域化系统均有显著不同。

\section{Holevo 信息与非马尔可夫动力学}
\label{sec:holevo}

除了 OTOC,本文还引入了 Holevo 信息来研究系统中的量子信息动力学。虽然 ZZ-OTOC 成功揭示了“塌缩与复苏”现象,但 $\hat\sigma^z$ 扰动在特定条件下可能失效。如图~\ref{Fig:3}j, l 所示,在 $0.6\text{--}0.7\,\si{\micro\second}$ 和 $1.2\text{--}1.3\,\si{\micro\second}$ 期间,所有量子比特的 OTOC 值接近 1。这是因为此时受扰动的量子比特接近纯态 $\ket{\uparrow}$ 或 $\ket{\downarrow}$,使得蝴蝶算符 $\hat\sigma^z$ 无法产生有效扰动。

为了获得动力学约束系统中信息传播更连续的图像,我们转向 Holevo 信息。Holevo 信息最初被提出用于界定两个分离主体间可访问信息的上限\cite{holevo1973bounds},在此它被用于量化从微小差异的初始条件演化而来的量子态的局域可区分性。通过将子系统上的约化哈密顿量演化视为量子通道,利用 Holevo 信息可以补充 OTOC 测量,特别是在蝴蝶算符扰动失效的情况下。这种基于量子态层析(Quantum State Tomography, QST)的方法,为我们提供了更全面的量子信息动力学理解。

我们的实验探索了两组不同初态下的哈密顿量演化:$\ket{\mathbb{Z}_2}$ 和 $\hat\sigma^x_{c}\ket{\mathbb{Z}_2}$。$\hat\sigma^x_{c}$ 算符作用于一维链的中心量子比特,在初态中编码了一比特的局域信息(如图~\ref{Fig:1}a 所示)。在随后的演化中,我们测量不同格点 $j$ 处的 Holevo 信息 $\mathbb{X}_j(t)$,它量化了编码在中心量子比特的信息有多少可以通过在远处格点的局域测量被提取:
\begin{equation}
\mathbb{X}_j(t) = S\left(\frac{\hat\rho_j(t) + \hat\rho'_j(t)}{2}\right) - \frac{S(\hat\rho_j(t)) + S(\hat\rho'_j(t))}{2},
\end{equation}
其中 $\hat\rho_j(t)$ 和 $\hat\rho'_j(t)$ 分别是初态为 $\ket{\mathbb{Z}_2}$ 和 $\hat\sigma^x_{c}\ket{\mathbb{Z}_2}$ 时第 $j$ 个自旋演化后的约化密度矩阵,$S(\hat\rho) = - \mathrm{Tr}(\hat\rho \log \hat\rho)$ 是冯·诺依曼熵。

在 PXP 动力学约束下进行全量子态层析极具挑战性。虽然对角元可以直接从里德堡布居数测量中提取,但获取非对角元需要对目标量子比特进行自旋旋转,这一过程会受到邻近量子比特相互作用的干扰。为了旋转目标量子比特,必须确保其最近邻和次近邻量子比特既不处于里德堡态,也不与基态-里德堡态跃迁共振。为此,本文开发了一种新技术:利用 480 nm 寻址激光作用于目标量子比特周围的四个邻近格点(如图~\ref{Fig:4}a 所示)。这些 480 nm 激光首先通过中间态 $\ket{e}$ 的自发辐射将邻近格点的里德堡布居转移回基态,然后构建电磁诱导透明(EIT)条件,阻止这些邻近格点被全局激发激光激发。最后,我们施加全局激发激光,仅对被“隔离”的目标量子比特实施态层析所需的自旋旋转。

图~\ref{Fig:4}b(c) 展示了实验测量(数值模拟)的 Holevo 信息时空演化,两者吻合良好。我们观察到了清晰的线性光锥结构和独特的 $\mathbb{X}_j(t)$ “塌缩与复苏”模式,这与 OTOC 结果有所不同。值得注意的是,在 ZZ-OTOC 值接近 1(即扰动失效)的时间段内,我们仍然观察到非均匀的 Holevo 信息分布。这证明了 Holevo 信息能够连续跟踪量子信息动力学。

此外,本文的研究结果揭示了一种独特的信息输运行为:最初编码在某一点的局域信息,经过多体相互作用哈密顿量演化后,可以在初始位置和其他远处格点被恢复。这些动力学展示了持续的**信息回流(Information Backflow)**——即信息从环境(周围量子比特)流回系统(最初编码信息的量子比特),这是**非马尔可夫(Non-Markovian)**开放量子动力学的标志\cite{breuer2016colloquium}。这种非马尔可夫性可以通过 Holevo 信息和迹距离(trace distance)方法进行量化。观察到的非马尔可夫量子动力学强调了系统对局域编码信息的量子记忆效应及其跨越长距离传输信息的能力。

\section{讨论}
本文在强相互作用里德堡原子阵列中观察到了量子信息的“塌缩与复苏”模式。尽管这一现象通常在单粒子场景(如 Jaynes-Cummings 模型)中被观察到,但它在我们的多体相互作用系统中显现出来,提供了一幅在动力学约束下量子信息传播的全新图像。我们特别指出,这种新颖的信息加扰动力学不同于通用的混沌或多体局域化系统,并且是首次在实验上被观察到。

观察到的现象背后的机制可以归因于里德堡阻塞效应引起的动力学约束自旋旋转。更具体地说,由于 Holevo 信息动力学中的自旋翻转操作,量子信息的周期性重组源于涉及疤痕子空间和热化本征态库的动力学。这使得我们观察到的“塌缩与复苏”动力学有别于此前报道的纯疤痕态波函数的振荡\cite{bernien2017probing,turner2018weak}。

本文开发的实验技术对于精确测量受限自旋动力学、OTOC 和 Holevo 信息至关重要。特别是,我们演示了一种数字-模拟结合的方法来有效地反转里德堡原子系统中的多体演化。我们的平台具备的多重跃迁、位点选择性寻址能力,使我们能够实现高保真度的多体态制备,并在存在旋转约束的情况下执行全量子态层析。这些技术的紧密集成为里德堡原子量子模拟丰富了工具箱。

展望未来,我们的工作开启了一系列激动人心的研究方向。例如,可以研究由多个蝴蝶算符扰动引起的传播和干涉动力学。虽然目前观察到的“塌缩与复苏”效应的幅度和持续时间受到相干性的限制,但前人研究表明,如 Floquet 工程等技术可以显著延长相干时间并修改里德堡 PXP 系统中的相互作用\cite{bluvstein2021controlling,koyluouglu2024floquet}. 基于此,结合更多的单格点操作引入精确的空间调制,可能实现对受动力学约束动态保护的量子信息传播的可控引导(steering)。这些进展有望在近期的量子技术中获得实际应用,包括鲁棒的量子存储器、新型量子态传输协议以及多体相互作用系统中的量子精密测量。

\section{方法}
\label{sec:methods}

\textbf{无缺陷原子阵列的制备}:
在实验中,原子从磁光阱(MOT)中加载,并被空间光调制器(SLM)衍射生成的 808 nm 光镊捕获。通过测量和反馈过程消除原子链中的缺陷,能够快速创建全填充的原子阵列。

\textbf{实验参数}:
一旦制备好无缺陷阵列,所有原子被光泵浦到基态。原子间距约为 \SI{7}{\micro m},导致最近邻和次近邻相互作用分别为 $V_{i,i+1} = 2\pi \times \SI{7.3}{MHz}$ 和 $V_{i,i+2} = 2\pi \times \SI{0.11}{MHz}$。为了在 $V_{i,i+1} / V_{i,i+2} \approx 64$ 的固定比率下近似 PXP 模型,我们将基态-里德堡态驱动拉比频率 $\Omega$ 设置为 $V_{i,i+1}/6 = 2\pi \times \SI{1.21(1)}{\MHz}$,以在两种相互作用强度间取得平衡。此外,引入了小量失谐 $\Delta =2\pi\times\SI{0.22(1)}{\MHz}$ 以补偿残留的次近邻相互作用。

\textbf{数值方法}:
文中展示的数值模拟对于 13 个原子以下的链采用精确对角化(Exact Diagonalization),对于 25 个原子的链采用矩阵乘积算符(MPO)方法。模拟使用了全里德堡哈密顿量,并根据系统特性考虑了实验噪声和缺陷(详见补充材料)。


\end{document}
