% chapters/06-quantum_scar.tex
\documentclass[../main.tex]{subfiles}
\begin{document}
% 章节内容...
\chapter{里德堡原子阵列中的量子多体疤痕与信息输运}


摘要: 封闭量子多体系统的热化动力学通常遵循本征态热化假说(ETH),初始的局部量子信息会迅速扩散到整个系统而变得不可恢复。然而,近期发现的“量子多体疤痕”(Quantum Many-Body Scars)现象挑战了这一范式,表明在某些强相互作用系统中存在弱各态历经性破缺,允许系统在长时间内保持相干性。 本章首先介绍基于里德堡原子阵列实现的 PXP 模型及其动力学约束机制。我们详细阐述了如何在实验上高保真度地制备与疤痕态子空间重叠极大的反铁磁 $\mathbb{Z}_2$ 初态。为了深入探究该系统中的量子信息动力学,我们引入了 Holevo 信息(Holevo Information)作为探测工具。与传统的布居数测量不同,Holevo 信息能够量化量子通道中可传输信息的上限。实验结果显示,在疤痕态演化过程中,Holevo 信息表现出显著的非马尔可夫(Non-Markovian)特征,即原本耗散到环境中的信息会周期性地回流到局部子系统中。这一发现从量子信息输运的角度揭示了动力学约束对多体系统热化的抑制作用,为后续章节研究更复杂的信息加扰动力学奠定了基础。

\end{document}
