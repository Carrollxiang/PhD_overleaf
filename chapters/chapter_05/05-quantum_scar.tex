\documentclass[../main.tex]{subfiles}
\begin{document}

\chapter{里德堡原子阵列中的量子多体疤痕}
\label{ch5:sec:scar}

\section{引言}
遍历量子多体系统通常表现为快速热化,其长期行为可由本征态热化假说(eigenstate thermalization hypothesis, ETH)理解\cite{deutsch1991quantum,srednicki1994chaos,rigol2008thermalization,deutsch2018eigenstate,lewis2019dynamics}。在这一图景下,局域可观测量在有限时间内趋近于热平衡值,初态信息仅以少数守恒量的形式保留。然而,近年的理论与实验研究表明,某些相互作用多体系统能够在远长于微观时间尺度的区间内维持显著的相干振荡与弱热化行为,从而体现各态历经性破缺的动力学结构\cite{bernien2017probing,turner2018weak,choi2019emergent,ho2019periodic,bluvstein2021controlling,turner2021correspondence,serbyn2021quantum,desaules2024ergodicity}。量子多体疤痕(quantum many-body scars, QMBS)为其中一类代表机制:在能谱中存在一组与热本征态共存的非热本征态,使得特定低熵初态(例如反铁磁尼尔态)与这些本征态具有异常大的重叠,从而产生大振幅、缓慢衰减的复苏振荡。

可编程里德堡原子阵列为研究 QMBS 提供了可控且可扩展的平台\cite{kumar2018sorting,sheng2022defect,graham2022multi,barnes2022assembly,kim2022rydberg,okuno2022high,chen2023continuous,liu2023realization,singh2023mid,eckner2023realizing,ma2023high,zhao2023floquet,bluvstein2024logical,gregory2024second,pause2024supercharged,shaw2024benchmarking}。里德堡原子之间的强范德瓦尔斯相互作用诱导里德堡阻塞机制\cite{lukin2001dipole,urban2009observation_combined},抑制相邻格点的同时激发,从而对局域自旋翻转施加动力学约束。在适当参数区域,该受限动力学可由 PXP 模型近似描述\cite{lesanovsky2012interacting,turner2018weak},并天然支持从 $\ket{\mathbb{Z}_2}$ 初态出发的疤痕复苏动力学。

本章目标是建立从里德堡原子阵列到 PXP 模型的理论对应关系,并在实验上通过参数优化与局域寻址制备高保真度的 $\ket{\mathbb{Z}_2}$ 疤痕态,随后用局域可观测量的复苏与寿命表征动力学受限系统中的弱热化行为。第六章将在此实验平台基础上进一步研究受限多体动力学中的量子信息问题。

\todo{需要深入拓展 QMBS 理论:详细论述本征态热化假说 (ETH) 的破缺,引入量子多体疤痕的严格定义;解释为何反铁磁尼尔态$\ket{\mathbb{Z}_2}$会与特定的非热本征态(Scar states)产生异常大的重叠(可加入能谱与纠缠熵特征的文献综述)}

\section{动力学受限与 PXP 模型}
\subsection{里德堡哈密顿量与阻塞约束}
本文考虑由二维光镊俘获并重排得到的一维 $^{87}$Rb 原子链(实验平台与通用操控见第~\ref{ch3:sec:setup}、~\ref{ch4:sec:state_manipulation} 章)。每个原子编码为二能级量子比特:基态 $\ket{\downarrow}$ 与里德堡态 $\ket{\uparrow}$。系统由里德堡哈密顿量支配:
\begin{equation}
\hat{H}_\text{R}=
\frac{\Omega}{2}\sum_i\hat{\sigma}^x_i-
\Delta\sum_i\hat{n}_i+
\sum_{i<j}V_{ij}\hat{n}_i\hat{n}_j,
\label{ch5:eq:rydberg_hamiltonian}
\end{equation}
其中 $\Omega$ 为全局驱动的拉比频率,$\Delta$ 为驱动光相对于基态-里德堡态跃迁的失谐,$\hat{n}_i=\ket{\uparrow_i}\bra{\uparrow_i}$ 为第 $i$ 个格点上的里德堡布居算符,$V_{ij}\sim C_6/R_{ij}^6$ 表示两个里德堡原子间的范德瓦尔斯相互作用。当原子间距足够近时,强烈的最近邻相互作用($V_{i, i+1}\gg\Omega$),使得相邻激发组态 $\ket{\cdots\uparrow_i\uparrow_{i+1}\cdots}$ 在能量上被强烈移动,进而相邻原子很难同时被激发到里德堡态。

\subsection{有效 PXP 模型与对称性}
为了提炼系统在里德堡阻塞下的核心多体动力学,我们考虑理想极限条件:共振驱动($\Delta=0$)、无限大最近邻相互作用($V_{i, i+1}\rightarrow\infty$),并忽略次近邻及更长程的相互作用尾巴。在此极限下,我们可以利用施里弗—沃尔夫变换(Schrieffer-Wolff transformation)将系统投影到不包含相邻激发的受限子空间中,从而得到低能有效 PXP 哈密顿量:
\begin{equation}
\hat{H}_\text{PXP}=\sum_i \hat{P}_i\hat{\sigma}^x_{i+1}\hat{P}_{i+2},
\label{ch5:eq:pxp}
\end{equation}
其中 $\hat{P}_i=(1-\hat{\sigma}^z_i)/2$ 为格点 $i$ 上向基态 $\ket{\downarrow}$ 的投影算符。局域的三体作用项 $\hat{P}_i\hat{\sigma}^x_{i+1}\hat{P}_{i+2}$ 直观地体现动力学约束:仅当相邻原子均处于 $\ket{\downarrow}$ 时,格点 $i+1$ 才允许翻转。

这一物理约束从根本上改变了系统的希尔伯特空间结构。我们可以定义受限子空间的全局投影算符为:
\begin{equation}
\mathcal{P}=\prod_j\left(1-\hat{n}_j\hat{n}_{j+1}\right)
\label{ch5:eq:projector}
\end{equation}
在此受限子空间内,去除了包含 ∣↑↑⟩ 的微观态,导致有效希尔伯特空间的维数不再按$2^N$增长,而是遵循斐波那契数列增长,其缩放关系近似为$\phi^N$(其中 $\phi=(1+\sqrt{5})/2$,N 为系统尺寸)。这种维度的降低是动力学受限系统的典型特征。

\subsubsection{希尔伯特空间对称性与非热态}

除了特殊的希尔伯特空间结构,PXP 模型还表现出三种深刻影响其多体动力学的对称性:
\begin{itemize}
    \item \textbf{离散空间反演对称性} $\mathcal{I}$:满足映射 j→N−j+1,反映了一维开边界链(或周期边界下)的空间反演不变性。
    \item \textbf{平移对称性} $\mathcal{T}$:在应用周期性边界条件时严格成立。
    \item \textbf{粒子-空穴对称性} $\mathcal{C}$:其算符定义为 $\mathcal{C}=\prod_j \sigma_j^z$。该对称性导致 PXP 哈密顿量满足反交换关系$\mathcal{C}\hat{H}_\text{PXP}\mathcal{C}=-\hat{H}_\text{PXP}$,这意味着 PXP 模型的能谱必然严格关于零能量对称。
\end{itemize}

这种受限子空间的特殊代数结构与上述对称性共同作用,打破了本征态热化假说(ETH)\todo{需要增加ETH引用}。在 PXP 模型的能谱中,嵌入了一系列特殊的非热本征态(量子多体疤痕态)。这使得当系统从特定的低熵初态(例如$\ket{\mathbb{Z}_2}=\ket{\uparrow\downarrow\uparrow\downarrow...}$ )出发时,系统不会迅速热化,而是展现出显著且缓慢衰减的复苏振荡行为\cite{bernien2017probing,turner2018weak,serbyn2021quantum,bluvstein2021controlling,desaules2024ergodicity}。
 
在本章后续实验中,本文以 $\ket{\mathbb{Z}_2}$ 初态为核心研究对象,并通过局域可观测量的时间演化来表征疤痕复苏与弱热化行为。
\todo[inline, color=blue!20]{对于“为何 $\ket{\mathbb{Z}_2}$ 对疤痕子空间具有异常大重叠”这一更深入的本征态结构解释,可在此处补充简要谱分解或引用关键文献并给出物理图像。}

\section{数值模拟方法}


为了在理论上精确还原并验证里德堡原子阵列中的多体动力学,本文针对不同系统尺寸和物理需求,分别自主开发了基于 GPU 加速的精确对角化(ED)程序,以及基于张量网络的含时变分原理(TDVP)算法。同时,为了真实反映实验环境,本文在数值模拟中引入了基于蒙特卡罗采样的含噪哈密顿量演化模型。

\subsection{基于 GPU 加速的全空间精确对角化与时间演化}

对于系统尺寸 $N \leq 13$ 的较短原子链,由于其全希尔伯特空间维数(最大为 $2^{13}=8192$ 维)尚在现代计算硬件的处理能力范围内,本文采用精确对角化与精确时间演化算符来求解薛定谔方程。为了克服常规 CPU 数值库(如 QuTiP)在处理多体张量积与常微分方程求解时效率低下的问题,本文利用 PyTorch 框架自主编写了底层演化程序,将哈密顿量的克罗内克积(Kronecker product)张量构建与矩阵指数运算全部迁移至 GPU(CUDA)上执行,实现了计算量级的加速。

对于包含复杂脉冲序列、多段演化或局部微扰的多体动力学过程,该数值框架具有极高的通用性与灵活性。在实际的量子模拟中,系统往往需要经历参数切换的演化路径(例如正向与反向时间演化、插入自由演化的 Ramsey 间隙或施加局域单比特门)。针对此类复杂的演化路径,本框架将全局时间演化算符一般化为各分段演化算符的有序连乘: \begin{equation} \hat{U}_\text{total} = \prod_k \exp\left[-i \tau_k \hat{H}_k\right] \end{equation} 其中$\hat{H}_k$与 $\tau_k$分别代表第 k 段演化的哈密顿量与持续时间。通过在 GPU 上直接高效地求解上述多体矩阵指数(Matrix Exponential),程序能够无截断误差地获取系统在任意时刻的精确多体态 ∣Ψ(t)⟩。基于该精确态,通过张量收缩与迹运算(Trace),本框架不仅能够重构任意单格点或多格点的约化密度矩阵(Reduced density matrix),还能测量任意形式的物理量——从基础的局域布居数$\hat{n}_i$、自旋期望值,到复杂的空间/时间关联函数以及量子信息熵。

这一自主开发的全空间演化框架不仅为本章探讨的量子多体疤痕与弱热化动力学提供了理论验证,更为后续章节研究复杂的量子信息动力学(如信息加扰、全状态层面的量子信息度量等)奠定了数值基础,同时也为大尺寸张量网络算法(TeNPy)提供了绝对精度的基准参考。

\subsection{基于双点含时变分原理(Two-site TDVP)的张量网络模拟}

当系统尺寸扩展至 $N=25$ 时,全空间维数激增至 $3.3 \times 10^7$,精确对角化方法不再适用。为了在保持数值精度的前提下突破维度灾难,本文利用 TeNPy 库实现了基于矩阵乘积态(MPS)和矩阵乘积算符(MPO)的张量网络模拟。

对于包含长程范德瓦尔斯相互作用($V_{ij} \sim 1/R_{ij}^6$)的里德堡哈密顿量,传统的 Trotter-Suzuki 分解(如 TEBD 算法)会产生显著的 Trotter 误差且难以高效处理次近邻项。因此,本文采用了双点含时变分原理(Two-site TDVP)算法对多体态进行时间演化。在模拟中,最大时间步长设定为 $\delta t = \SI{0.05}{\us}$。为了在纠缠熵随时间增长时准确描述系统,算法在每一步演化中动态更新张量网络,并采用奇异值分解(SVD)对纠缠谱进行截断。本文设定最大键维数(Maximum bond dimension)为 $\chi_{\max} = 100$,奇异值截断阈值(Truncation tolerance)严格控制在 $10^{-8}$ 量级。由于我们重点关注的是受限子空间内的非热疤痕态动力学,其纠缠熵增长极其缓慢(遵循对数增长或面积律),该截断参数足以确保在整个实验时间尺度($t_{\max} \approx \SI{4}{\us}$)内保持极高的保真度。

\subsection{含噪哈密顿量的蒙特卡罗采样模拟}

真实的里德堡原子阵列不可避免地受到环境与操控激光的噪声干扰。为了使数值模拟精准复现实验结果,本文构建了含噪时变哈密顿量,并通过蒙特卡罗方法(Monte Carlo method)对系统轨迹进行随机采样。综合考虑的误差源及其在程序中的建模如下:
\begin{itemize}
    \item \textbf{拉比频率与激光相位噪声}:由于激光功率波动与伺服回路(Servo bump)引入的高频相位噪声,驱动光拉比频率建模为 $\Omega(t) = \Omega_0 (1 + \delta_\Omega) e^{-i \delta_\phi}$,其中振幅相对涨落 $\delta_\Omega$ 满足标准差为 $4\%$ 的高斯分布,相位涨落 $\delta_\phi$ 满足标准差为 $0.07\pi$ 的高斯分布。
    \item \textbf{多普勒频移与失谐波动}:有限原子温度($\sim 10\ \mu\text{K}$)导致的多普勒频移被建模为各格点独立的全局失谐涨落 $\delta_\Delta$,满足标准差为 $4\%$ 的高斯分布。
    \item \textbf{原子位置热涨落(相互作用噪声)}:由于光镊中原子的零点运动与热涨落,原子间距 $R$ 存在不确定性。由于里德堡相互作用 $V \propto 1/R^6$,微小的位置偏差会被急剧放大。本文将最近邻相互作用强度建模为 $V = V_0 (1 + \delta_V)$,其中 $\delta_V$ 遵循标准差高达 $20\%$的高斯分布。
\end{itemize}


在含噪模拟中,每次演化均从上述高斯分布中独立抽取一组随机参数生成具体的噪声演化轨迹。本文通常执行 200 次以上的蒙特卡罗轨迹(Trajectories)并对最终的格点密度矩阵取系综平均。这种全方位的噪声建模方法,为后续讨论实验误差分解以及局域物理量的耗散机制提供了可靠的理论依据。

\section{实验平台操控时序与 \SI{795}{\nm} 局域寻址技术}

在本章及后续章节的实验中,冷原子的俘获、偏振梯度冷却(PGC)、基于二维声光偏转器(AOD)的光镊无缺陷重排以及至初态 $\ket{\downarrow}$ 的光抽运(Optical pumping)等基础操控步骤均与前文(第 3、4 章)详述的流程保持一致。本节将重点阐述针对里德堡多体动力学研究所特有的实验操控时序,以及用于复杂初态制备与局域微扰的核心技术——\SI{795}{\nm} 局域寻址技术。

\subsection{里德堡操控与探测的实验时序}
为了在高保真度的前提下观测里德堡原子阵列的多体动力学,必须严格控制实验光操作期间的环境噪声,特别是光镊诱导的微分交流斯塔克频移(Differential AC Stark shift)。本实验的里德堡操控与探测时序主要包含以下关键步骤:
光镊关闭与自由演化:在完成原子重排与光抽运后,原子被冷却至约 $\SI{10}{\micro\kelvin}$。在施加任何里德堡激发或寻址光脉冲之前,系统会短暂关闭 \SI{808}{\nm} 光镊。这一操作彻底消除了强偶极囚禁光对基态 $\ket{\downarrow}$ 和里德堡态 $\ket{\uparrow}$ 造成的不均匀能级频移。
里德堡实验光操作:在光镊关闭的空窗期内,系统施加 795 nm 寻址光以及 780 nm / 480 nm 的全局拉曼(Raman)光,执行目标态制备与哈密顿量演化。整个自由演化窗口的时间被严格控制在 $\sim\SI{10}{\micro\second}$ 以内(即释放-俘获时间,Release and recapture time),以确保原子在热膨胀后仍能以约 99\% 的概率被重新俘获。
光镊重新开启与状态转化:里德堡动力学演化结束后,光镊在 \SI{1}{\us} 内被迅速重新开启,并将势阱深度提升至 $\SI{1.3}{\milli\kelvin}$。此时,处于基态 $\ket{\downarrow}$ 的原子被重新俘获,而处于里德堡态 $\ket{\uparrow}$ 的原子则受到光镊的抗势场(Anti-trapping)作用而被迅速排斥出囚禁区域。
原子布居探测:为了进一步确保处于里德堡态的原子不被误判为基态,系统紧接着施加一个高功率的 2.4 GHz 微波脉冲,将可能残留的里德堡态原子耗散掉。最终,利用相机采集基态原子的荧光信号,即可高保真度地重构出阵列中每个格点上的里德堡态布居数。

\subsection{795 nm 局域寻址技术与光路对准}
为了在等间距的原子阵列中制备打破平移对称性的反铁磁$\ket{\mathbb{Z}_2}$态,以及在后续加扰实验中施加局域微扰,本实验开发了一套基于 795 nm 激光的局域寻址(Local addressing)系统。

该寻址光束通过空间光调制器(Spatial Light Modulator, SLM)生成定制的相位全息图,随后经由数值孔径 NA=0.4 的定制物镜聚焦至目标原子所在位置。在物理参数的选择上,795 nm 寻址光相对于 $^{87}\mathrm{Rb}$的 D1 线共振跃迁($\ket{5S_{1/2}}\rightarrow\ket{5P_{1/2}}$)蓝失谐(Blue-detuned) $2\pi \times \SI{15}{\GHz}$。在我们的实验功率下,该光束可在目标原子的基态上诱导约 $2\pi \times \SI{12.2(3)}{\MHz}$ 的交流斯塔克频移(AC Stark shift)。得益于较大的大失谐量,寻址光引入的自发辐射散射率极低(约 $2\pi \times \SI{5}{\kHz}$),且对相邻未被寻址原子的串扰(Crosstalk)被严格抑制在 $2\pi \times \SI{1}{\kHz}$ 以下。

在实际实验调试中,寻址光束与 \SI{1}{\micro\meter} 级别目标原子的精确空间对齐是实现高保真度寻址的核心难点。由于 795 nm 寻址光的波长与 D1 线的原子荧光波长基本一致,我们设计了由粗至精的两步对准方案:
\begin{itemize}
    \item \textbf{相机粗对准}:利用波长的一致性,我们首先将寻址光束直接引入 EMCCD 相机中,使其与捕捉到的原子荧光阵列图像叠加。通过在相机图像上比较两者的空间重合度,即可快速完成寻址光斑位置的粗对准。
    \item \textbf{原子信号精对准}:在粗对准的基础上,我们通过测量目标原子的实际交流斯塔克频移信号(即寻址光引起的能级移动),进一步精细优化 SLM 的相位图与光路调节,直至目标原子上的光频移达到最大且相邻格点串扰最小。
\end{itemize}


此外,需要特别指出的是,寻址光波长与原子荧光波长的重合在带来对准便利的同时,也引入了探测干扰。由于光学器件(如二向色镜或滤光片)的隔离比有限,较强的 795 nm 寻址光在实验过程中极易发生漏光并串扰进高灵敏度的 EMCCD 相机,这不仅会淹没微弱的单原子荧光信号,甚至可能导致相机传感器的不可逆损坏。为了彻底隔断这一串扰,我们在相机的光路前端物理性地集成了一个机械快门(Mechanical shutter)。在执行里德堡序列与局域寻址操作期间,该快门保持严格闭合状态;仅在最终的原子布居探测阶段,快门才被同步触发开启以收集原子荧光。这一物理隔断设计显著提升了寻址操作与荧光探测的独立性与稳定性。



\section{$\ket{\mathbb{Z}_2}$ 疤痕态的高保真度制备}
本章制备 25 个原子的无缺陷一维链,并将原子间距设置为 $a\approx\SI{7}{\micro\meter}$。该几何下最近邻与次近邻相互作用强度分别为 $V_{i,i+1}=2\pi\times\SI{7.3}{\MHz}$ 与 $V_{i,i+2}=2\pi\times\SI{0.11}{\MHz}$,其比值在装置中基本固定。

\subsection{局域寻址制备方案与物理参数设计}

在大规模量子系统中,以往制备打破平移对称性的反铁磁 $\ket{\mathbb{Z}_2}$ 态常采用绝热演化(Adiabatic sweeping)方案,即通过缓慢扫描全局驱动光的失谐,将系统基态绝热演化至多体哈密顿量的目标基态。然而,随着系统尺寸的增大,希尔伯特空间维数呈指数增长,导致能隙急剧减小,绝热制备的保真度会发生断崖式下降。

为了获得高保真度且可扩展的 $\ket{\mathbb{Z}_2}$ 初态,本实验放弃了绝热路径,转而采用全局里德堡拉曼激发结合 795 nm 局域寻址光的单步直接制备方案。如前节所述,我们在目标原子(不希望被激发的原子)上施加 795 nm 寻址光束。如图 \ref{ch5:fig:SI_Z2}a 中的里德堡激发光谱所示,在全局相干拉曼驱动下,未被寻址的原子(蓝色曲线)的里德堡布居数随激发光失谐呈现出典型的共振峰,而处于寻址光斑下的原子(红色曲线)则因受到约 $2\pi \times \SI{12.2(3)}{\MHz}$ 的极大光频移(AC Stark shift)而严重失谐,被牢牢锁定在基态 $\ket{\downarrow}$。这种选择性失谐使得我们能够一步到位地形成 $\ket{\uparrow\downarrow\uparrow\downarrow\cdots}$ 的交替组态。同时,在态制备阶段,为了最大程度地压制激光相位噪声,我们选取了较高的全局拉比频率 $\Omega = 2\pi \times \SI{1.5}{\MHz}$。

\begin{figure}[htb]   
\centering   
\includegraphics[width=\textwidth]{chapters/chapter_06/figure/SI_Z2.png}  
\caption{$\ket{\mathbb{Z}_2}$ 态制备细节。}   
\label{ch5:fig:SI_Z2} 
\end{figure}


在应用此方案时,光频移符号(失谐方向)的选取至关重要。图 \ref{ch5:fig:SI_Z2}b 展示了在最近邻里德堡相互作用强度 $V_{i,i+1}=3\Omega$ 的条件下,针对不同系统尺寸模拟的 $\ket{\mathbb{Z}_2}$ 态制备保真度随寻址光频移(以拉比频率 $\Omega$ 为单位)变化的函数关系。模拟结果清晰地表明:为了最大化 $\ket{\mathbb{Z}_2}$ 态的保真度,光频移的符号必须与里德堡范德瓦尔斯相互作用($V > 0$)的符号严格相反。如果两者符号相同,在特定的驱动参数下会导致能级共振,从而引发严重的反阻塞效应(Anti-blockade effect)(见图 \ref{ch5:fig:SI_Z2}b 中的保真度凹陷区)。一旦发生反阻塞,相邻原子被同时激发至里德堡态($\ket{\cdots\uparrow\uparrow\cdots}$)的概率将大幅增加,从而彻底破坏受限动力学的初态前提。图 \ref{ch5:fig:SI_Z2}b 中的红星标记了本实验最终选定的无反阻塞工作点,在该条件下,数值模拟给出的制备非保真度极低,约为每个量子比特 $0.012$。

\subsection{制备保真度与微观误差分解}

该寻址制备方案在系统尺寸扩展时展现出了极佳的稳定性和优异的性能。如图 \ref{ch5:fig:SI_Z2}c 所示,我们测量并提取了 $\ket{\mathbb{Z}_2}$ 态制备保真度作为系统尺寸的函数关系。对于中心包含 13 个量子比特的子系统,目标 $\ket{\mathbb{Z}_2}$ 态的原始测量保真度达到了 $70(1)\%$;在校正了单比特状态探测误差(SPAM errors)后,真实制备保真度(图 \ref{ch5:fig:SI_Z2}c 中红线所示的指数拟合结果)高达 $78(1)\%$。当系统扩展至 25 个量子比特的长链时,测量的原始保真度(图 \ref{ch5:fig:SI_Z2}c 中蓝线)为 $49(3)\%$,校正探测误差后依然保持在 $60(3)\%$ 的极高水平。这一结果已非常逼近当前实验寻址方法所能达到的理论上限(图 \ref{ch5:fig:SI_Z2}c 中黄线)。

为了深入理解制备过程中的非理想因素,我们对实验获取的大量微观态分布(Microstate distribution)进行了误差分解。图 \ref{ch5:fig:SI_Z2}d 统计了在 1,774 次 13 量子比特链的实验中,不同微观态随出现次数的分布情况:成功制备的完美 $\ket{\mathbb{Z}_2}$ 态贡献了 1,246 次计数(占比 70\%)。进一步分析测量的微观态分布(图 \ref{ch5:fig:SI_Z2}e)揭示了误差发生的非泊松(Non-Poissonian)特征。如 图 \ref{ch5:fig:SI_Z2}e 的插图(主要错误态)所示,在所有偏离完美 $\ket{\mathbb{Z}_2}$ 态的错误计数中,高达 $86(2)\%$ 的错误来源于单比特从里德堡态到基态的衰减翻转($\ket{\uparrow} \rightarrow \ket{\downarrow}$)。根据误差模型分析,每个量子比特的制备非保真度主要由以下不相关的局域误差通道构成:
\begin{itemize}
    \item 里德堡激发不完美导致的单比特误差(约 $0.8\%$);
    \item 寻址光有限能级偏移引起的泄漏误差(约 $1.2\%$);
    \item 状态检测阶段的探测串扰与损耗,即 $\ket{\downarrow} \rightarrow \ket{\uparrow}$(约 $1\%$)与 $\ket{\uparrow} \rightarrow \ket{\downarrow}$(约 $0.5\%$)的 SPAM 误差。
\end{itemize}

这种以单比特 $\ket{\uparrow} \rightarrow \ket{\downarrow}$ 翻转为主导的误差分布不仅证实了我们的多体态制备瓶颈可被解构为局域的单比特操作误差,也为后续在量子信息加扰和复杂动力学演化数据中执行确定性的误差缓解(Error mitigation)提供了关键的理论输入。




\section{逼近 PXP 动力学的参数优化与边界效应屏蔽}

在完成了 $\ket{\mathbb{Z}_2}$ 态的高保真度制备后,下一阶段的实验目标是驱动系统演化,以观测动力学受限下的量子多体疤痕与弱热化现象。理想的 PXP 哈密顿量(方程 \ref{ch5new:eq:pxp})是一个理论上的极致抽象,它要求物理系统同时满足两个极其严苛的极限条件:(1) 完美的里德堡阻塞,即最近邻相互作用远大于驱动强度($V_{i,i+1} \gg \Omega$);(2) 零长程相互作用尾巴,即次近邻及更远距离的相互作用可被完全忽略($V_{i,j>i+1} = 0$)。

然而,在真实的二维光镊一维原子阵列中,原子间的范德瓦尔斯相互作用遵循 $V \propto 1/R^6$ 的衰减规律。对于等间距的原子链,这意味着最近邻与次近邻相互作用的比值是一个被硬件几何严格锁定的常数,即 $V_{i,i+1}/V_{i,i+2} = 2^6 = 64$。这种固有的物理约束使得我们无法在实验中同时完美满足上述两个极限条件,从而迫使我们在演化参数的设计上寻找最佳的物理折中。

\begin{figure}[htb]   
\centering   
\includegraphics[width=0.1\textwidth]{chapters/chapter_06/figure/SI_Z2.png}  
\caption{
假想图 (Fig. 5.X) 设计方案
图 a(权衡寻找最佳 $V/\Omega$):纵坐标为经过几个周期演化后的$\ket{\mathbb{Z}_2}$态多体保真度(或$\braket{n_c}$的振荡包络对比度),横坐标为$ V_{i,i+1}/\Omega $的比值。曲线呈现一个倒 V 型,最高点落在$ V_{i,i+1}/\Omega=6 $。
图 b(失谐补偿):横坐标为失谐量 $\Delta$,纵坐标为与理想 PXP 演化态的态重叠(State Overlap)。最高点落在$\Delta = 2V_{i, i+2}$。
图 c(边界效应导致加速):(这是你已有数据的完美复用!) 画出开边界(OBC)13 比特链中,边缘比特(红色)和中心比特(蓝色)的 $\braket{n_j(t)}$演化曲线。明显看出边缘比特振荡更快,导致相位差累积。
图 d(尺寸屏蔽效应):对比 25 比特开边界链(OBC)的中心 13 个原子,与 10 比特周期边界链(PBC)的演化曲线,证明两者在实验时间尺度内高度重合。
}   
\label{ch5:fig:fake_img} 
\end{figure}

\subsection{$V_{i,i+1}/\Omega$ 的内在权衡与演化拉比频率的选取}

由于 $V_{i,i+1}/V_{i,i+2} \approx 64$ 固定,系统的演化动力学对拉比频率 $\Omega$ 的选取极为敏感。如果为了追求极致的里德堡阻塞而将 $\Omega$ 设得极小(例如 $V_{i,i+1}/\Omega = 16$),那么次近邻相互作用将不可避免地与驱动强度变得相当($\Omega \sim 4V_{i,i+2}$),这会严重破坏 PXP 模型的受限动力学前提;反之,如果为了忽略次近邻相互作用而增大 $\Omega$,则会导致阻塞不彻底,使得双激发态($\ket{\uparrow\uparrow}$)的布居数增加。

为了寻找这一内在权衡(Trade-off)的最佳工作点,我们数值模拟了系统从 $\ket{\mathbb{Z}_2}$ 初态出发的受限演化,并以演化多个周期后局部里德堡布居数 $\langle n_i(t) \rangle$ 的复苏振荡对比度(或多体保真度)作为优化指标。如图 \ref{ch5:fig:fake_img}a 所示,模拟结果表明,偏离简单理论几何平均值的 $V_{i,i+1} = 6\Omega$ 是最小化疤痕动力学衰减的最佳比例。在我们的实验几何下,最近邻相互作用强度标定为 $V_{i,i+1} = 2\pi \times \SI{7.3}{\MHz}$。基于上述优化结果,我们在动力学演化阶段,将全局驱动的拉比频率严格从态制备阶段的 1.5 MHz 降低至 $\Omega = 2\pi \times \SI{1.21(1)}{\MHz}$。

\subsection{全局失谐对长程相互作用尾巴的补偿}

尽管确立了最佳的 $V/\Omega$ 比例,次近邻相互作用 $V_{i,i+2} \approx 2\pi \times \SI{0.11}{\MHz}$ 依然作为残余的微扰存在。在 $\ket{\mathbb{Z}_2} = \ket{\uparrow\downarrow\uparrow\downarrow\cdots}$ 态中,每一个处于里德堡态的原子都会感受到来自其两侧次近邻原子的额外范德瓦尔斯排斥。这种长程尾巴会使得疤痕态的能谱发生轻微的非谐性扭曲,进而导致复苏动力学的退相干。

在平均场(Mean-field)近似的图像下,这种来自次近邻的恒定排斥可以等效为一个局域的能级移动。因此,我们可以在演化驱动中引入一个全局的微小负失谐 $\Delta$ 来进行物理补偿。如图 \ref{ch5:fig:fake_img}b 的数值模拟所示,当施加 $\Delta = 2V_{i,i+2} \approx 2\pi \times \SI{0.22(1)}{\MHz}$ 的失谐时,系统演化的轨迹与理想 PXP 模型的态重叠(State overlap)达到最大化。这一巧妙的失谐补偿策略进一步抹平了真实里德堡气体与理想 PXP 模型之间的差距。

\subsection{边缘效应的破坏与“隔离带”屏蔽策略}

除了相互作用参数的优化,一维原子链的有限尺寸(Finite-size)与开边界条件(OBC)也会对体(Bulk)内的 PXP 动力学造成致命破坏。在开边界下,处于原子链最边缘的量子比特缺少了一侧的相邻原子,这打破了系统的平移对称性。从动力学约束的角度来看,边缘原子受到的里德堡阻塞约束减半,导致其感受到的有效拉比驱动强度显著大于链内部的原子。

为了定量分析这一边界效应,我们数值模拟了 13 量子比特开边界链的演化。如图 \ref{ch5:fig:fake_img}c 所示,边缘量子比特的拉比振荡频率明显快于中心量子比特。随着演化时间的推移,这种由边界诱导的相位差会像水波一样从边缘向链的内部传播,导致原本应保持同步相干旋转的 $\ket{\mathbb{Z}_2}$ 态“波前(Wavefront)”发生严重的向内弯曲,最终彻底污染中心区域的体动力学表征。

为了在实验上规避这一破坏性影响,我们采取了一种物理上的“隔离带(Isolation zone)”策略。实验中,我们利用光镊重排技术制备长达 25 个原子的超长 $\ket{\mathbb{Z}_2}$ 链,但在最终的探测与数据分析环节,我们完全丢弃两端各 6 个原子的信号,仅将分析窗口严格聚焦在中心的 13 个量子比特上。如图 \ref{ch5:fig:fake_img}d 的数值证明所示,在实验有限的演化时间尺度($\sim 2\ \mu\text{s}$)内,边缘的扰动尚未有足够的时间传播至中心区域。比较 25 比特开边界系统(OBC)中心区域的动力学与 10 比特周期边界系统(PBC)的演化,两者展现出了极高的重合度(差异可忽略不计)。这一“空间截取”策略不仅在实验上成功屏蔽了边界效应,完美逼近了体 PXP 动力学,同时也为我们在后续数值模拟中采用 10 比特 PBC 模型来高效替代大型 OBC 系统提供了坚实的合法性依据。




\section{动力学受限系统中的疤痕演化与弱热化观测}

在确立了逼近 PXP 动力学的最佳实验参数并应用“隔离带”策略屏蔽边界效应后,我们正式对系统进行时间演化,以观测一维受限里德堡阵列中的量子多体疤痕(QMBS)及其弱热化现象。

\subsection{受限多体动力学此时空图像与局域缺陷传播}

为了全面刻画系统的受限动力学,我们对比研究了两种初态的演化:完美的全局反铁磁态 $\ket{\mathbb{Z}_2}$,以及带有一个局域缺陷的 $\sigma_c^x\ket{\mathbb{Z}_2}$ 态(即仅将中心原子的自旋人为翻转,轻微打破全局平移对称性)。

如图 \ref{ch5:fig:dynamics}a 所示,当系统从完美的 $\ket{\mathbb{Z}_2}$ 初态出发时,动力学受限使得微观态无法快速遍历整个希尔伯特空间。在 PXP 哈密顿量的支配下,原本应迅速热化的系统表现出了显著的、跨越多个周期的同步相干旋转(Synchronized coherent rotations)。然而,正如我们在 5.6 节中所探讨的,即便仅分析中心区域的原子,边界效应导致的相位超前依然会随着时间推移逐渐向内传播。这在时空热图上直观地表现为原本平直的等振幅“波前(Wavefront)”在演化后期出现了明显的向内弯曲为了在实验与数值上定量且严格地提取这一波前结构,我们采用了一种基于等概率点的追踪算法:通过在时空演化数据中寻找相邻原子达到相同里德堡激发概率(即$P_i(\uparrow)=P_{i+1}(\uparrow)$)的时间节点,并将这些离散的等概率点连接起来,从而准确重构出激发在系统中的传播波前(如图 \ref{ch5:fig:dynamics}a 中红色虚线所示)。

相较于全局演化,引入局域缺陷的 $\sigma_c^x\ket{\mathbb{Z}_2}$ 态为理解受限动力学提供了另一种视角。在传统的遍历(Ergodic)系统中,一个局域缺陷通常会迅速扩散并导致周边区域快速热化。然而,在里德堡阻塞的动力学约束下,缺陷的传播行为被深刻地重塑。如图 \ref{ch5:fig:dynamics}b 所示,中心自旋的翻转在周围原子的演化中诱导了一种显著的“局部迟滞(Local retardation)”效应。这种动力学迟滞随着系统的演化向外传播,形成了一个非常清晰的线性光锥(Linear light cone)结构(图 \ref{ch5:fig:dynamics}b 中蓝色虚线)。在光锥外部,自旋依然保持类似完美 $\ket{\mathbb{Z}_2}$ 态的同步振荡;而在光锥内部,局部自旋由于相互作用的约束出现了相位的错位,形成了弯曲的弧状波前。实验观测与考虑了实验噪声的数值模拟(图 \ref{ch5:fig:dynamics} 对应下半部分)展现出了极高的一致性,证明该平台能够精确捕捉动力学约束对局域微扰传播的塑造作用。

\todo{(此处插入图 \ref{ch5:fig:dynamics}:即原 `Z2.png` 的 d-f,展示两类初态的演化热图及光锥、波前的标注。)}

\begin{figure}[htb]
  \centering
  \includegraphics[width=\textwidth]{chapters/chapter_06/figure/Z2.png}
  \caption{
\textbf{态制备与动力学受限多体态演化。} 
\textbf{a}, 在全局相干驱动下,\SI{795}{\nm} 激光寻址的原子(红色)保持静止,而未寻址的原子(蓝色)表现出基态-里德堡态拉比振荡。
\textbf{b}, $\ket{\mathbb{Z}_2}$ 态制备保真度 $\mathcal{F}_{\mathbb{Z}_2}$ 作为系统尺寸的函数。
\textbf{c}, 测量的 25 量子比特 $\ket{\mathbb{Z}_2}$ 态的格点分辨里德堡概率 $P_i(\uparrow)$。上图:$\ket{\mathbb{Z}_2}$ 态的原子荧光图示,其中红圈表示里德堡原子。
\textbf{d}--\textbf{g}, 动力学受限多体动力学。
\textbf{d} 和 \textbf{f}, 分别从初态 $\ket{\mathbb{Z}_2}$ 和 $\hat\sigma^x_c\ket{\mathbb{Z}_2}$ 开始,测量的作为演化时间函数的格点分辨里德堡态概率 $P_i(\uparrow)$。
在 $\hat\sigma^x_c\ket{\mathbb{Z}_2}$ 情况 (\textbf{f}) 中观察到线性光锥结构,其起源在第~\ref{ch5:sec:boundary} 节中详述。
\textbf{e} 和 \textbf{g} 是来自数值模拟的相应结果。
这些模拟采用了完整的里德堡哈密顿量,并根据我们对这些噪声源的表征考虑了实验噪声。
\textbf{d}--\textbf{g} 中的红色虚线曲线展示了传播波前,而 \textbf{f} 和 \textbf{g} 中的蓝色虚线显示了线性光锥。}
\label{ch5:fig:dynamics}
\end{figure}

\subsection{弱热化动力学与疤痕态复苏寿命}

上述的时空图像直观地展示了受限动力学的局域特征,为了定量表征量子多体疤痕的全局弱热化(Weak thermalization)行为以及系统的相干寿命,我们提取并分析了中心 13 个量子比特的全局观测量演化。

我们重点关注了两个对疤痕态极为敏感的可观测量:全局里德堡态布居数 $P(\uparrow)$ 以及平均畴壁密度(Domain-Wall Density, DWD)。如图 \ref{ch5:fig:Z2_lifetime}a 所示,在仅包含正向时间演化($e^{-iHt}$)的过程中,里德堡布居数表现出大幅度且缓慢衰减的复苏振荡。通过高达 5 阶的阻尼傅里叶级数对布居数演化进行拟合,我们提取出其 $1/e$ 衰减时间(Lifetime)约为 $\SI{2.8(2)}{\micro\second}$。同时,如图 \ref{ch5:fig:Z2_lifetime}b 所示,反映相邻原子关联状态的平均畴壁密度也呈现出相似的复苏行为,利用阻尼正弦函数拟合得到的相干寿命约为 $\SI{1.5(1)}{\micro\second}$。

在理想的 PXP 模型中,这种衰减主要源于 $\ket{\mathbb{Z}_2}$ 初态不仅投影在疤痕态子空间,还包含了少量与热本征态的重叠,从而导致内禀的退相干。而在真实的实验系统中,高达数微秒的可分辨振荡窗口虽然证明了疤痕态的鲁棒性,但有限的寿命不可避免地受到了外部耗散的限制。结合系统的独立标定数据,实验中主导疤痕态衰减的物理误差(Error Budget)主要来源于以下几个通道:首先是演化过程中的退极化误差(Depolarization),在双光子拉曼激发过程中,系统不可避免地会耦合到中间短寿命激发态($\ket{e}=\ket{5P_{3/2}}$),从而引入额外的自发辐射损耗;其次是有限温度效应,实验中释放-俘获测得的原子温度约为 $\sim \SI{10}{\micro\K}$,这不仅会带来多普勒频移引发的失谐波动,还会导致光镊中原子的位置涨落,由于里德堡相互作用对距离极其敏感($V\propto1/R^6$),微小的位置不确定性会被急剧放大为相互作用噪声;最后是驱动激光的高频相位噪声(例如伺服回路引入的 Servo bump)。这些机制共同构成了限制复苏寿命的物理瓶颈。

尽管不可避免地受到环境噪声与微小长程尾巴的影响,本节观测到的大振幅、长寿命的复苏振荡,以及受限的线性光锥传播,依然构成了量子多体系统违反本征态热化假说(ETH)的强有力证据。

\todo{(此处插入图 \ref{ch5:fig:Z2_lifetime}:即原 `SI\_lifetime.png` 的 b, d,展示仅正向演化时布居数和 DWD 的衰减拟合曲线。)}

\begin{figure}[htb] 
\centering  
\includegraphics[width=0.75\textwidth]{chapters/chapter_06/figure/SI_lifetime.png} 
\caption{ \textbf{PXP 哈密顿量演化下的 $\ket{\mathbb{Z}_2}$ 态动力学。}  
\textbf{a} 和 \textbf{b}, 初始化量子比特在 \textbf{a}, 正向-反向演化 \((e^{-iHt}\) 后跟 \(e^{iHt})\) 和 \textbf{b}, 仅正向 \((e^{-iHt})\) 演化期间的 \(\ket{\uparrow}\)(蓝色)和 \(\ket{\downarrow}\)(红色)布居。 \textbf{a} 中的曲线代表指数拟合,而 
\textbf{b} 用高达 5 阶的阻尼傅里叶级数拟合。 
\textbf{c} 和 \textbf{d}, 中心 13 个量子比特在 \textbf{c}, 正向-反向演化和 \textbf{d}, 仅正向演化期间的平均畴壁密度 (DWD)。\textbf{c} 中的蓝色曲线是指数拟合,而 \textbf{d} 拟合为阻尼正弦函数。  }   
\label{ch5:fig:Z2_lifetime} 
\end{figure}

\section{本章小结}
本章围绕“动力学约束如何支撑量子多体疤痕”建立了从里德堡原子阵列到 PXP 模型的理论与实验闭环:通过里德堡阻塞诱导的受限希尔伯特子空间获得有效 PXP 动力学;在固定几何与长程相互作用背景下,通过选择合适的 $\Omega$ 与 $\Delta$ 工作点并关注中心 13 个量子比特,尽可能逼近体 PXP 行为;利用 SLM 寻址方案高保真制备 $\ket{\mathbb{Z}_2}$ 疤痕态,并以里德堡布居与畴壁密度的复苏与衰减定量表征弱热化动力学。

传统 QMBS 研究主要关注波函数重叠或局域可观测量的复苏。基于本章建立的受限多体动力学平台,一个自然的问题是:若在局域编码量子信息,它在多体相互作用与动力学约束共同作用下会如何传播与演化?第六章将引入更直接面向量子信息的观测量,对疤痕态背景下的时空动力学展开讨论。

\newpage

% \section{实验系统的参数设计与优化}
% 本章实验基于第 3、4 章所述可编程里德堡原子阵列平台。本文制备多达 25 个原子的无缺陷一维链,并将原子间距设置为 $a\approx\SI{7}{\micro\meter}$。该几何下最近邻与次近邻相互作用强度分别为 $V_{i,i+1}=2\pi\times\SI{7.3}{\MHz}$ 与 $V_{i,i+2}=2\pi\times\SI{0.11}{\MHz}$,其比值在装置中基本固定。

% \subsection{逼近理想 PXP 的工作点选择}
% 真实装置中次近邻相互作用 $V_{i,i+2}$ 不能完全忽略。为了在固定的相互作用比值下尽可能逼近 PXP 模型,需要满足 $V_{i,i+2}\ll\Omega\ll V_{i,i+1}$,从而在 $V_{i,i+1}/\Omega$ 的选择上形成权衡。综合考虑后,本文在演化过程中选取基态—里德堡驱动拉比频率 $\Omega=V_{i,i+1}/6=2\pi\times\SI{1.21(1)}{\MHz}$,并引入小失谐 $\Delta=2\pi\times\SI{0.22(1)}{\MHz}$ 以部分补偿残余的次近邻相互作用 $V_{i,i+2}$。在该参数下,实验里德堡哈密顿量可由 PXP 模型较好近似。

% 在真实实验中,次近邻相互作用 $V_{i,i+2}$ 不能完全忽略。为在一维等间距几何下逼近 PXP 模型,需要满足 $V_{i,i+2} \ll \Omega \ll V_{i,i+1}$。由于 $V_{i,i+1}/V_{i,i+2}$ 在本文装置中固定约为 64,上述不等式对应一组不可同时极限满足的约束,从而在 $V_{i,i+1}/\Omega$ 的选择上形成权衡。

% 为确定最优比值,本文在不同参数下对 ZZ-OTOC 进行数值模拟,并以塌缩与复苏振荡对比度的衰减率作为优化指标。图~\ref{ch5:fig:experimental-parameters}a 显示最佳选择为 $V_{i,i+1} = 6\Omega$,该结果不同于简单的理论中间值 $\Omega=\sqrt{V_{i,i+1} V_{i,i+2}}$。

% 拉比频率的选取需要在两个因素之间折衷:(1) 时间反演保真度:图~\ref{ch5:fig:experimental-parameters}b 表明,当 $V_{i,i+1}/\Omega$ 固定为 6 且正向与反向演化间隔固定为 \SI{200}{\nano\second} 时,提高 $\Omega$ 会降低时间反演保真度;(2) 单原子相干性:更高的拉比频率有助于抑制激光相位噪声主导的相干性限制。综合考虑后,本文取 $\Omega = 2\pi \times \SI{1.21(1)}{MHz}$。

% 除优化 $V_{i,i+1}$ 与 $\Omega$ 外,本文引入小失谐 $\Delta$ 以抵消残余次近邻相互作用 $V_{i,i+2}$。数值模拟表明,$\Delta = 2V_{i,i+2}$ 时可较好再现 OTOC 的塌缩与复苏动力学(图~\ref{ch5:fig:experimental-parameters}c)。

% OTOC 测量中,正向与反向演化之间插入间隙以实施局域与全局单量子比特 $\sigma^z$ 操作。为将间隙时间压缩至 $\sim \SI{200}{ns}$,本文并行执行局域微扰与全局 $\sigma^z$ 旋转,并利用其对易关系。图~\ref{ch5:fig:experimental-parameters}d 给出有限间隙对 OTOC 动力学的数值评估,结果表明本文采用的 \SI{200}{ns} 单量子比特操作时间不会显著改变塌缩与复苏振荡的可观测性。

% 综上,参数优化使本文实验在固有几何与相互作用约束下仍能紧密逼近 PXP 模型(图~\ref{ch5:fig:experimental-parameters}d),并为高保真度地实现动力学约束下的相干自旋旋转与疤痕动力学提供支撑。

% \begin{figure}[htb]
% \centering
% \includegraphics[width=0.7\textwidth]{chapters/chapter_06/figure/SI_experimental_parameters.png}
% \caption{\textbf{优化实验参数以紧密近似理想 PXP 哈密顿量动力学。} 
% 具有周期性边界条件的 10 量子比特链中中心量子比特的 OTOC 和时间反演保真度的数值模拟。
% \textbf{a}, 模拟的不同最近邻 (NN) 里德堡相互作用 $V_{\text{NN}}$ 下的 OTOC 动力学,与理想 PXP 模型进行比较。相互作用强度从 $2\pi \times \SI{6}{MHz}$ 变化到 $2\pi \times \SI{18}{MHz}$,步长为 $2\pi \times \SI{1.5}{MHz}$。最佳选择 $V_{\text{NN}} = 2\pi \times \SI{9}{MHz}$ 最匹配理想情况。 
% \textbf{b}, 时间反演保真度作为拉比频率 $\Omega$ 的函数,其中 $V_{\text{NN}}/\Omega$ 固定为 6。显示了 $t=0.5\mu s$(蓝色符号)、$1.0\mu s$(黄色符号)和 $1.5\mu s$(红色符号)的不同时间演化。随着 $\Omega$ 增加,保真度降低。 
% \textbf{c}, 不同失谐值 $\delta$ 下的 OTOC 动力学,范围从 $-0.3 V_{\text{NNN}}$ 到 $0.3 V_{\text{NNN}}$,其中 $V_{\text{NNN}}$ 是次近邻 (NNN) 相互作用。黑色曲线代表理想 PXP 模型。最佳失谐 $\delta = 0.1 V_{\text{NNN}}$ 最匹配理想动力学。 \textbf{d}, 使用优化实验参数(紫色曲线)和理想 PXP 模型(黑色曲线)的 OTOC 动力学比较,显示出紧密一致。优化参数有效地再现了 OTOC 中观察到的关键振荡。} 
% \label{ch5:fig:experimental-parameters}
% \end{figure}


% 有效 PXP 哈密顿量的近似依赖于两个关键假设:(1) 忽略长程相互作用 $V_{i,j>i+1}$。(2) 确保最近邻相互作用主导拉比频率($V_{i,i+1} \gg \Omega$)。然而,实际上,次近邻相互作用 $V_{i,i+2}$ 不能被忽略。为了紧密近似 PXP 模型,系统必须在 $V_{i,i+1}/V_{i,i+2} \gg 1$ 的区域运行。这是一个挑战,因为在我们具有等间距原子的一维几何结构中,比率 $V_{i,i+1}/V_{i,i+2}$ 固定在约 64。例如,设置 $V_{i,i+1}/\Omega = 3$ 以满足第二个假设会导致 $V_{i,i+2}/\Omega \approx 0.05$,使得难以完全抑制 $V_{i,i+2}$ 的影响。这在两个关键假设之间产生了权衡。  为了找到 $V_{\text{NN}}/\Omega$ 的最佳比率,我们对不同参数下的 ZZ-OTOC 进行了数值模拟。我们确定了最小化塌缩与复苏振荡对比度衰减的最佳比率,如图~\ref{ch5:fig:experimental-parameters}a 所示。基于这些结果,我们将 $V_{\text{NN}}/\Omega$ 的实验比率设置为 6,这与理论中间值 $3$ 不同。  拉比频率的选择涉及平衡两个相互竞争的因素:(1) 时间反演保真度:我们的模拟(图~\ref{ch5:fig:experimental-parameters}b)表明,当 $V_{\text{NN}}/\Omega$ 固定为 6 且正向和反向演化之间的间隔固定为 \SI{200}{\nano\second} 时,增加拉比频率会降低时间反演保真度。(2) 单原子相干性:较高的拉比频率提高了主要受激光相位噪声限制的单原子相干性。在仔细权衡这些因素后,我们选择了 $\Omega = 2\pi \times \SI{1.5}{MHz}$ 的拉比频率。  除了优化 $V_{\text{NN}}$ 和 $\Omega$ 外,我们还引入了一个小的失谐 $\delta$ 来抵消残留的次近邻相互作用 $V_{\text{NNN}}$。我们的模拟表明,设置 $\delta = 0.1 V_{\text{NNN}}$ 最能捕捉 OTOC 动力学(图~\ref{ch5:fig:experimental-parameters}c)。  

% \subsection{边界与有限尺寸效应}
% \label{ch5:sec:boundary}
% 有限尺寸量子比特链中,边界量子比特缺乏部分相邻约束,从而使边缘与体量子比特的有效驱动环境不同,并引入显著边界效应。数值与实验结果均显示,边界影响会从外向内传播并逐步弯曲波前,进而影响对体动力学的表征。为减弱边界与有限尺寸效应,本文采用 25 量子比特链并将动力学分析聚焦在中心 13 个量子比特区域,从而更好逼近体 PXP 动力学。类似的实验策略也可在后续章节的更精细测量中复用(此处不展开)。

% \begin{figure}[htb]
% \centering
% \includegraphics[width=0.8\textwidth]{chapters/chapter_06/figure/SI_Boundary_and_finite-size_effect.png}
% \caption{{\textbf{边界和有限尺寸效应。}
% 初始化在 $\ket{\mathbb{Z}_2}$ 态的各种系统尺寸和边界条件下理想里德堡哈密顿量的模拟 OTOC 动力学与时间演化。
% \textbf{a}, 13 量子比特链中边界效应示意图。边界量子比特缺乏最近邻与次近邻,打破平移对称性。
% \textbf{b} 和 \textbf{c}, 分别从 $\ket{\downarrow}$ 态与 $\ket{\uparrow}$ 态出发的 13 量子比特链中边缘(红色)与中心(蓝色)量子比特的 $\left\langle n_j(t) \right\rangle$ 数值结果。边缘量子比特的加速振荡反映边界效应。
% \textbf{d} 和 \textbf{e}, 9 量子比特链(红色曲线)中边缘量子比特(\textbf{d})及其相邻量子比特(\textbf{e})的 OTOC 动力学,并与 25 量子比特链(蓝色曲线)中相应量子比特对比。插图给出偏差 $|\delta_\text{OTOC}|$,表明边界效应可传播到链内部。
% \textbf{f}, 理想 PXP 哈密顿量(黄色),以及在含噪里德堡哈密顿量下使用 IZ-OTOC(蓝色)与边缘量子比特 ZZ-OTOC(红色)归一化的 OTOC 演化。
% \textbf{g}, 归一化 OTOC 与理想 PXP 情况的偏差(红色:边缘 ZZ-OTOC;蓝色:IZ-OTOC)。
% \textbf{h} 和 \textbf{i}, 开放边界条件下不同链长(5、9、13、25)中中心量子比特(\textbf{h})与最近邻量子比特(\textbf{i})的模拟 OTOC 动力学。
% \textbf{j} 和 \textbf{k}, 10 量子比特链(周期性边界条件,PBC)中距微扰最远的量子比特与 25 量子比特链(开放边界条件,OBC)中边缘量子比特的 OTOC 动力学对比。}}
% \label{ch5:fig:boundary}
% \end{figure}



% 在有限尺寸量子比特链中,边界量子比特的存在打破了 PXP 哈密顿量对于某些初态如 $\ket{\mathbb{Z}_2} = \ket{...\uparrow \downarrow\uparrow\downarrow\uparrow...}$ 和 $\ket{\mathbf{0}} = \ket{...\downarrow \downarrow\downarrow\downarrow\downarrow...}$ 的平移对称性。这种对称性破缺导致边缘和体量子比特之间的相互作用强度不同(图~\ref{ch5:fig:boundary}a),引入显著的边界效应。  为了定量分析边界效应,我们数值模拟了 13 量子比特链中中心和边缘量子比特的演化动力学,从理想里德堡哈密顿量下的 $\ket{\mathbb{Z}_2}$ 态开始。如图~\ref{ch5:fig:boundary}b,c 所示,边界效应导致边缘量子比特表现出加速的周期性振荡。这种加速源于对边缘量子比特的约束减少,导致更强的有效驱动强度。此外,数值模拟和实验结果都表明,边界效应逐渐从外边缘向内改变自旋旋转,导致最初均匀传播的波前在 $\ket{\mathbb{Z}_2}$ 初态的演化过程中发生弯曲,强调了边界效应在塑造系统动力学中的关键作用。  我们通过比较不同长度(具有开放边界条件的 5、9、13 和 25 量子比特链)链中中心两个量子比特的 OTOC 动力学进一步研究有限尺寸效应。数值模拟表明,在较小的原子系统中,有限尺寸效应非常明显,如图~\ref{ch5:fig:boundary}h,i 所示。这表明我们需要延长链长以确保我们的实验观测准确反映量子信息的加扰。然而,即使使用 MPO 方法,模拟 25 个原子也需要大量的计算资源。为了解决这个挑战,我们模拟了具有开放边界条件 (OBC) 的 25 量子比特链和具有周期性边界条件 (PBC) 的 10 量子比特链,如图~\ref{ch5:fig:boundary}j,k 所示。我们比较了距离微扰最远的原子的 OTOC 动力学。结果显示在实验时间尺度内差异可忽略不计,因为在此期间原子不受两端等效局域微扰的影响。基于此分析,在模拟 OTOC 动力学时,我们采用具有 PBC 的 10 量子比特链的模拟。 

% \section{$\mathbb{Z}_2$ 疤痕态的高保真度制备与表征}
% \subsection{寻址制备方案与动机}
% 在大规模系统中,通过绝热路径将设计哈密顿量基态从 $\ket{\mathbf{0}}$ 变换到 $\ket{\mathbb{Z}_2}$ 的方案会随系统尺寸增大而迅速失效:希尔伯特空间维数指数增长并导致能隙变小,从而使绝热制备保真度下降\cite{bernien2017probing,omran2019generation,bluvstein2021controlling}。为获得可扩展的高保真度 $\ket{\mathbb{Z}_2}$ 初态,本文将全局里德堡激发与格点选择性寻址相结合:空间光调制器(Spatial Light Modulator, SLM)在原子阵列上写入定制光频移图案,使指定格点对基态—里德堡跃迁失谐,从而形成可激发与不可激发原子的交替排列。

% \subsection{保真度与误差分解}

% \SI{795}{\nm} 寻址光束,相对于 D1 线共振失谐 $2\pi \times \SI{15}{\GHz}$,在基态上诱导 $2\pi \times \SI{12.2(3)}{MHz}$ 的能量移动。由 \SI{795}{\nm} 寻址光束引起的散射率约为 $2\pi \times \SI{5}{kHz}$,明显低于拉曼激光束引起的散射率。数值模拟(图~\ref{ch5:fig:SI_Z2}b)表明,为了最大化 $\ket{\mathbb{Z}_2}$ 态的保真度,光频移的符号必须与里德堡相互作用的符号相反,以避免反阻塞效应。在我们的实验条件下,数值模拟产生的制备非保真度约为每个量子比特 0.012。  

% 实验制备的 $\ket{\mathbb{Z}_2}$ 态的微观态分布分析揭示了非泊松误差发生(图~\ref{ch5:fig:SI_Z2}d)。我们对微观态分布的测量表明,$\ket{\mathbb{Z}_2}$ 态制备中的误差主要归因于单量子比特 $\ket{\uparrow}\rightarrow\ket{\downarrow}$ 翻转。这些误差态占所有与 $\ket{\mathbb{Z}_2}$ 态相关的出现次数的 86(2)\%(图~\ref{ch5:fig:SI_Z2}d,e)。其余的误差贡献主要源于探测误差(每个量子比特约 1\%)。图~\ref{ch5:fig:SI_Z2}e 显示了测量的微观态分布,突显了主要误差是从 $\ket{\uparrow}$ 到 $\ket{\downarrow}$ 的单量子比特翻转,进一步证实了我们的方法采用单量子比特操作来制备多体态。这种可预测的误差分布有助于在 $\ket{\mathbb{Z}_2}$ 态中量子信息加扰的实验测量数据中进行误差缓解。

% \begin{figure}[htb]   
% \centering   
% \includegraphics[width=\textwidth]{chapters/chapter_06/figure/SI_Z2.png}  
% \caption{     
% \textbf{$\ket{\mathbb{Z}_2}$ 态制备细节。}     
% \textbf{a}, {被寻址(红色)和未被寻址(蓝色)原子的里德堡激发光谱。里德堡布居显示为拉曼激发激光失谐的函数。}     
% \textbf{b}, {反阻塞效应和寻址激光诱导的光频移对态制备保真度的影响。模拟的 $\ket{\mathbb{Z}_2}$ 态制备保真度作为 \SI{795}{\nm} 寻址激光光频移(以里德堡拉曼激发拉比频率 $\Omega$ 为单位)的函数,针对各种系统尺寸,最近邻里德堡相互作用强度 $V_{i,i+1}=3\Omega$。红星标记了我们的实验条件。}     
% \textbf{c}, {$\ket{\mathbb{Z}_2}$ 态制备保真度作为系统尺寸的函数。红线和蓝线:校正和未校正 $\ket{\mathbb{Z}_2}$ 态保真度的指数拟合。黄线:当前实验寻址方法的理论上限。}     
% \textbf{d}, {13 量子比特微观态数量作为来自 1,774 次实验的出现次数的函数。成功制备的 $\ket{\mathbb{Z}_2}$ 态:1,246 次计数(70\% 的事件)。}     
% \textbf{e}, {测量的 13 量子比特微观态分布。插图:主要错误态。}   
% }   
% \label{ch5:fig:SI_Z2} 
% \end{figure}


% 该方案在系统尺寸扩展时保持良好性能:中心 13 个量子比特的 $\ket{\mathbb{Z}_2}$ 态测量保真度为 70(1)\%,探测误差校正后为 78(1)\%;整个 25 量子比特链的测量保真度为 49(3)\%,校正后为 60(3)\%。误差分解表明,制备非保真度主要来自不相关的单量子比特翻转误差,其中主要错误通道为 $\ket{\uparrow}\rightarrow\ket{\downarrow}$。该微观态误差结构一方面用于评估本章动力学观测的可靠性,另一方面也为下一章更精细的误差分析提供输入(此处不展开)。

% \section{动力学受限系统中的疤痕演化与弱热化观测}
% \subsection{受限多体动力学的时空图像}

% \begin{figure}[htb]
%   \centering
%   \includegraphics[width=\textwidth]{chapters/chapter_06/figure/Z2.png}
%   \caption{
% \textbf{态制备与动力学受限多体态演化。} 
% \textbf{a}, 在全局相干驱动下,\SI{795}{\nm} 激光寻址的原子(红色)保持静止,而未寻址的原子(蓝色)表现出基态-里德堡态拉比振荡。
% \textbf{b}, $\ket{\mathbb{Z}_2}$ 态制备保真度 $\mathcal{F}_{\mathbb{Z}_2}$ 作为系统尺寸的函数。
% \textbf{c}, 测量的 25 量子比特 $\ket{\mathbb{Z}_2}$ 态的格点分辨里德堡概率 $P_i(\uparrow)$。上图:$\ket{\mathbb{Z}_2}$ 态的原子荧光图示,其中红圈表示里德堡原子。
% \textbf{d}--\textbf{g}, 动力学受限多体动力学。
% \textbf{d} 和 \textbf{f}, 分别从初态 $\ket{\mathbb{Z}_2}$ 和 $\hat\sigma^x_c\ket{\mathbb{Z}_2}$ 开始,测量的作为演化时间函数的格点分辨里德堡态概率 $P_i(\uparrow)$。
% 在 $\hat\sigma^x_c\ket{\mathbb{Z}_2}$ 情况 (\textbf{f}) 中观察到线性光锥结构,其起源在第~\ref{ch5:sec:boundary} 节中详述。
% \textbf{e} 和 \textbf{g} 是来自数值模拟的相应结果。
% 这些模拟采用了完整的里德堡哈密顿量,并根据我们对这些噪声源的表征考虑了实验噪声。
% \textbf{d}--\textbf{g} 中的红色虚线曲线展示了传播波前,而 \textbf{f} 和 \textbf{g} 中的蓝色虚线显示了线性光锥。}
% \label{ch5:fig:dynamics}
% \end{figure}

% 如图~\ref{ch5:fig:dynamics} 所示,本文比较了 PXP 有效动力学下 $\ket{\mathbb{Z}_2}$ 与 $\sigma_c^x\ket{\mathbb{Z}_2}$ 两类自旋配置的演化差异。对于 $\ket{\mathbb{Z}_2}$ 初态,自旋在约束下表现出近似同步的相干旋转;受边界效应影响,边缘处动力学差异逐步向内传播并弯曲波前。对于 $\sigma_c^x\ket{\mathbb{Z}_2}$ 初态,中心翻转引入局域迟滞并形成清晰的线性光锥结构,反映动力学约束对局域扰动传播的重塑。实验数据与数值结果的一致性表明,该里德堡量子模拟器能够可靠捕捉受限多体动力学的核心行为。

% \subsection{弱热化与复苏寿命}


% 为定量表征 $\ket{\mathbb{Z}_2}$ 态的疤痕复苏与弱热化,本文测量里德堡布居 $P(\uparrow)$ 与平均畴壁密度(DWD)随时间的演化(图~\ref{ch5:fig:Z2_lifetime})。指数拟合给出正向—反向回波下的衰减率:布居的 $1/e$ 寿命约为 $\SI{1.6(1)}{\micro\second}$,平均畴壁密度的 $1/e$ 衰减时间约为 $\SI{1.0(3)}{\micro\second}$。仅正向演化下,平均畴壁密度可由阻尼正弦拟合得到约 $\SI{1.5(1)}{\micro\second}$ 的寿命,里德堡布居动力学可由阻尼傅里叶级数拟合得到约 $\SI{2.8(2)}{\micro\second}$ 的衰减时间。上述结果表明,在本文参数区间内,疤痕复苏在微秒量级时间窗口内保持可分辨振荡,其衰减既反映实验退相干与原子损失,也包含真实哈密顿量与有限尺寸下的非理想性贡献。

% 为了表征演化,测量了里德堡态布居 $P(\uparrow)$(图~\ref{ch5:fig:Z2_lifetime}a,b)和平均畴壁密度(图~\ref{ch5:fig:Z2_lifetime}c,d)。图~\ref{ch5:fig:Z2_lifetime}a,c 中数据的指数拟合显示了 $\ket{\mathbb{Z}_2}$ 态在正向-反向演化下的衰减率,布居的 $1/e$ 寿命约为 \SI{1.6(1)}{\micro\second},平均畴壁密度为 \SI{1.0(3)}{\micro\second}。这种衰减限制了正文图 3f 中呈现的原始 ZZ-OTOC 数据的对比度。平均畴壁密度的仅正向演化(图~\ref{ch5:fig:Z2_lifetime}d)拟合为阻尼正弦函数,产生约 \SI{1.5(1)}{\micro\second} 的 $\ket{\mathbb{Z}_2}$ 态寿命。里德堡布居动力学(图~\ref{ch5:fig:Z2_lifetime}b)使用阻尼傅里叶级数拟合,产生约 \SI{2.8(2)}{\us} 的 $1/e$ 衰减时间。  当初始化在 $\ket{\mathbb{Z}_2}$ 态时,里德堡原子系统表现出具有缓慢衰减的波函数振荡。这种衰减部分归因于实验演化过程中的态丢失和退相干,但也源于多体疤痕的不完美以及与基础 PXP 模型中热态的小重叠。为了更好地理解内在动力学对观察到的衰减的贡献,我们对复苏行为进行了数值模拟,将尺寸 $N=25$ 的系统初始化在 $\ket{\mathbb{Z}_2}$ 态。结果提供了证据表明观察到的衰减的很大一部分可以归因于 PXP 模型的内在特征。我们的系统中的演化误差源于三个主要来源:自发辐射引起的退极化、原子的有限温度和激光噪声。  
% \begin{figure}[htb] 
% \centering  
% \includegraphics[width=0.75\textwidth]{chapters/chapter_06/figure/SI_lifetime.png} 
% \caption{ \textbf{PXP 哈密顿量演化下的 $\ket{\mathbb{Z}_2}$ 态动力学。}  
% \textbf{a} 和 \textbf{b}, 初始化量子比特在 \textbf{a}, 正向-反向演化 \((e^{-iHt}\) 后跟 \(e^{iHt})\) 和 \textbf{b}, 仅正向 \((e^{-iHt})\) 演化期间的 \(\ket{\uparrow}\)(蓝色)和 \(\ket{\downarrow}\)(红色)布居。 \textbf{a} 中的曲线代表指数拟合,而 
% \textbf{b} 用高达 5 阶的阻尼傅里叶级数拟合。 
% \textbf{c} 和 \textbf{d}, 中心 13 个量子比特在 \textbf{c}, 正向-反向演化和 \textbf{d}, 仅正向演化期间的平均畴壁密度 (DWD)。\textbf{c} 中的蓝色曲线是指数拟合,而 \textbf{d} 拟合为阻尼正弦函数。  }   
% \label{ch5:fig:Z2_lifetime} 
% \end{figure}

% 为使“衰减机制分析”更完整,应在此处补充一个简洁的误差来源清单(例如:激光相位/强度噪声、原子温度与位置不确定性、长程相互作用修正、里德堡态有限寿命与探测误差等),并明确它们对复苏对比度与寿命的主导关系;若已有独立的噪声表征数据(如 Ramsey 相干时间 $T_2^*$),可在此处与复苏寿命做数量级对照。

% \section{本章小结}
% 本章围绕“动力学约束如何支撑量子多体疤痕”建立了从里德堡原子阵列到 PXP 模型的理论与实验闭环:通过里德堡阻塞诱导的受限希尔伯特子空间获得有效 PXP 动力学;在固定几何与长程相互作用背景下,通过选择合适的 $\Omega$ 与 $\Delta$ 工作点并关注中心 13 个量子比特,尽可能逼近体 PXP 行为;利用 SLM 寻址方案高保真制备 $\ket{\mathbb{Z}_2}$ 疤痕态,并以里德堡布居与畴壁密度的复苏与衰减定量表征弱热化动力学。

% 传统 QMBS 研究主要关注波函数重叠或局域可观测量的复苏。基于本章建立的受限多体动力学平台,一个自然的问题是:若在局域编码量子信息,它在多体相互作用与动力学约束共同作用下会如何传播与演化?第六章将引入更直接面向量子信息的观测量,对疤痕态背景下的时空动力学展开讨论。

\end{document}
