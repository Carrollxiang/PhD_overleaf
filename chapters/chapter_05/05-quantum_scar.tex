\documentclass[../main.tex]{subfiles}
\begin{document}

\chapter{里德堡原子阵列中的量子多体疤痕}

\section{引言}
遍历量子多体系统通常表现为快速热化,其长期行为可由本征态热化假说(eigenstate thermalization hypothesis, ETH)理解\cite{deutsch1991quantum,srednicki1994chaos,rigol2008thermalization,deutsch2018eigenstate,lewis2019dynamics}。在这一图景下,局域可观测量在有限时间内趋近于热平衡值,初态信息仅以少数守恒量的形式保留。然而,近年的理论与实验研究表明,某些相互作用多体系统能够在远长于微观时间尺度的区间内维持显著的相干振荡与弱热化行为,从而体现各态历经性破缺的动力学结构\cite{bernien2017probing,turner2018weak,choi2019emergent,ho2019periodic,bluvstein2021controlling,turner2021correspondence,serbyn2021quantum,desaules2024ergodicity}。量子多体疤痕(quantum many-body scars, QMBS)为其中一类代表机制:在能谱中存在一组与热本征态共存的非热本征态,使得特定低熵初态(例如反铁磁尼尔态)与这些本征态具有异常大的重叠,从而产生大振幅、缓慢衰减的复苏振荡。

可编程里德堡原子阵列为研究 QMBS 提供了可控且可扩展的平台\cite{kumar2018sorting,sheng2022defect,graham2022multi,barnes2022assembly,kim2022rydberg,okuno2022high,chen2023continuous,liu2023realization,singh2023mid,eckner2023realizing,ma2023high,zhao2023floquet,bluvstein2024logical,gregory2024second,pause2024supercharged,shaw2024benchmarking}。里德堡原子之间的强范德瓦尔斯相互作用诱导里德堡阻塞机制\cite{lukin2001dipole,urban2009observation_combined},抑制相邻格点的同时激发,从而对局域自旋翻转施加动力学约束。在适当参数区域,该受限动力学可由 PXP 模型近似描述\cite{lesanovsky2012interacting,turner2018weak},并天然支持从 $\ket{\mathbb{Z}_2}$ 初态出发的疤痕复苏动力学。

本章目标是建立从里德堡原子阵列到 PXP 模型的理论对应关系,并在实验上通过参数优化与局域寻址制备高保真度的 $\ket{\mathbb{Z}_2}$ 疤痕态,随后用局域可观测量的复苏与寿命表征动力学受限系统中的弱热化行为。第六章将在此实验平台基础上进一步研究受限多体动力学中的量子信息问题。

\todo[inline, color=blue!20]{深入拓展 QMBS 理论:详细论述本征态热化假说 (ETH) 的破缺,引入量子多体疤痕的严格定义;解释为何反铁磁尼尔态$\ket{\mathbb{Z}_2}$会与特定的非热本征态(Scar states)产生异常大的重叠(可加入能谱与纠缠熵特征的文献综述)}

\section{动力学受限与 PXP 模型}
\subsection{里德堡哈密顿量与阻塞约束}
本文考虑由二维光镊俘获并重排得到的一维 $^{87}$Rb 原子链(实验平台与通用操控见第 3、4 章)。每个原子编码为二能级量子比特:基态 $\ket{\downarrow}$ 与里德堡态 $\ket{\uparrow}$。系统由微观里德堡哈密顿量支配:
\begin{equation}
\hat{H}_\text{R}=
\frac{\Omega}{2}\sum_i\hat{\sigma}^x_i-
\Delta\sum_i\hat{n}_i+
\sum_{i<j}V_{ij}\hat{n}_i\hat{n}_j,
\label{ch5new:eq:rydberg_hamiltonian}
\end{equation}
其中 $\Omega$ 为拉比频率,$\Delta$ 为失谐,$\hat{n}_i=\ket{\uparrow_i}\bra{\uparrow_i}$ 为里德堡布居算符,$V_{ij}\sim C_6/R_{ij}^6$ 表示范德瓦尔斯相互作用。强最近邻相互作用(里德堡阻塞)使得相邻激发组态 $\ket{\cdots\uparrow_i\uparrow_{i+1}\cdots}$ 在能量上被强烈抬高,并在低能有效动力学中被压制。

\subsection{有效 PXP 模型与对称性}
在 $\Omega\ll V_{i,i+1}$ 且可忽略长程相互作用的理想极限下,哈密顿量可通过施里弗—沃尔夫变换投影到受限子空间,并化为有效 PXP 哈密顿量:
\begin{equation}
\hat{H}_\text{PXP}=\sum_i \hat{P}_i\hat{\sigma}^x_{i+1}\hat{P}_{i+2},
\label{ch5new:eq:pxp}
\end{equation}
其中 $\hat{P}_i=(1-\hat{\sigma}^z_i)/2$ 为格点 $i$ 上向 $\ket{\downarrow}$ 的投影算符。局域三体项 $\hat{P}_i\hat{\sigma}^x_{i+1}\hat{P}_{i+2}$ 体现动力学约束:仅当相邻原子均处于 $\ket{\downarrow}$ 时,格点 $i+1$ 才允许翻转。等价地,受限子空间可用投影算符
\begin{equation}
\mathcal{P}=\prod_j\left(1-\hat{n}_j\hat{n}_{j+1}\right)
\label{ch5new:eq:projector}
\end{equation}
表征,其有效希尔伯特空间维数按斐波那契数列增长,近似缩放为 $\phi^N$,其中 $\phi=(1+\sqrt{5})/2$。

PXP 模型具有离散空间反演对称性 $\mathcal{I}$($j\to N-j+1$)以及粒子—空穴对称性 $\mathcal{C}=\prod_j \sigma_j^z$,满足 $\mathcal{C}\hat{H}_\text{PXP}\mathcal{C}=-\hat{H}_\text{PXP}$。上述对称性与受限子空间结构共同使得从 $\ket{\mathbb{Z}_2}$ 初态出发的动力学呈现缓慢衰减的复苏振荡\cite{bernien2017probing,turner2018weak,serbyn2021quantum,bluvstein2021controlling,desaules2024ergodicity}。
% \todo[inline, color=blue!20]{PXP 模型的对称性与希尔伯特空间分析:补充受限子空间的维数按斐波那契数列增长(缩放为 $\phi^N$)的推导;详细写出离散空间反演、平移和粒子-空穴对称性 $\mathcal{C}$的数学表达}
这个受限子空间形成了维度降低的有效希尔伯特空间,支配着系统的受限动力学。值得注意的是,这个有效希尔伯特空间的维度按照斐波那契数列增长,缩放为 $\phi^N$,其中 $\phi$ 是黄金分割比,$N$ 是系统尺寸。这种维度降低反映了由于动力学约束而排除某些配置。  PXP 模型表现出三种显著的对称性:(1) 离散空间反演对称性 $\mathbb{Z}_2$,映射 $i \to N-i+1$。(2) 平移对称性,适用于周期性边界条件。(3) 粒子-空穴对称性,由 $\mathcal{P}$ 表示,导致 $H \to -H$。粒子-空穴对称性在我们实验中反转哈密顿量演化 $e^{-iHt} \to e^{iHt}$ 中起着至关重要的作用。然而,这种对称性仅存在于 PXP 哈密顿量 $H_{\text{PXP}}$ 中,而不存在于支配实验的完整里德堡哈密顿量 $H_{\text{Ryd}}$ 中,这导致了时间反演的不完美。 

在本章后续实验中,本文以 $\ket{\mathbb{Z}_2}$ 初态为核心研究对象,并通过局域可观测量的时间演化来表征疤痕复苏与弱热化行为。
\todo[inline, color=blue!20]{对于“为何 $\ket{\mathbb{Z}_2}$ 对疤痕子空间具有异常大重叠”这一更深入的本征态结构解释,可在此处补充简要谱分解或引用关键文献并给出物理图像。}

\section{数值模拟方法}
% \todo[inline, color=blue!20]{1. ED 与 MPO 方法:详细说明对 $N\leq13$ 使用精确对角化,对 N=25 使用矩阵乘积算符 (MPO/TeNPy) 方法;}
% \todo[inline, color=blue!20]{2. 含噪哈密顿量建模:写出考虑了 $\delta\Omega$ (拉比频率波动)、$\delta\varphi$ (激光相位噪声)、原子位置不确定性等实验不完美的蒙特卡罗 (Monte Carlo) 抽样模拟方法}

对于所有数值模拟,我们采用里德堡哈密顿量 (\ref{eq:rydberg_hamiltonian}) 来紧密近似实验物理条件。对于长度 $N \leq 13$ 的原子链,我们利用精确对角化来高效计算完整的时间演化。然而,对于超过 13 个原子的链,由于指数增加的内存需求和计算时间,以及可用计算资源的限制,精确对角化在计算上变得不可行。因此,我们在这些情况下实施矩阵乘积算符 (MPO) 方法来加速我们的数值计算。这种方法允许我们将模拟扩展到更长的原子链,同时利用 TeNPy 保持计算可行性和数值精度。  为了考虑实验不确定性,我们实施蒙特卡罗方法:考虑拉比频率和失谐的波动、激光噪声、原子位置的不确定性以及其他相关实验参数,所有变量均从其各自的概率分布中随机采样。通常,我们执行 200 次运行并平均结果。这种综合方法提供了在现实实验条件下系统行为的稳健表示。

\section{实验系统的参数设计与优化}
本章实验基于第 3、4 章所述可编程里德堡原子阵列平台。本文制备多达 25 个原子的无缺陷一维链,并将原子间距设置为 $a\approx\SI{7}{\micro\meter}$。该几何下最近邻与次近邻相互作用强度分别为 $V_{i,i+1}=2\pi\times\SI{7.3}{\MHz}$ 与 $V_{i,i+2}=2\pi\times\SI{0.11}{\MHz}$,其比值在装置中基本固定。

\subsection{逼近理想 PXP 的工作点选择}
真实装置中次近邻相互作用 $V_{i,i+2}$ 不能完全忽略。为了在固定的相互作用比值下尽可能逼近 PXP 模型,需要满足 $V_{i,i+2}\ll\Omega\ll V_{i,i+1}$,从而在 $V_{i,i+1}/\Omega$ 的选择上形成权衡。综合考虑后,本文在演化过程中选取基态—里德堡驱动拉比频率 $\Omega=V_{i,i+1}/6=2\pi\times\SI{1.21(1)}{\MHz}$,并引入小失谐 $\Delta=2\pi\times\SI{0.22(1)}{\MHz}$ 以部分补偿残余的次近邻相互作用 $V_{i,i+2}$。在该参数下,实验里德堡哈密顿量可由 PXP 模型较好近似。

在真实实验中,次近邻相互作用 $V_{i,i+2}$ 不能完全忽略。为在一维等间距几何下逼近 PXP 模型,需要满足 $V_{i,i+2} \ll \Omega \ll V_{i,i+1}$。由于 $V_{i,i+1}/V_{i,i+2}$ 在本文装置中固定约为 64,上述不等式对应一组不可同时极限满足的约束,从而在 $V_{i,i+1}/\Omega$ 的选择上形成权衡。

为确定最优比值,本文在不同参数下对 ZZ-OTOC 进行数值模拟,并以塌缩与复苏振荡对比度的衰减率作为优化指标。图~\ref{ch5new:fig:experimental-parameters}a 显示最佳选择为 $V_{i,i+1} = 6\Omega$,该结果不同于简单的理论中间值 $\Omega=\sqrt{V_{i,i+1} V_{i,i+2}}$。

拉比频率的选取需要在两个因素之间折衷:(1) 时间反演保真度:图~\ref{ch5new:fig:experimental-parameters}b 表明,当 $V_{i,i+1}/\Omega$ 固定为 6 且正向与反向演化间隔固定为 \SI{200}{\nano\second} 时,提高 $\Omega$ 会降低时间反演保真度;(2) 单原子相干性:更高的拉比频率有助于抑制激光相位噪声主导的相干性限制。综合考虑后,本文取 $\Omega = 2\pi \times \SI{1.21(1)}{MHz}$。

除优化 $V_{i,i+1}$ 与 $\Omega$ 外,本文引入小失谐 $\Delta$ 以抵消残余次近邻相互作用 $V_{i,i+2}$。数值模拟表明,$\Delta = 2V_{i,i+2}$ 时可较好再现 OTOC 的塌缩与复苏动力学(图~\ref{ch5new:fig:experimental-parameters}c)。

OTOC 测量中,正向与反向演化之间插入间隙以实施局域与全局单量子比特 $\sigma^z$ 操作。为将间隙时间压缩至 $\sim \SI{200}{ns}$,本文并行执行局域微扰与全局 $\sigma^z$ 旋转,并利用其对易关系。图~\ref{ch5new:fig:experimental-parameters}d 给出有限间隙对 OTOC 动力学的数值评估,结果表明本文采用的 \SI{200}{ns} 单量子比特操作时间不会显著改变塌缩与复苏振荡的可观测性。

综上,参数优化使本文实验在固有几何与相互作用约束下仍能紧密逼近 PXP 模型(图~\ref{ch5new:fig:experimental-parameters}d),并为高保真度地实现动力学约束下的相干自旋旋转与疤痕动力学提供支撑。

\begin{figure}[htb]
\centering
\includegraphics[width=0.7\textwidth]{chapters/chapter_06/figure/SI_experimental_parameters.png}
\caption{\textbf{优化实验参数以紧密近似理想 PXP 哈密顿量动力学。} 
具有周期性边界条件的 10 量子比特链中中心量子比特的 OTOC 和时间反演保真度的数值模拟。
\textbf{a}, 模拟的不同最近邻 (NN) 里德堡相互作用 $V_{\text{NN}}$ 下的 OTOC 动力学,与理想 PXP 模型进行比较。相互作用强度从 $2\pi \times \SI{6}{MHz}$ 变化到 $2\pi \times \SI{18}{MHz}$,步长为 $2\pi \times \SI{1.5}{MHz}$。最佳选择 $V_{\text{NN}} = 2\pi \times \SI{9}{MHz}$ 最匹配理想情况。 
\textbf{b}, 时间反演保真度作为拉比频率 $\Omega$ 的函数,其中 $V_{\text{NN}}/\Omega$ 固定为 6。显示了 $t=0.5\mu s$(蓝色符号)、$1.0\mu s$(黄色符号)和 $1.5\mu s$(红色符号)的不同时间演化。随着 $\Omega$ 增加,保真度降低。 
\textbf{c}, 不同失谐值 $\delta$ 下的 OTOC 动力学,范围从 $-0.3 V_{\text{NNN}}$ 到 $0.3 V_{\text{NNN}}$,其中 $V_{\text{NNN}}$ 是次近邻 (NNN) 相互作用。黑色曲线代表理想 PXP 模型。最佳失谐 $\delta = 0.1 V_{\text{NNN}}$ 最匹配理想动力学。 \textbf{d}, 使用优化实验参数(紫色曲线)和理想 PXP 模型(黑色曲线)的 OTOC 动力学比较,显示出紧密一致。优化参数有效地再现了 OTOC 中观察到的关键振荡。} 
\label{ch5new:fig:experimental-parameters}
\end{figure}


% \todo[inline, color=blue!20]{1. 详细讨论为何在 $V_{i,i+1}/V_{i,i+2}\simeq64$ 下,选择 $V_{i,i+1}=6\Omega$ 是最佳折中;}
% \todo[inline, color=blue!20]{2. 讨论如何通过模拟引入失谐 $Δ=2V_{i,i+2}$ 来补偿长程尾巴;}
% \todo[inline, color=blue!20]{3. 展示不同参数下时间反演保真度的演化图表}


有效 PXP 哈密顿量的近似依赖于两个关键假设:(1) 忽略长程相互作用 $V_{i,j>i+1}$。(2) 确保最近邻相互作用主导拉比频率($V_{i,i+1} \gg \Omega$)。然而,实际上,次近邻相互作用 $V_{i,i+2}$ 不能被忽略。为了紧密近似 PXP 模型,系统必须在 $V_{i,i+1}/V_{i,i+2} \gg 1$ 的区域运行。这是一个挑战,因为在我们具有等间距原子的一维几何结构中,比率 $V_{i,i+1}/V_{i,i+2}$ 固定在约 64。例如,设置 $V_{i,i+1}/\Omega = 3$ 以满足第二个假设会导致 $V_{i,i+2}/\Omega \approx 0.05$,使得难以完全抑制 $V_{i,i+2}$ 的影响。这在两个关键假设之间产生了权衡。  为了找到 $V_{\text{NN}}/\Omega$ 的最佳比率,我们对不同参数下的 ZZ-OTOC 进行了数值模拟。我们确定了最小化塌缩与复苏振荡对比度衰减的最佳比率,如图~\ref{FIG:experimental-parameters}a 所示。基于这些结果,我们将 $V_{\text{NN}}/\Omega$ 的实验比率设置为 6,这与理论中间值 $3$ 不同。  拉比频率的选择涉及平衡两个相互竞争的因素:(1) 时间反演保真度:我们的模拟(图~\ref{FIG:experimental-parameters}b)表明,当 $V_{\text{NN}}/\Omega$ 固定为 6 且正向和反向演化之间的间隔固定为 \SI{200}{\nano\second} 时,增加拉比频率会降低时间反演保真度。(2) 单原子相干性:较高的拉比频率提高了主要受激光相位噪声限制的单原子相干性。在仔细权衡这些因素后,我们选择了 $\Omega = 2\pi \times \SI{1.5}{MHz}$ 的拉比频率。  除了优化 $V_{\text{NN}}$ 和 $\Omega$ 外,我们还引入了一个小的失谐 $\delta$ 来抵消残留的次近邻相互作用 $V_{\text{NNN}}$。我们的模拟表明,设置 $\delta = 0.1 V_{\text{NNN}}$ 最能捕捉 OTOC 动力学(图~\ref{FIG:experimental-parameters}c)。  

\subsection{边界与有限尺寸效应}
有限尺寸量子比特链中,边界量子比特缺乏部分相邻约束,从而使边缘与体量子比特的有效驱动环境不同,并引入显著边界效应。数值与实验结果均显示,边界影响会从外向内传播并逐步弯曲波前,进而影响对体动力学的表征。为减弱边界与有限尺寸效应,本文采用 25 量子比特链并将动力学分析聚焦在中心 13 个量子比特区域,从而更好逼近体 PXP 动力学。类似的实验策略也可在后续章节的更精细测量中复用(此处不展开)。

\begin{figure}[htb]
\centering
\includegraphics[width=0.7\textwidth]{chapters/chapter_06/figure/SI_Boundary_and_finite-size_effect.png}
\caption{{\textbf{边界和有限尺寸效应。}
初始化在 $\ket{\mathbb{Z}_2}$ 态的各种系统尺寸和边界条件下理想里德堡哈密顿量的模拟 OTOC 动力学与时间演化。
\textbf{a}, 13 量子比特链中边界效应示意图。边界量子比特缺乏最近邻与次近邻,打破平移对称性。
\textbf{b} 和 \textbf{c}, 分别从 $\ket{\downarrow}$ 态与 $\ket{\uparrow}$ 态出发的 13 量子比特链中边缘(红色)与中心(蓝色)量子比特的 $\left\langle n_j(t) \right\rangle$ 数值结果。边缘量子比特的加速振荡反映边界效应。
\textbf{d} 和 \textbf{e}, 9 量子比特链(红色曲线)中边缘量子比特(\textbf{d})及其相邻量子比特(\textbf{e})的 OTOC 动力学,并与 25 量子比特链(蓝色曲线)中相应量子比特对比。插图给出偏差 $|\delta_\text{OTOC}|$,表明边界效应可传播到链内部。
\textbf{f}, 理想 PXP 哈密顿量(黄色),以及在含噪里德堡哈密顿量下使用 IZ-OTOC(蓝色)与边缘量子比特 ZZ-OTOC(红色)归一化的 OTOC 演化。
\textbf{g}, 归一化 OTOC 与理想 PXP 情况的偏差(红色:边缘 ZZ-OTOC;蓝色:IZ-OTOC)。
\textbf{h} 和 \textbf{i}, 开放边界条件下不同链长(5、9、13、25)中中心量子比特(\textbf{h})与最近邻量子比特(\textbf{i})的模拟 OTOC 动力学。
\textbf{j} 和 \textbf{k}, 10 量子比特链(周期性边界条件,PBC)中距微扰最远的量子比特与 25 量子比特链(开放边界条件,OBC)中边缘量子比特的 OTOC 动力学对比。}}
\label{ch5new:fig:boundary}
\end{figure}

% \todo[inline, color=blue!20]{1. 比较 L=5,9,13,25 时中心量子比特的动力学差异;}
% \todo[inline, color=blue!20]{2. 比较 OBC(开放边界,实验)与 PBC(周期性边界,理论)在链长增加时的收敛性;}
% \todo[inline, color=blue!20]{3. 数值证明边缘比特的加速振荡如何向内弯曲波前}

在有限尺寸量子比特链中,边界量子比特的存在打破了 PXP 哈密顿量对于某些初态如 $\ket{\mathbb{Z}_2} = \ket{...\uparrow \downarrow\uparrow\downarrow\uparrow...}$ 和 $\ket{\mathbf{0}} = \ket{...\downarrow \downarrow\downarrow\downarrow\downarrow...}$ 的平移对称性。这种对称性破缺导致边缘和体量子比特之间的相互作用强度不同(图~\ref{Fig:Boundary-and-finite-size-effect}a),引入显著的边界效应。  为了定量分析边界效应,我们数值模拟了 13 量子比特链中中心和边缘量子比特的演化动力学,从理想里德堡哈密顿量下的 $\ket{\mathbb{Z}_2}$ 态开始。如图~\ref{Fig:Boundary-and-finite-size-effect}b,c 所示,边界效应导致边缘量子比特表现出加速的周期性振荡。这种加速源于对边缘量子比特的约束减少,导致更强的有效驱动强度。此外,数值模拟和实验结果都表明,边界效应逐渐从外边缘向内改变自旋旋转,导致最初均匀传播的波前在 $\ket{\mathbb{Z}_2}$ 初态的演化过程中发生弯曲,强调了边界效应在塑造系统动力学中的关键作用。  我们通过比较不同长度(具有开放边界条件的 5、9、13 和 25 量子比特链)链中中心两个量子比特的 OTOC 动力学进一步研究有限尺寸效应。数值模拟表明,在较小的原子系统中,有限尺寸效应非常明显,如图~\ref{Fig:Boundary-and-finite-size-effect}h,i 所示。这表明我们需要延长链长以确保我们的实验观测准确反映量子信息的加扰。然而,即使使用 MPO 方法,模拟 25 个原子也需要大量的计算资源。为了解决这个挑战,我们模拟了具有开放边界条件 (OBC) 的 25 量子比特链和具有周期性边界条件 (PBC) 的 10 量子比特链,如图~\ref{Fig:Boundary-and-finite-size-effect}j,k 所示。我们比较了距离微扰最远的原子的 OTOC 动力学。结果显示在实验时间尺度内差异可忽略不计,因为在此期间原子不受两端等效局域微扰的影响。基于此分析,在模拟 OTOC 动力学时,我们采用具有 PBC 的 10 量子比特链的模拟。 

\begin{figure}[htb]
\centering 
\includegraphics[width=1.0\textwidth]{chapters/chapter_06/figure/SI_Boundary_and_finite-size_effect.png} 
\caption{\textbf{边界和有限尺寸效应。} 初始化在 $\ket{\mathbb{Z}_2}$ 态的各种系统尺寸和边界条件下理想里德堡哈密顿量下的模拟 OTOC 动力学和时间演化。 
\textbf{a}, 13 量子比特链中边界效应的示意图。边界量子比特缺乏最近邻和次近邻,打破了平移对称性。 
\textbf{b} 和 \textbf{c}, 从 $\ket{\downarrow}$ 态 (\textbf{b}) 和 $\ket{\uparrow}$ 态 (\textbf{c}) 开始的 13 量子比特链中边缘(红色)和中心(蓝色)量子比特的 $\left\langle n_j(t) \right\rangle$ 数值结果。边缘量子比特的加速振荡反映了边界效应。 
\textbf{d} 和 \textbf{e}, {9 量子比特链(红色曲线)中边缘量子比特 (\textbf{d}) 及其相邻量子比特 (\textbf{e}) 的 OTOC 动力学,与 25 量子比特链(蓝色曲线)中的相应量子比特进行比较。每个图上方的示意图用彩色圆圈标记,说明了它们在各自链中的具体量子比特位置。插图显示了两种动力学之间的偏差 $|\delta_\text{OTOC}|$,表明显著的边界效应传播到链的内部。} 
\textbf{f}, 理想 PXP 哈密顿量(黄色)下,以及使用 IZ-OTOC(蓝色)和边缘量子比特的 ZZ-OTOC(红色)归一化的含噪里德堡哈密顿量下的 OTOC 演化。 
\textbf{g}, 归一化 OTOC 与理想 PXP 情况的偏差(红色:边缘 ZZ-OTOC,蓝色:IZ-OTOC)。在使用边缘量子比特的 ZZ-OTOC 归一化的情况下观察到显著偏差。 
\textbf{h} 和 \textbf{i}, 具有开放边界条件的各种长度(5, 9, 13, 和 25 量子比特)链中中心量子比特 (\textbf{h}) 和最近邻 (NN) 量子比特 (\textbf{i}) 的模拟 OTOC 动力学。较小系统中 OTOC 动力学的明显变化显示了有限尺寸效应。 
\textbf{j} 和 \textbf{k}, 10 量子比特链(周期性边界条件,PBC,深蓝色)中距离微扰最远的量子比特与 25 量子比特链(开放边界条件,OBC,浅蓝色)中边缘量子比特的 OTOC 动力学比较。可忽略的偏差 $|\delta_\text{OTOC}|$ 验证了使用 10 量子比特 PBC 系统来近似较大的 25 量子比特 OBC 系统中的体量子比特行为,符合我们的实验条件。} 
\label{Fig:Boundary-and-finite-size-effect} 
\end{figure}


\section{$\mathbb{Z}_2$ 疤痕态的高保真度制备与表征}
\subsection{寻址制备方案与动机}
在大规模系统中,通过绝热路径将设计哈密顿量基态从 $\ket{\mathbf{0}}$ 变换到 $\ket{\mathbb{Z}_2}$ 的方案会随系统尺寸增大而迅速失效:希尔伯特空间维数指数增长并导致能隙变小,从而使绝热制备保真度下降\cite{bernien2017probing,omran2019generation,bluvstein2021controlling}。为获得可扩展的高保真度 $\ket{\mathbb{Z}_2}$ 初态,本文将全局里德堡激发与格点选择性寻址相结合:空间光调制器(Spatial Light Modulator, SLM)在原子阵列上写入定制光频移图案,使指定格点对基态—里德堡跃迁失谐,从而形成可激发与不可激发原子的交替排列。

\subsection{保真度与误差分解}
% \begin{figure}[htb]
% \centering
% \includegraphics[width=0.7\textwidth]{chapters/chapter_06/figure/SI_Z2.png}
% \caption{\textbf{$\ket{\mathbb{Z}_2}$ 态制备与误差表征。} 图中展示寻址诱导失谐对里德堡激发谱与制备保真度的影响,以及微观态统计分布与主要错误通道。}
% \label{ch5new:fig:z2prep}
% \end{figure}

% \todo[inline, color=blue!20]{1. 展示反阻塞效应模拟图,解释寻址光频移方向的选择}
% \todo[inline, color=blue!20]{2. 列出 13 比特微观态的统计分布直方图,明确指出 86\% 的误差源于 $\ket{\uparrow}\rightarrow\ket{\downarrow}$ 单比特翻转}

795-\SI{}{\nano\meter} 寻址光束,相对于 D1 线共振失谐 $2\pi \times \SI{15}{\GHz}$,在基态上诱导 $2\pi \times \SI{12.2(3)}{MHz}$ 的能量移动。由 795-\SI{}{\nano\meter} 寻址光束引起的散射率约为 $2\pi \times \SI{5}{kHz}$,明显低于拉曼激光束引起的散射率。数值模拟(图~\ref{Fig:S2}b)表明,为了最大化 $\ket{\mathbb{Z}_2}$ 态的保真度,光频移的符号必须与里德堡相互作用的符号相反,以避免反阻塞效应。在我们的实验条件下,数值模拟产生的制备非保真度约为每个量子比特 0.012。  

实验制备的 $\ket{\mathbb{Z}_2}$ 态的微观态分布分析揭示了非泊松误差发生(图~\ref{Fig:S2}d)。我们对微观态分布的测量表明,$\ket{\mathbb{Z}_2}$ 态制备中的误差主要归因于单量子比特 $\ket{\uparrow}\rightarrow\ket{\downarrow}$ 翻转。这些误差态占所有与 $\ket{\mathbb{Z}_2}$ 态相关的出现次数的 86(2)\%(图~\ref{Fig:S2}d,e)。其余的误差贡献主要源于探测误差(每个量子比特约 1\%)。图~\ref{Fig:S2}e 显示了测量的微观态分布,突显了主要误差是从 $\ket{\uparrow}$ 到 $\ket{\downarrow}$ 的单量子比特翻转,进一步证实了我们的方法采用单量子比特操作来制备多体态。这种可预测的误差分布有助于在 $\ket{\mathbb{Z}_2}$ 态中量子信息加扰的实验测量数据中进行误差缓解。

\begin{figure}[htb]   
\centering   
\includegraphics[width=\textwidth]{chapters/chapter_06/figure/SI_Z2.png}  
\caption{     
\textbf{$\ket{\mathbb{Z}_2}$ 态制备细节。}     
\textbf{a}, {被寻址(红色)和未被寻址(蓝色)原子的里德堡激发光谱。里德堡布居显示为拉曼激发激光失谐的函数。}     
\textbf{b}, {反阻塞效应和寻址激光诱导的光频移对态制备保真度的影响。模拟的 $\ket{\mathbb{Z}_2}$ 态制备保真度作为 795-\SI{}{\nano\meter} 寻址激光光频移(以里德堡拉曼激发拉比频率 $\Omega$ 为单位)的函数,针对各种系统尺寸,最近邻里德堡相互作用强度 $V_{i,i+1}=3\Omega$。红星标记了我们的实验条件。}     
\textbf{c}, {$\ket{\mathbb{Z}_2}$ 态制备保真度作为系统尺寸的函数。红线和蓝线:校正和未校正 $\ket{\mathbb{Z}_2}$ 态保真度的指数拟合。黄线:当前实验寻址方法的理论上限。}     
\textbf{d}, {13 量子比特微观态数量作为来自 1,774 次实验的出现次数的函数。成功制备的 $\ket{\mathbb{Z}_2}$ 态:1,246 次计数(70\% 的事件)。}     
\textbf{e}, {测量的 13 量子比特微观态分布。插图:主要错误态。}   
}   
\label{Fig:S2} 
\end{figure}


% \SI{795}{\nano\meter} 寻址光束相对于 D1 线共振失谐 $2\pi\times\SI{15}{\GHz}$,在基态上诱导 $2\pi\times\SI{12.2(3)}{\MHz}$ 的能级移动,其散射率约为 $2\pi\times\SI{5}{\kHz}$。数值模拟表明,为最大化 $\ket{\mathbb{Z}_2}$ 态制备保真度,光频移符号需与里德堡相互作用符号相反以避免反阻塞效应。在本文实验条件下,模拟得到的制备非保真度约为每个量子比特 0.012。

该方案在系统尺寸扩展时保持良好性能:中心 13 个量子比特的 $\ket{\mathbb{Z}_2}$ 态测量保真度为 70(1)\%,探测误差校正后为 78(1)\%;整个 25 量子比特链的测量保真度为 49(3)\%,校正后为 60(3)\%。误差分解表明,制备非保真度主要来自不相关的单量子比特翻转误差,其中主要错误通道为 $\ket{\uparrow}\rightarrow\ket{\downarrow}$。该微观态误差结构一方面用于评估本章动力学观测的可靠性,另一方面也为下一章更精细的误差分析提供输入(此处不展开)。

\section{动力学受限系统中的疤痕演化与弱热化观测}
\subsection{受限多体动力学的时空图像}
% \begin{figure}[htb]
% \centering
% \includegraphics[width=0.7\textwidth]{chapters/chapter_06/figure/Z2.png}
% \caption{\textbf{态制备与受限多体动力学。} 图中对比 $\ket{\mathbb{Z}_2}$ 与 $\sigma_c^x\ket{\mathbb{Z}_2}$ 两类初态在受限动力学下的格点分辨里德堡布居演化,并体现边界与局域微扰诱导的迟滞波前结构。}
% \label{ch5new:fig:dynamics}
% \end{figure}

\begin{figure}[htb]
  \centering
  \includegraphics[width=\textwidth]{chapters/chapter_06/figure/Z2.png}
  \caption{
\textbf{态制备与动力学受限多体态演化。} 
\textbf{a}, 在全局相干驱动下,795-\unit{\nano\meter} 激光寻址的原子(红色)保持静止,而未寻址的原子(蓝色)表现出基态-里德堡态拉比振荡。
\textbf{b}, $\ket{\mathbb{Z}_2}$ 态制备保真度 $\mathcal{F}_{\mathbb{Z}_2}$ 作为系统尺寸的函数。
\textbf{c}, 测量的 25 量子比特 $\ket{\mathbb{Z}_2}$ 态的格点分辨里德堡概率 $P_i(\uparrow)$。上图:$\ket{\mathbb{Z}_2}$ 态的原子荧光图示,其中红圈表示里德堡原子。
\textbf{d}--\textbf{g}, 动力学受限多体动力学。
\textbf{d} 和 \textbf{f}, 分别从初态 $\ket{\mathbb{Z}_2}$ 和 $\hat\sigma^x_c\ket{\mathbb{Z}_2}$ 开始,测量的作为演化时间函数的格点分辨里德堡态概率 $P_i(\uparrow)$。
在 $\hat\sigma^x_c\ket{\mathbb{Z}_2}$ 情况 (\textbf{f}) 中观察到线性光锥结构,其起源在第~\ref{section:boundary} 节中详述。
\textbf{e} 和 \textbf{g} 是来自数值模拟的相应结果。
这些模拟采用了完整的里德堡哈密顿量,并根据我们对这些噪声源的表征考虑了实验噪声(见方法和补充材料第 4.3 节)。
\textbf{d}--\textbf{g} 中的红色虚线曲线展示了传播波前,而 \textbf{f} 和 \textbf{g} 中的蓝色虚线显示了线性光锥。}
\label{ch5new:fig:dynamics}
\end{figure}

如图~\ref{ch5new:fig:dynamics} 所示,本文比较了 PXP 有效动力学下 $\ket{\mathbb{Z}_2}$ 与 $\sigma_c^x\ket{\mathbb{Z}_2}$ 两类自旋配置的演化差异。对于 $\ket{\mathbb{Z}_2}$ 初态,自旋在约束下表现出近似同步的相干旋转;受边界效应影响,边缘处动力学差异逐步向内传播并弯曲波前。对于 $\sigma_c^x\ket{\mathbb{Z}_2}$ 初态,中心翻转引入局域迟滞并形成清晰的线性光锥结构,反映动力学约束对局域扰动传播的重塑。实验数据与数值结果的一致性表明,该里德堡量子模拟器能够可靠捕捉受限多体动力学的核心行为。

\subsection{弱热化与复苏寿命}
% \begin{figure}[htb]
% \centering
% \includegraphics[width=0.7\textwidth]{chapters/chapter_06/figure/SI_lifetime.png}
% \caption{\textbf{$\ket{\mathbb{Z}_2}$ 态的复苏与衰减表征。} 图中给出 $\ket{\uparrow}$、$\ket{\downarrow}$ 布居与平均畴壁密度(DWD)在正向演化与正向—反向回波序列下的振荡与衰减,用于量化疤痕复苏的可观测窗口。}
% \label{ch5new:fig:lifetime}
% \end{figure}

为定量表征 $\ket{\mathbb{Z}_2}$ 态的疤痕复苏与弱热化,本文测量里德堡布居 $P(\uparrow)$ 与平均畴壁密度(DWD)随时间的演化(图~\ref{ch5new:fig:lifetime})。指数拟合给出正向—反向回波下的衰减率:布居的 $1/e$ 寿命约为 $\SI{1.6(1)}{\micro\second}$,平均畴壁密度的 $1/e$ 衰减时间约为 $\SI{1.0(3)}{\micro\second}$。仅正向演化下,平均畴壁密度可由阻尼正弦拟合得到约 $\SI{1.5(1)}{\micro\second}$ 的寿命,里德堡布居动力学可由阻尼傅里叶级数拟合得到约 $\SI{2.8(2)}{\micro\second}$ 的衰减时间。上述结果表明,在本文参数区间内,疤痕复苏在微秒量级时间窗口内保持可分辨振荡,其衰减既反映实验退相干与原子损失,也包含真实哈密顿量与有限尺寸下的非理想性贡献。
% \todo[inline, color=blue!20]{详细给出 DWD(平均畴壁密度)和 P(↑) 演化的阻尼正弦与傅里叶级数拟合过程,提取 1.6 $\mu$s 等寿命数据,并与数值模拟中的内禀衰减做对比}
为了表征演化,测量了里德堡态布居 $P(\uparrow)$(图~\ref{Fig:S3}a,b)和平均畴壁密度(图~\ref{Fig:S3}c,d)。图~\ref{Fig:S3}a,c 中数据的指数拟合显示了 $\ket{\mathbb{Z}_2}$ 态在正向-反向演化下的衰减率,布居的 $1/e$ 寿命约为 \SI{1.6(1)}{\micro\second},平均畴壁密度为 \SI{1.0(3)}{\micro\second}。这种衰减限制了正文图 3f 中呈现的原始 ZZ-OTOC 数据的对比度。平均畴壁密度的仅正向演化(图~\ref{Fig:S3}d)拟合为阻尼正弦函数,产生约 \SI{1.5(1)}{\micro\second} 的 $\ket{\mathbb{Z}_2}$ 态寿命。里德堡布居动力学(图~\ref{Fig:S3}b)使用阻尼傅里叶级数拟合,产生约 \SI{2.8(2)}{\us} 的 $1/e$ 衰减时间。  当初始化在 $\ket{\mathbb{Z}_2}$ 态时,里德堡原子系统表现出具有缓慢衰减的波函数振荡。这种衰减部分归因于实验演化过程中的态丢失和退相干,但也源于多体疤痕的不完美以及与基础 PXP 模型中热态的小重叠。为了更好地理解内在动力学对观察到的衰减的贡献,我们对复苏行为进行了数值模拟,将尺寸 $N=25$ 的系统初始化在 $\ket{\mathbb{Z}_2}$ 态。结果提供了证据表明观察到的衰减的很大一部分可以归因于 PXP 模型的内在特征。我们的系统中的演化误差源于三个主要来源:自发辐射引起的退极化、原子的有限温度和激光噪声。  
\begin{figure}[htb] 
\centering  
\includegraphics[width=0.75\textwidth]{chapters/chapter_06/figure/SI_lifetime.png} 
\caption{ \textbf{PXP 哈密顿量演化下的 $\ket{\mathbb{Z}_2}$ 态动力学。}  
\textbf{a} 和 \textbf{b}, 初始化量子比特在 \textbf{a}, 正向-反向演化 \((e^{-iHt}\) 后跟 \(e^{iHt})\) 和 \textbf{b}, 仅正向 \((e^{-iHt})\) 演化期间的 \(\ket{\uparrow}\)(蓝色)和 \(\ket{\downarrow}\)(红色)布居。 \textbf{a} 中的曲线代表指数拟合,而 
\textbf{b} 用高达 5 阶的阻尼傅里叶级数拟合。 
\textbf{c} 和 \textbf{d}, 中心 13 个量子比特在 \textbf{c}, 正向-反向演化和 \textbf{d}, 仅正向演化期间的平均畴壁密度 (DWD)。\textbf{c} 中的蓝色曲线是指数拟合,而 \textbf{d} 拟合为阻尼正弦函数。  }   
\label{Fig:S3} 
\end{figure}

为使“衰减机制分析”更完整,应在此处补充一个简洁的误差来源清单(例如:激光相位/强度噪声、原子温度与位置不确定性、长程相互作用修正、里德堡态有限寿命与探测误差等),并明确它们对复苏对比度与寿命的主导关系;若已有独立的噪声表征数据(如 Ramsey 相干时间 $T_2^*$),可在此处与复苏寿命做数量级对照。%待补充

\section{本章小结}
本章围绕“动力学约束如何支撑量子多体疤痕”建立了从里德堡原子阵列到 PXP 模型的理论与实验闭环:通过里德堡阻塞诱导的受限子空间获得有效 PXP 动力学;在固定几何与长程相互作用背景下,通过选择合适的 $\Omega$ 与 $\Delta$ 工作点并关注中心 13 个量子比特,尽可能逼近体 PXP 行为;利用 SLM 寻址方案高保真制备 $\ket{\mathbb{Z}_2}$ 疤痕态,并以里德堡布居与畴壁密度的复苏与衰减定量表征弱热化动力学。

传统 QMBS 研究主要关注波函数重叠或局域可观测量的复苏。基于本章建立的受限多体动力学平台,一个自然的问题是:若在局域编码量子信息,它在多体相互作用与动力学约束共同作用下会如何传播与演化?第六章将引入更直接面向量子信息的观测量,对疤痕态背景下的时空动力学展开讨论。

\end{document}
