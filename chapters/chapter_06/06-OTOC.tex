% chapters/01-introduction.tex
\documentclass[../main.tex]{subfiles}
\begin{document}
% 章节内容...
\chapter{基于序时关联函数的量子信息加扰与复苏观测}

量子信息加扰(Scrambling)描述了局部量子信息在多体系统中随时间扩散至非局域自由度的过程,是理解量子混沌与热化的核心概念。序时关联函数(OTOC)是诊断这一过程的关键探针,但其测量需要实现复杂的多体哈密顿量时间反演,这对实验控制提出了极高要求。 本章利用上一章搭建的里德堡原子量子模拟器,通过精确控制光场相位与失谐,成功实现了多体动力学的时间反演,并首次在受约束的量子多体系统中测量了 ZZ-OTOC 动力学。为了消除实验噪声的影响,我们发展了一套基于辅助测量(IZ-OTOC)的误差缓解方案。实验观测表明,当系统处于热化的 0 态时,OTOC 快速衰减,标志着信息的快速加扰;而当系统处于$\mathbb{Z}_2$疤痕态时,OTOC 展现出独特的“塌缩-复苏”(Collapse-and-Revival)行为。结合上一章的 Holevo 信息结果,我们证明了这种复苏并非简单的波函数重现,而是反映了量子信息在受限希尔伯特空间中的周期性重聚。该工作建立了一种在含噪中性原子平台上探究复杂量子信息动力学的通用方法。
\end{document}
