% chapters/01-introduction.tex
\documentclass[../main.tex]{subfiles}
\begin{document}
% 章节内容...

\chapter{实验装置与控制系统}
\label{chap:experimental_setup}

在前一章中,我们讨论了中性原子俘获与操控的物理原理。本章将详细阐述为了实现这些物理目标而搭建的实验硬件平台。我们的实验系统采用紧凑的双真空腔体设计,结合了用于产生大规模静态阵列的空间光调制器(SLM)技术和用于实现单原子动态重排的声光偏转(AOD)技术。此外,为了克服传统成像技术中的加热问题,我们开发了一套基于 795 nm D1 线 $\Lambda$-增强型灰莫拉斯(Gray Molasses)机制的成像系统,实现了高保真度的原位探测。本章将按照“真空系统 $\to$ 光学系统 $\to$ 阵列生成 $\to$ 探测与控制”的逻辑顺序,对各个子系统进行逐一介绍。

\section{真空系统设计}
\label{sec:vacuum_system}

为了在保持较长原子寿命的同时实现快速的实验重复频率,真空系统必须满足两个看似矛盾的要求:科学腔(Science Chamber)需要维持极低的背景气压($< 10^{-10}$ Torr)以抑制背景碰撞损失,而原子源区域则需要维持较高的铷原子蒸气压以保证足够的装载通量。为此,我们设计了一套双腔差分泵浦系统,通过流阻极大的细长管道将两个压力区隔离开来。

\subsection{双腔差分系统架构}
\label{subsec:vacuum_architecture}

我们的真空系统采用垂直堆叠结构。上层为 2D MOT 源腔,用于从铷金属源中产生初级冷原子束;下层为 3D MOT 科学腔,是进行单原子俘获与量子模拟实验的核心区域。两层腔体通过一个精密的差分管(Differential Pumping Tube)连接。该设计不仅有效地隔离了两个区域的真空环境,还利用重力辅助了原子的输运过程。

% [PLACEHOLDER: 科学内容填充]
% 1. [图表] 真空系统整体 CAD 渲染图(侧视),标注上层源腔、中间差分管、下层科学腔及离子泵位置。
% 2. 描述上层 2D MOT 腔体的尺寸与材料(玻璃/不锈钢)。
% 3. 描述下层科学腔的设计:光学窗口的镀膜参数(针对 780nm, 795nm, 1064nm, 480nm 等波长优化)及数值孔径(NA)限制。
% 4. 差分管的流阻计算:根据管道长度和直径估算压差比(Pressure ratio)。

\subsection{原子源与输运}
\label{subsec:atom_source}

原子源的稳定性直接决定了实验的占空比。我们在上层腔体中配置了可独立控制电流的铷金属分配器(Dispenser)。通过 2D MOT 的冷却,热原子被横向压缩并沿差分管轴向推出,形成高通量的冷原子束流。为了进一步提高传输效率,我们引入了一束沿轴向传播的推原子光(Push Beam),将原子从 2D MOT 高效地“抛射”至下方的 3D MOT 俘获区。

% [PLACEHOLDER: 科学内容填充]
% 1. Rb 分配器的型号与电流工作点。
% 2. 2D MOT 的磁场构型(永磁体或线圈)及冷却光路配置。
% 3. Push 光的参数:失谐量、功率、光斑大小及对原子束通量的影响数据。

\section{激光与光学系统}
\label{sec:optical_systems}

本实验涉及多个波段的激光系统,包括用于冷却和态制备的 780 nm 光源、用于 D1 线成像的 795 nm 光源,以及用于产生光镊势阱的远红外光源。为了保证激光频率和功率的长期稳定性,我们搭建了基于外腔半导体激光器(ECDL)和锥形放大器(TA)的稳频与放大链路。

\subsection{冷却与俘获光源 (780 nm)}
\label{subsec:cooling_laser}

780 nm 激光对应于 $^{87}\text{Rb}$ 的 D2 跃迁线 ($5S_{1/2} \to 5P_{3/2}$),承载了 MOT 冷却、光抽运(Optical Pumping)以及里德堡激发中的第一步跃迁任务。我们采用饱和吸收谱(SAS)技术将种子激光锁定在原子超精细跃迁线上,并通过声光调制器(AOM)网络实现不同光路所需的快速频率切换。

% [PLACEHOLDER: 科学内容填充]
% 1. [图表] 780 nm 激光系统光路图(ECDL -> Isolator -> SAS Lock -> TA -> Fiber Distribution)。
% 2. 频率锁定方案细节:PDH 稳频或 MTS 稳频的选择。
% 3. 锥形放大器(TA)的输出功率与耦合效率。

\subsection{D1 线成像与冷却光源 (795 nm)}
\label{subsec:d1_laser}

传统的 780 nm 偏振梯度冷却(PGC)成像往往伴随着较大的光子散射加热,限制了成像的保真度和原子留存率。为此,我们构建了基于 795 nm D1 线 ($5S_{1/2} \to 5P_{1/2}$) 的光源系统。利用电光调制器(EOM)产生边带,我们能够实施 $\Lambda$-增强型灰莫拉斯($\Lambda$-enhanced gray molasses)冷却。该机制利用相干暗态(Dark State)在冷却的同时显著抑制了加热效应,是实现高信噪比单原子成像的关键。

% [PLACEHOLDER: 科学内容填充]
% 1. [图表] 795 nm 激光光路及 EOM 边带产生示意图。
% 2. 描述 EOM 的调制频率(对应基态超精细分裂 6.8 GHz)及调制深度优化。
% 3. 描述成像光与冷却光在时序上的复用策略。

\subsection{偶极阱光源}
\label{subsec:trap_laser}

为了在真空中捕获原子,我们需要极高亮度的远红外激光来构建深光镊势阱。实验中我们选用了高功率的单频激光器(如 1064 nm 光纤激光器或 Ti:Sa 激光器)。该光源经过精密的光束整形和功率稳定控制后,被送入后续的 SLM 和 AOD 系统中。

% [PLACEHOLDER: 科学内容填充]
% 1. 陷阱激光器的选型(波长、功率、线宽)。
% 2. 强度噪声控制:AOM 噪声吞噬(Noise Eater)电路的设计与性能。
% 3. 光束质量($M^2$ 因子)的测量与优化。

\section{光镊阵列生成与控制}
\label{sec:tweezer_generation}

本实验的一大特色是结合了静态全息光镊与动态扫描光镊的优势。我们利用空间光调制器(SLM)生成大规模、无缺陷的静态背景晶格,同时利用声光偏转器(AOD)生成可快速移动的“光镊手”,用于对原子进行逐个分拣和重排。

\subsection{静态阵列生成 (SLM)}
\label{subsec:slm_generation}

空间光调制器通过对入射激光波前施加可编程的相位调制,在物镜的焦平面上重构出任意几何分布的光强图案。为了获得高均匀性的阵列,我们采用了加权 Gerchberg-Saxton (GS) 迭代算法。该系统不仅能生成成百上千个光镊,还能通过相位图校正光学系统中的像差,从而保证每个格点的俘获性能一致。

% [PLACEHOLDER: 科学内容填充]
% 1. [图表] SLM 光路示意图(4f 系统映射到物镜入瞳)。
% 2. 描述 GS 算法的工程实现:如何处理阵列均匀性反馈(Feedback Loop)。
% 3. 描述高 NA 物镜的选择与像差校正效果。

\subsection{动态阵列生成 (AOD)}
\label{subsec:aod_generation}

为了实现原子的动态输运,我们搭建了基于双轴声光偏转器(2D AOD)的扫描系统。通过精密控制射频驱动信号的频率,我们可以精确控制光镊在焦平面上的位置。与 SLM 不同,AOD 产生的光镊具有微秒级的响应速度,这使得我们能够在原子相干时间内完成复杂的重排操作。

% [PLACEHOLDER: 科学内容填充]
% 1. [图表] 交叉 AOD (Crossed AOD) 光路构型。
% 2. 射频驱动系统(RF Drivers):频率合成器的选择与带宽控制。
% 3. 频率-位置映射校准(Calibration):如何建立查找表(LUT)。

\section{实验时序与控制}
\label{sec:control_system}

冷原子实验是一个对时序要求极高的过程,涉及数十路数字/模拟信号的同步控制,时间跨度从微秒级的激光脉冲到秒级的 MOT 加载。本节介绍基于 FPGA 和 DDS(直接数字合成)技术的中央控制系统,以及一个典型的单原子实验周期。

% [PLACEHOLDER: 科学内容填充]
% 1. 硬件架构:FPGA 主控板与各子模块(DDS, DIO, AO)的连接。
% 2. 软件界面:时序编写与编译流程。
% 3. [图表] 典型实验时序图:MOT Load -> PGC -> Trap Load -> Imaging 1 -> Rearrangement -> Experiment -> Imaging 2。

\section{单原子成像与探测}
\label{sec:imaging_detection}

在量子模拟实验中,判定每个格点中原子的有无(0 或 1)是信息读出的基础。我们利用高灵敏度的 EMCCD 相机收集原子荧光。为了在曝光期间防止原子丢失,我们实施了基于 795 nm 的原位冷却成像技术。

\subsection{成像光路设计}
\label{subsec:imaging_path}

成像系统需要收集尽可能多的荧光光子,同时滤除背景杂散光。我们设计了一套与光镊生成光路共用高 NA 物镜的成像光路,利用二色镜将 795 nm 的原子荧光从 1064 nm 的陷阱光中分离出来,并成像在 EMCCD 的靶面上。

% [PLACEHOLDER: 科学内容填充]
% 1. [图表] 荧光收集光路图:二色镜、滤光片、管镜(Tube lens)及相机位置。
% 2. 相机配置:EM 增益设置、ROI(感兴趣区域)划分。
% 3. 背景噪声抑制策略。

\subsection{D1 线成像机制}
\label{subsec:d1_imaging_mechanism}

相比于传统的 D2 线成像,D1 线灰莫拉斯成像技术在提供荧光信号的同时,能够产生更强的冷却力,从而抵消光散射引起的反冲加热。这种机制使得我们能够在更浅的势阱中,或者更长的曝光时间下,依然保持极低的原子损失率,从而显著提升成像的保真度。

% [PLACEHOLDER: 科学内容填充]
% 1. 实验参数:795 nm 激光的失谐量、光强、曝光时间。
% 2. [数据图] 单原子成像直方图(Histogram):双峰结构(0原子/1原子)。
% 3. 保真度计算:利用直方图重叠度估算探测误差。

\section{原子重排与移动系统}
\label{sec:atom_assembly}

本章的最后一部分介绍如何利用 AOD 移动光镊将随机装载的原子阵列重排为确定性的有序阵列。这一过程被称为“原子组装”(Atom Assembly),是实现大规模量子寄存器的前提。

\subsection{移动光镊算法}
\label{subsec:assembly_algorithm}

原子重排的核心是路径规划算法。系统首先根据第一张原子图像计算出当前的原子分布,然后计算出将原子从初始位置移动到目标位置所需的移动步骤。我们需要设计防碰撞算法(Collision-free algorithms)以确保在移动过程中原子不会丢失,并且尽量减少总的移动时间。

% [PLACEHOLDER: 科学内容填充]
% 1. 算法逻辑描述:如何分配源原子到目标空位(如匈牙利算法或贪心算法)。
% 2. 移动波形设计:平滑加减速曲线(S-curve 或 Minimum-jerk)以减少加热。

\subsection{移动与静态光镊的切换与对齐}
\label{subsec:handover}

为了实现高效的原子抓取和释放,AOD 生成的移动光镊必须与 SLM 生成的静态光镊在空间上完美重合。我们开发了一套自动化的配准(Registration)程序,通过扫描 AOD 频率并监测原子荧光或光镊光强,建立两个坐标系之间的精确映射关系,从而优化原子在“静态阱 $\leftrightarrow$ 移动阱”之间的交接(Handover)效率。

% [PLACEHOLDER: 科学内容填充]
% 1. 坐标系配准流程。
% 2. 交接过程中的光强时序控制(Ramp down/up)。
% 3. [数据图] 重排前后的阵列荧光图对比,展示完美阵列的生成。


\end{document}
