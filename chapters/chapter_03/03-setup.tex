% chapters/01-introduction.tex
\documentclass[../main.tex]{subfiles}
\begin{document}
% 章节内容...

\chapter{实验装置与控制系统}
\label{chap:experimental_setup}

在前一章中,我们讨论了中性原子俘获与操控的物理原理。本章将详细阐述为了实现这些物理目标而搭建的实验硬件平台。我们的实验系统采用紧凑的双真空腔体设计,结合了用于产生大规模静态阵列的空间光调制器(SLM)技术和用于实现单原子动态重排的声光偏转(AOD)技术。此外,为了克服传统成像技术中的加热问题,我们开发了一套基于 795 nm D1 线 $\Lambda$-增强型灰光学黏团(Gray Molasses)机制的成像系统,实现了高保真度的原位探测。本章将按照“真空系统 $\to$ 光学系统 $\to$ 阵列生成 $\to$ 探测与控制”的逻辑顺序,对各个子系统进行逐一介绍。

\section{真空系统设计}
\label{ch3:sec:vacuum_system}

为了在保持较长原子寿命的同时实现快速的实验重复频率,真空系统必须满足两个看似矛盾的要求:科学腔(Science Chamber)需要维持极低的背景气压($< 10^{-10}$ Torr)以抑制背景碰撞损失,而原子源区域则需要维持较高的铷原子蒸气压以保证足够的装载通量。为此,我们设计了一套双腔差分泵浦系统,通过流阻极大的细长管道将两个压力区隔离开来。

\subsection{双腔差分系统架构}
\label{ch3:sec:vacuum_architecture}

我们的真空系统采用垂直堆叠结构,如图\ref{ch3:fig:vacuum_system}所示。上层为 二维磁光阱(2D MOT)腔,用于从铷源中产生初级冷原子束;下层为三维磁光阱(3D MOT)科学腔,是进行单原子俘获与量子模拟实验的核心区域。两层腔体通过一个精密的差分管(Differential Pumping Tube)连接。该设计不仅有效地隔离了两个区域的真空环境,还利用重力辅助了原子的输运过程。

上层的2D MOT腔主要包含以下几个部分:铷源、1.33英寸角阀(小阀门)、异形四通、六面体腔、离子泵、2.75英寸真空视窗等构成,所有的真空部件通过标准的2.75英寸(CF35)或者1.33英寸(CF16)法兰连接在一起。铷源是一个有5g高纯铷单质的玻璃安瓿瓶,安瓿瓶中充有惰性保护气体,中间细两头粗,放到一个长条形的波纹管内,折断中间的部分即可让铷原子蒸气释放出来。长条形的波纹管连接在一个1.33英寸角阀上,可以通过角阀来控制释放到2D MOT腔中的铷蒸汽的速率。铷单质中${}^{87}\mathrm{Rb}$的自然丰度只有大约$27.8\%$,其余都是${}^{85}\mathrm{Rb}$,我们选择${}^{87}\mathrm{Rb}$作为实验的原子种类。小阀门连接的是一个有两个1.33英寸法兰和两个2.75英寸法兰的异形四通。异形四通的另一个1.33法兰连接的是一个三井的小型离子泵,上面的2.75英寸法兰连接一个推送光的视窗,下方连接2D MOT腔。离子泵用来将2D MOT腔的真空度维持在实验所需的量级,一般是$10^{-9}\sim10^{-8}$Torr之间。

2D MOT腔的主体是一个立方体的六通,六个面都有2.75英寸的法兰,周围四个面连接四个视窗,用来入射冷却原子的激光。在2D MOT腔的外面放有永磁体,安装在永磁体座中,永磁体制备的梯度磁场(磁场梯度大致48G/cm)可以用来构建2D MOT。原子在激光的作用下将会在水平的两个方向上减速,而原子又因为互相碰撞,可以实现对原子的整体减速,最终可以制备一个沿着竖直方向分布的细长冷原子团。在2D MOT腔上方的真空视窗处,可以入射一束与原子近共振的推送光,用于将冷却后的原子从2D MOT腔中垂直向下穿过中间的差分管小孔推送到下方的3D MOT腔中。2D MOT腔下方连接一个2.75英寸的四通,在四通的上面的法兰上安装有一个差分管组件。差分管组件是一个中间开了一个长24mm直径1mm的小孔的真空部件,主要用来隔绝2D MOT腔和3D MOT腔的真空。一般3D MOT腔的真空度则需要维持在$ 10^{-11}$ Torr量级,因此需要采用差分管结构来产生两三个数量级差值的真空梯度。

\begin{figure}[ht]
  \centering
  \includegraphics[width=0.8\textwidth]{chapters/chapter_03/figure/vacuum_system.png}
  \caption{\textbf{双腔结构的真空系统} (a)设计图,(b)实物图}
  \label{ch3:fig:vacuum_system}
\end{figure}

下层的3D MOT腔部分主要由一个石英玻璃腔、2.75英寸角阀(大阀门)、SAES化学吸气剂泵以及四通组成。石英玻璃腔是主要的科学腔,最下方是方形的,通过中间的热胀冷缩系数渐变的玻璃材料连接到上方的金属法兰上,金属法兰则与四通下方连接在一起,中间用热胀冷缩系数过渡的材料主要是考虑到加热情况下玻璃腔不会因为玻璃与金属的热胀冷缩系数不一样导致出现很大的内部应力而碎掉。四通的左边是大阀门,主要是用来作为分子泵预抽口的,用来将3D MOT腔的真空从大气压抽到$10^{-8}$ Torr量级,从而可以开启SAES化学吸气剂泵。SAES化学吸气剂泵是用来维持3D MOT腔的超高真空环境,它的主体结构是一个表面可以很高效地吸附各种气体分子的化学材料,拧在一个真空管道上。SAES泵后面还搭配了一个小型的离子泵电离掉不能被吸附的一些气体,比如惰性气体。选择抽速200L/s的吸气剂泵,可以实现$10^{-11}$ Torr量级的超高真空环境,这对于进行原子阵列实验是必不可少的条件。

为了制备超高真空的实验环境,我们采用如下的操作。首先需要对所有的真空部件进行清洗,用以清除掉表面所有的残留的油污。之后对一些金属的部件进行空气烤,逐渐将真空部件加热到300-400℃,一方面进一步除掉金属部件表面吸附的气体,二是可以在表面形成氧化膜,从而减小放气量。需要注意的是,升温和降温过程中需要控制到温度的变化速率,不能太快,否则容易导致真空部件因为热胀冷缩而变形,从而影响真空密封效果。然后是需要进行真空组装,在拧紧法兰的过程中,需要用质地较软的退火铜垫圈或者银垫圈来连接,还需要控制好所加的力矩大小,避免出现一边紧一边松的情况,否则很容易漏。在全部组装好后,就需要进行真空烤,也就是一边在抽真空的情况下,一边进行烘烤除气。首先需要用一些加热带和隔热棉搭建一个烤炉来包裹住真空系统,给加热带通电来加热真空系统,需要注意用热电偶温度计监控各处的温度均匀性,真空烤加热到150-200℃,之后用分子泵机组分别将2D MOT和3D MOT腔从大气压抽到高真空的状态。在此过程中,腔体内部吸附的气体会跑出来被分子泵抽走。与此同时,离子泵和SAES化学吸气剂泵在刚开始启动的过程中也会有一些气体喷射出来,也可以借此通过分子泵机组抽走。在进行真空烤足够长的时间后,一般来说,2D MOT和3D MOT腔内的真空度就会逐渐稳定到一个比较高的水平,之后就可以逐步将温度降低到室温并撤去加热带了。经过以上的一系列步骤,可以在3D MOT腔中很好地制备$10^{-11}$ Torr量级的超高真空环境。

% [PLACEHOLDER: 科学内容填充]
% 1. [图表] 真空系统整体 CAD 渲染图(侧视),标注上层源腔、中间差分管、下层科学腔及离子泵位置。
% 2. 描述上层 2D MOT 腔体的尺寸与材料(玻璃/不锈钢)。
% 3. 描述下层科学腔的设计:光学窗口的镀膜参数(针对 780nm, 795nm, 1064nm, 480nm 等波长优化)及数值孔径(NA)限制。
% 4. 差分管的流阻计算:根据管道长度和直径估算压差比(Pressure ratio)。

% \subsection{原子源与输运}
% \label{ch3:sec:atom_source}

% 原子源的稳定性直接决定了实验的占空比。我们在上层腔体中配置了可独立控制电流的铷金属分配器(Dispenser)。通过 2D MOT 的冷却,热原子被横向压缩并沿差分管轴向推出,形成高通量的冷原子束流。为了进一步提高传输效率,我们引入了一束沿轴向传播的推原子光(Push Beam),将原子从 2D MOT 高效地“抛射”至下方的 3D MOT 俘获区。



% [PLACEHOLDER: 科学内容填充]
% 1. Rb 分配器的型号与电流工作点。
% 2. 2D MOT 的磁场构型(永磁体或线圈)及冷却光路配置。
% 3. Push 光的参数:失谐量、功率、光斑大小及对原子束通量的影响数据。

\section{激光与光学系统}
\label{ch3:sec:optical_systems}

本实验涉及多个波段的激光系统,包括用于冷却和态制备的 780 nm 光源、用于 D1 线成像的 795 nm 光源,以及用于产生光镊势阱的远红外光源。为了保证激光频率和功率的长期稳定性,我们搭建了基于外腔半导体激光器(ECDL)和锥形放大器(TA)的稳频与放大链路。

\subsection{冷却与俘获光源 (780 nm)}
\label{ch3:sec:cooling_laser}

780 nm 激光对应于 $^{87}\text{Rb}$ 的 D2 跃迁线 ($5S_{1/2} \to 5P_{3/2}$),承载了 MOT 冷却、光抽运(Optical Pumping)以及里德堡激发中的第一步跃迁任务。我们采用饱和吸收谱(SAS)或者PDH稳频技术将种子激光锁定在原子超精细跃迁线或者超稳腔的腔模上,并通过声光调制器(AOM)网络实现不同光路所需的快速频率切换。

\begin{figure}[ht]
  \centering
  \includegraphics[width=0.8\textwidth]{chapters/chapter_03/figure/MOT_freq_setup.png}
  \caption{\textbf{MOT光频率设置} }
  \label{ch3:fig:MOT_freq_setup}
\end{figure}

首先需要考虑到2D MOT和3D MOT所需的冷却光和重泵浦光。如图\ref{ch3:fig:MOT_freq_setup}所示,冷却光(cooling laser)对应的频率是$|5S_{1/2},F=2\rangle\to |5P_{3/2},F=3\rangle$,红失谐$\delta\nu=-6\Gamma$,这里$\Gamma=6\mathrm{MHz}$是$|5P_{3/2},F=3\rangle$的自然线宽,冷却光主要用来实现对原子进行冷却;重泵浦光(repumping laser)对应的频率是$|5S_{1/2},F=1\rangle\to |5P_{3/2},F=2\rangle$,一般是近共振的,主要用来将原子从$|5S_{1/2},F=1\rangle$重新泵浦到$|5S_{1/2},F=2\rangle$。冷却光和重泵浦光的协同作用,即可实现对原子的持续冷却。要实现有效的冷却效果,还需要考虑到激光的功率,尤其是2D MOT和3D MOT所需要的冷却光的功率比较大。如图\ref{ch3:fig:2DMOT_3DMOT}所示,2D MOT采用反射式MOT结构,冷却光只需要两束光,每路大约45mW,重泵浦光需要一束,通常几毫瓦就足够了,其中重泵浦光可以与冷却光通过光纤合束从同一根光纤出射打到原子上。3D MOT光在上下方向上有一对斜MOT光入射到3D MOT腔中,另外还有垂直的一对正MOT光,每路大约25mW。斜MOT光容易通过玻璃腔的腔壁产生干涉条纹,因此需要精细调节3D MOT光的角度以消除干涉条纹的影响。

\begin{figure}[ht]
  \centering
  \includegraphics[width=0.8\textwidth]{chapters/chapter_03/figure/2DMOT_3DMOT.png}
  \caption{(a) 2D MOT光路,(b)3D MOT光路}
  \label{ch3:fig:2DMOT_3DMOT}
\end{figure}

为了将基态原子全部制备到$|5S_{1/2},F=2, m_F=2\rangle$这个磁子能级,还需要入射一个指定量子化轴方向下$\sigma^+$偏振的光泵浦光。为了通过双光子激发的方式将原子从基态激发到里德堡态上,还需要考虑一个780nm里德堡激发光。此外,还需要考虑到将原子从2D MOT推送到3D MOT的推送光,以及对原子团进行成像的探测光(图\ref{ch3:fig:MOT_freq_setup}中的detection laser)。

\begin{figure}[ht]
  \centering
  \includegraphics[width=\textwidth]{chapters/chapter_03/figure/780_optical_setup.png}
  \caption{(a) 780 TA激光器,(b)double-pass移频光路}
  \label{ch3:fig:780_optical_setup}
\end{figure}

在频率上,由于冷却光、光泵浦光、780nm里德堡激发光、推送光以及探测光之间的频率相差比较小,因此可以共用一台激光器,激光器(图\ref{ch3:fig:MOT_freq_setup}中的TA laser)的频率锁定在$|5S_{1/2},F=2\rangle\to |5P_{3/2},F=3\rangle$红失谐460MHz的位置,之后通过200MHz的声光调制器分别进行double-pass移频即可得到上述所有的激光。如图\ref{ch3:fig:780_optical_setup}(a)所示,我们采用Toptica的带锥形半导体激光放大器(tapered amplifier,TA)的外腔半导体激光器,其出光功率可以达到2W左右,由于锥形激光放大器的光斑模式在一个方向上是拉长的,一般还需要调节它的光斑形状,使其尽可能接近高斯模式,之后通过损伤阈值较高的玻片和偏振分束镜(PBS)进行分光,并耦合到高功率光纤中。之后可以对输出的激光通过声光调制器进行double-pass移频(图\ref{ch3:fig:780_optical_setup}(b)),即可得到实验所需的各种激光,这些激光可以通过光纤拉到实验平台入射到原子上。而重泵浦光的频率与冷却光频率相差了大约6.8GHz,很难通过AOM移频的方式实现,因此,需要使用另外一台激光器,然后用类似的方式将其入射到2D MOT和3D MOT腔中。



% [PLACEHOLDER: 科学内容填充]
% 1. [图表] 780 nm 激光系统光路图(ECDL -> Isolator -> SAS Lock -> TA -> Fiber Distribution)。
% 2. 频率锁定方案细节:PDH 稳频或 MTS 稳频的选择。
% 3. 锥形放大器(TA)的输出功率与耦合效率。


\subsection{D1 线成像与冷却光源 (795 nm)}
\label{ch3:sec:d1_laser}

铷原子的$D_1$ 线也是非常重要的能级,对应激光波长为795nm,一方面也可以用作重泵浦光,另外也可以用作用到单原子成像中。因此我们构建了基于 795 nm $D_1$ 线 ($5S_{1/2} \to 5P_{1/2}$) 的光源系统,如图\ref{ch3:fig:795_laser}所示。795nm的激光器也是Toptica的外腔半导体激光器,能输出几十毫瓦的激光,通过PDH稳频的方法锁定到实验所需的频率上。输出的激光通过一些1/2玻片和PBS分成很多路,之后再分别耦合到单模保偏光纤中,之后将光纤拉到主光学平台上即可入射到原子上。

% [PLACEHOLDER: 科学内容填充]
% 1. [图表] 795 nm 激光光路及 EOM 边带产生示意图。
% 2. 描述 EOM 的调制频率(对应基态超精细分裂 6.8 GHz)及调制深度优化。
% 3. 描述成像光与冷却光在时序上的复用策略。

\begin{figure}[ht]
  \centering
  \includegraphics[width=0.6\textwidth]{chapters/chapter_03/figure/795_laser.png}
  \caption{795nm激光器}
  \label{ch3:fig:795_laser}
\end{figure}

\subsection{激光稳频系统}

\begin{figure}[ht]
  \centering
  \includegraphics[width=\textwidth]{chapters/chapter_03/figure/780_SAS.png}
  \caption{(a) 饱和吸收谱光路,(b)铷原子饱和吸收谱}
  \label{ch3:fig:780_SAS}
\end{figure}

在我们的实验中,我们将780重泵浦光通过饱和吸收谱稳频的方式将其锁定在${}^{87}\mathrm{Rb}$的F=1线的饱和吸收谱的交叉峰上。其具体方法是,由于从780重泵浦光激光器分出100uW左右的光,分成两部分将其对射到一个装有饱和铷蒸气的圆柱形原子气室的端面,其中一路光穿过原子气室后打到一个光电探测器上进行探测。在通过压电陶瓷(PZT)对外腔半导体激光器进行扫频时,如果激光的频率刚好与原子的共振吸收峰对应,那么将会贡献一个饱和吸收峰,从而得到原子的整个饱和吸收谱。其中最高的一个峰是$|5S_{1/2},F=1\rangle\to |5P_{3/2},F=1\rangle$和$|5S_{1/2},F=1\rangle\to |5P_{3/2},F=2\rangle$的交叉峰(如图\ref{ch3:fig:780_SAS}所示),之后通过一个+78.5MHz左右的声光调制器单通移频即可将重泵浦光频率锁定到与$|5S_{1/2},F=1\rangle\to |5P_{3/2},F=2\rangle$共振的激光频率上。

\begin{figure}[ht]
  \centering
  \includegraphics[width=\textwidth]{chapters/chapter_03/figure/PDH.png}
  \caption{(a) 超稳腔,(b)PDH稳频原理路}
  \label{ch3:fig:PDH}
\end{figure}

对于其他的780 cooling激光、795光以及偶极阱光则都是通过标准的PDH稳频方案将其频率锁定在超稳腔的腔模上。超稳腔的腔模频率主要取决于腔长,要实现足够稳定的腔模频率,就需要非常稳定的腔长。超稳腔一般由ULE玻璃制成,其热胀冷缩系数非常低,同时对常见的近红外波段透光性比较好。将超稳腔放到一个密闭的带有控温装置的真空环境中,同时将温度设置到超稳腔的零膨胀点,超稳腔的热胀冷缩可以进一步被抑制,从而可以实现更高的频率稳定性。超稳腔端面镀有两个高反镜,可以实现精细度达到30000以上的腔,从而实现更加精细的频率分辨。在PDH稳频方案中,我们首先将一束10uW量级的光经过一个相位型电光调制器后入射到超稳腔并得到它的反射谱,反射谱通过一个雪崩光电探测器探测得到。当待锁定激光(频率$\nu$)被电光调制器调制以$f$的频率进行相位调制后,激光会产生$\nu\pm f$的两个边带,如果$\nu\pm f$刚好与超稳腔的腔模$\nu_{cavity}$共振,就可以利用腔模的反射谱对激光频率进行负反馈锁定,即可将待锁定激光锁定到$\nu=\nu_{cavity}\pm f$的频率上,通过改变调制频率$f$,即可将待锁定激光的频率锁定到我们想要的频率上。超稳腔的一系列腔模$\nu_{cavity}$的频率是等间距排列的一系列纵模,频率分布范围广,因此可以实现很大范围内的激光频率稳定。在我们的实验中,可以实现覆盖实验所需的近红外波段的精度达到0.1MHz的频率锁定。



\subsection{偶极阱光源}
\label{ch3:sec:trap_laser}
% [PLACEHOLDER: 科学内容填充]
% 1. 陷阱激光器的选型(波长、功率、线宽)。
% 2. 强度噪声控制:AOM 噪声吞噬(Noise Eater)电路的设计与性能。
% 3. 光束质量($M^2$ 因子)的测量与优化。

为了在冷却后的原子团中捕获原子,我们需要使用光强很集中的远红外激光来构建深光镊势阱。光镊通过一个高数值孔径的物镜聚焦到接近衍射极限来实现,在光镊数量较多时对激光功率的要求比较高。实验中我们选用808nm的高功率Toptica窄线宽外腔半导体激光器作为光镊光源,该光源一个锥形激光放大器(TA)后可以输出1W左右的激光。后续经过精细的光束整形使得经过TA后的光斑形状足够接近高斯,从而尽可能提高光纤耦合效率。输出的激光分成两束,分别经过两个单通的声光调制器后耦合进两根光纤,分别入射到空间光调制器上以及声光偏转器上,以生成静态或者动态的光镊阵列。

\begin{figure}[ht]
  \centering
  \includegraphics[width=0.6\textwidth]{chapters/chapter_03/figure/808_laser.png}
  \caption{808nm光镊激光器}
  \label{ch3:fig:808_laser}
\end{figure}

光镊势阱的功率噪声会对被囚禁的原子有加热效果,尤其是与势阱中声子频率共振的功率噪声分量。噪声的来源有光纤耦合不稳定、空间光调制器带来的功率噪声、激光器本身的功率噪声、激光偏振抖动等多种因素,因此需要对光镊光进行功率噪声的抑制。抑制光功率噪声主要通过光功率反馈来实现。首先将入射到原子上光分出一小部分到一个光电探测器,将光功率转化为电压信号,之后将该电压信号与设置的直流电压做差分得到误差信号,之后将误差信号通过PID电路输出一个电压控制信号,该电压控制信号输出到一个压控射频衰减器上,进而通过控制输入到声光调制器的射频功率来控制衍射的激光的功率,这样就实现了测量到控制的闭环控制。由于负反馈的环路有延迟,因此该方法只能有效抑制反馈带宽范围内的功率噪声,反馈带宽外的高频噪声则无法被抑制。特别地,在接近反馈带宽的频率处,环路将由负反馈转变为正反馈,反而会让该频段的功率噪声增大,因此需要设计反馈电路让该频段噪声尽可能远离光镊中的声子频率。

\section{光镊阵列生成与控制}
\label{ch3:sec:tweezer_generation}

本实验的一大特色是结合了静态全息光镊与动态扫描光镊的优势。我们利用空间光调制器(SLM)生成大规模、无缺陷的静态背景晶格,同时利用声光偏转器(AOD)生成可快速移动的“光镊手”,用于对原子进行逐个分拣和重排。

\subsection{静态阵列生成 (SLM)}
\label{ch3:sec:slm_generation}
% [PLACEHOLDER: 科学内容填充]
% 1. [图表] SLM 光路示意图(4f 系统映射到物镜入瞳)。
% 2. 描述 GS 算法的工程实现:如何处理阵列均匀性反馈(Feedback Loop)。
% 3. 描述高 NA 物镜的选择与像差校正效果。

空间光调制器通过对入射激光波前施加可编程的相位调制,在物镜的焦平面上重构出任意几何分布的光强图案。为了获得高均匀性的阵列,我们采用了加权 Gerchberg-Saxton (GS) 迭代算法。该系统不仅能生成成百上千个光镊,还能通过相位图校正光学系统中的像差,从而保证每个格点的俘获性能一致。

我们搭建了如下的光路,首先用一束808nm的平行光入射到空间光调制器(SLM)上,然后在空间光调制上加载相图,不同衍射角度的平行光经过4f系统映射到一个高数值孔径的物镜镜头的后焦面上,之后经过物镜聚焦即可在前焦面上形成光镊阵列。物镜的入瞳决定了能够进入的光的空间频率的大小,从而间接限制了光镊阵列的分布范围。光镊阵列需要对准聚焦到3D MOT玻璃腔中的冷原子团上,当光镊势阱深度很深时(比如达到1mK以上),即可在冷却到几十uK的冷原子团中高效地捕获原子。由于SLM的刷新率很慢,通常只有60Hz左右,因此光镊阵列一旦生成,就很难快速的动态调整,因此由SLM生成的光镊阵列一般是作为静态光镊存在的。与此同时,SLM的刷新还会导致光镊阵列的光功率会发生周期性的抖动,因此需要对SLM衍射后的光斑加上光电探测器进行功率监控并通过负反馈的方法将功率抖动降低到最低。

为了将光镊聚焦得足够小同时又能保证光镊阵列的范围足够大,选择合适参数的物镜非常重要。首先需要较长的工作距离,因为光镊阵列平面在石英玻璃气室内部,与真空气室外部有一定距离,同时也需要考虑对玻璃气室外壁的成像补偿。另外,也需要高的数值孔径,才能将光镊聚焦得更小。为了容纳多个波长,需要对实验所需的激光波长同时具有高透射率以及比较小的像差。较大的入瞳(即较大的后孔径)能够捕获由SLM产生的大角度衍射光,从而在焦面形成更大范围或更多数量的光镊。为了确保边缘光镊的质量,物镜在所需视场内还应具有良好的像差校正(如使用平场复消色差物镜)。同时,视场范围还要足够大,一方面可以减小物镜安装对准的难度,另一方面,一般视场范围越大,成像质量良好的区域也会越大。

为了实现对光镊阵列的成像,还需要在聚焦物镜的对面再放一个物镜镜头,用于将光镊阵列通过4f成像放大几十倍后成像到一个sCMOS相机上,这样就可以判断光镊阵列的形貌、光斑均匀性、聚焦程度等等。理想的情况是光镊阵列能够以接近衍射极限的程度聚焦,而光镊阵列也能以接近衍射极限的程度成像。但是在实际光学系统中,由于经过的光学元件数量众多,每个光学表面的波前误差都会叠加到光镊阵列的波前上,因此,通常在一开始的时候很难一下子就得到理想的结果。通常表现为光镊形态畸变、光斑均匀度不够好等等问题,因此需要进行反馈迭代得到高均匀性的光镊阵列。

\begin{figure}[ht]
  \centering
  \includegraphics[width=0.5\textwidth]{chapters/chapter_03/figure/array.png}
  \caption{通过加权Gerchberg-Saxton (GS) 算法通过SLM生成的31*31的光镊阵列}
  \label{ch3:fig:array}
\end{figure}

这里我们采用加权Gerchberg-Saxton (WGS) 迭代算法来实现高均匀性的光镊阵列,它的算法逻辑如下。第一步,首先在SLM上加载一个随机的初始相图$\phi_{in}(u,v)$。第二步,根据入射光强分布$A_{in}(u,v)$与SLM上的相位分布进行傅里叶变换得到目标光镊处的振幅$A_t'(x,y)$和相位$\phi_t'(x,y)$。第三步,用一个加权振幅分布$\bar{A}_t(x,y)=w\sqrt{I_{target}(x,y)}+(1-w)A_t(x,y)$替代掉计算的$A_t'(x,y)$,得到新的复振幅$\bar{A}_t(x,y)\exp(i\phi_t'(x,y))$,这里$I_{target}(x,y)$是目标光镊强度分布,$A_t(x,y)$是相机拍到的光镊阵列强度分布,$w$是权重。第四步,将$\bar{A}_t(x,y)\exp(i\phi_t'(x,y))$进行逆傅里叶变换,得到新的入射平面相图$\phi_{in}'(u,v)$并将其加载到SLM上。重复以上第二步到第四步的流程,直到收敛到最够小的误差。这里计算加权振幅分布$\bar{A}_t(x,y)$时也可以采用一些其他的加权方法,比如根据光镊强度来设置权重等等,以尽可能得到强度均匀的光镊阵列。如图\ref{ch3:fig:array},我们通过该方法成功了生成了31*31的大光镊阵列,光镊均匀度达到0.98。


\subsection{动态阵列生成 (AOD)}
\label{ch3:sec:aod_generation}

为了实现原子的动态输运,我们搭建了基于双轴声光偏转器(2D AOD)的扫描系统。通过精密控制射频驱动信号的频率,我们可以精确控制光镊在焦平面上的位置。与 SLM 不同,AOD 产生的光镊具有微秒级的响应速度,这使得我们能够在原子相干时间内完成复杂的重排操作。

% [PLACEHOLDER: 科学内容填充]
% 1. [图表] 交叉 AOD (Crossed AOD) 光路构型。
% 2. 射频驱动系统(RF Drivers):频率合成器的选择与带宽控制。
% 3. 频率-位置映射校准(Calibration):如何建立查找表(LUT)。

首先是二维AOD产生动态光镊阵列的光路构型。二维AOD由轴垂直的两个一维声光偏转器结合在一起构成,采购自AA-optics公司,具有衍射角大的特点。两个一维AOD可以分别添加独立的射频信号,从而可以实现两个方向上的相互独立的偏转。将二维AOD通过4f系统映射到物镜的后焦面上,之后经过物镜聚焦即可生成动态的光镊。光镊位置由衍射角度决定,而衍射角的大小又与所添加的射频信号的频率有关,因此只需要改变所加的射频信号的频率即可实现对光镊位置的控制。如果要生成等间距的光镊阵列,就需要生成频率等间隔的众多射频信号加载到AOD上。但是在阵列规模较大时,射频信号太多也会导致其他问题,比如不同频率信号之间的串扰问题,一些非线性现象导致的混频问题,多个射频信号导致的器件发热问题,等等。因此,一般动态光镊的数量不会太大。

%射频信号的产生,可以通过高性能的任意波形发生器(AWG)来实现。AWG主要优势是可以生成频率、幅度和初始相位都可控的多路射频信号,同时可以实现射频信号频率的快速切换。高性能AWG卡可以通过PCIe直接与计算机主板相连,从而显著减小信号延迟。之后AWG信号经过射频功率放大器后再接到AOD上。

在本系统中,驱动二维声光偏转器(AOD)生成动态光镊阵列所需的复杂射频信号,是通过高性能任意波形发生器(AWG)来实现的。AWG是该系统的信号源头,其核心优势在于其无与伦比的波形合成灵活性:它能够根据预设的算法,精确生成频率、幅度和初始相位均可独立编程控制的多路射频信号。这种精细化的调控能力是抑制多频同时输入时产生的互调效应、保证光镊阵列均匀性的物理基础。此外,AWG还具备极快的序列跳变能力,能够在微秒甚至纳秒量级内完成射频信号参数的快速切换,这对于实现时分复用模式下的高速动态光镊操控至关重要。为了满足实时动态光镊系统对极低延迟的苛刻要求,系统架构中采用了基于PCIe总线的高性能AWG模块。该模块通过PCIe接口直接与上位计算机的主板相连,建立了一条高速、低延迟的数据通道。相比于传统的通过USB或以太网进行通信的台式仪器,这种板卡集成方式可以显著减少波形数据传输的命令排队与协议开销,从而将系统指令到信号输出的总延迟降至最低,确保了对光镊阵列进行实时、闭环控制的响应速度。

AWG直接输出的信号电压幅值通常较低(一般在毫伏至伏特量级),远不足以驱动AOD的压电换能器产生足够强度的超声波。因此,AWG生成的微弱射频信号必须经过射频功率放大器进行功率放大。在选用功率放大器时,需要特别关注其线性度和增益平坦度指标。尤其是在多频同时输入的模式下,功率放大器的非线性失真会产生新的互调分量,污染原有的纯净频谱,直接导致焦平面上出现杂散光斑。因此,通常需要选用线性度良好的射频放大器。经过放大后的射频信号具备了足够的驱动能力,最终被加载到AOD的换能器上,在晶体内部激发超声光栅,完成对入射激光束的偏转与整形。

在基于声光偏转器(AOD)的光镊系统中,光镊在焦平面上的位置本质上是由加载到AOD上的射频信号频率决定的。然而,这种对应关系并非一个简单的线性比例,而是受到诸多光学和电子学因素的综合影响。因此,若要将光镊精确地对准某一指定位置(例如特定的微粒或坐标点),必须事先进行系统且精细的频率-位置映射校准。首先建立初始映射关系:根据系统的基本物理参数(如光波长、声速、物镜焦距)计算出一个粗略的理论映射关系。基于此理论值,生成一组覆盖整个预期工作区域的稀疏频率组合,并在焦平面上产生相应的静态光斑阵列。然后利用相机(如CMOS或CCD)采集该光斑阵列的图像,并通过图像处理算法(通常是质心定位算法)精确计算出每一个光斑在相机像素坐标系下的实际坐标。之后将理论计算所用的频率值作为输入变量,将实测的光斑质心坐标作为输出变量,通过多项式拟合或插值算法,建立一个高精度的映射模型。

完成这一精细校准后,系统的操控灵活性和设计便捷性将得到显著提升。当用户需要在样品池中规划一条复杂的动态光镊路径时,不再需要关心复杂的物理公式或电压-角度转换。只需要在软件中定义好目标路径的几何坐标点,系统会自动调用校准好的映射模型,将这些几何坐标实时转换为AOD需要的精确射频频率序列。这种的设计模式,不仅让光镊路径的编程和调试变得直观、方便,更确保了光镊在高速动态扫描过程中,始终能精准地命中预设位置,极大地提高了光操控的准确性和实验的可重复性。


\section{实验时序与控制}
\label{ch3:sec:control_system}

冷原子实验是一个对时序要求极高的过程,涉及数十路数字/模拟信号的同步控制,时间跨度从微秒级的激光脉冲到秒级的 MOT 加载。本节介绍基于 FPGA 和 DDS(直接数字合成)技术的中央控制系统,以及一个典型的单原子实验周期。

% [PLACEHOLDER: 科学内容填充]
% 1. 硬件架构:FPGA 主控板与各子模块(DDS, DIO, AO)的连接。
% 2. 软件界面:时序编写与编译流程。
% 3. [图表] 典型实验时序图:MOT Load -> PGC -> Trap Load -> Imaging 1 -> Rearrangement -> Experiment -> Imaging 2。

为了确保所有操作在统一的时间轴上严格对齐,系统采用了FPGA主控板加上分布式子模块的星型拓扑结构。作为整个系统的指挥中心,FPGA(现场可编程门阵列)承担着核心的时序生成任务。其最大的优势在于硬实时性——不同于运行操作系统的计算机,FPGA的每一个逻辑门都在确定的时钟周期内执行,不会出现由软件中断导致的随机延迟。FPGA内部运行着一个高速状态机,根据预设的实验时序表,在每一个时钟滴答输出同步指令。DDS 模块(直接数字合成)用于产生冷却光、再抽运光、探测光等激光器的射频驱动信号。DDS芯片(如AD9910)具有快速频率切换速度和极佳的相位连续性,对于原子的冷却和操控过程至关重要。DIO 模块(数字输入/输出)用于控制光路的机械快门、AOM(声光调制器)的开关、以及触发科学相机。这些信号通常需要TTL电平,响应时间要求在纳秒级。AO 模块(模拟输出)用于精确控制激光的强度(通过伺服控制AOM的幅度)或磁场线圈的电流,提供高精度的模拟电压。

实验时序的编写通过Python实现,首先需要通过通过Python脚本定义一个实验序列的抽象描述。随后,Python脚本将这一高层描述传递给硬件抽象层的驱动程序。驱动程序负责将物理参数数据打包成FPGA能够识别的指令帧。转换后的二进制指令帧可以通过高速接口(如PCIe或千兆以太网)批量传输到FPGA板载的指令内存中。传输完成后,Python脚本发送一个“加载完成”的触发信号,FPGA随即脱离上位机的控制,独立开始执行指令。从这一刻起,整个实验周期的时序确定性完全由FPGA的硬件时钟保证,上位机的任何延迟(如操作系统调度、Python垃圾回收)都不会干扰实验进程。

整体上而言,一个典型的冷原子阵列实验周期通常在几十毫秒到几秒之间,可以分解为以下几个阶段。首先是2D MOT和3D MOT的原子装载阶段,需要开启梯度磁场、冷却光和重泵浦光等等,目标是从背景蒸汽中捕获并冷却大量原子,这个阶段时间最长。之后进行偏振梯度冷却,关闭梯度磁场,增大冷却光失谐,并大幅降低重泵浦光功率,以进一步降低原子团温度。第三个阶段是进行单原子装载,将冷原子团装载到远红失谐的光镊阵列中,形成单原子阵列。在光镊开启瞬间,需要确保冷却光和再泵浦光仍然存在几微秒,通过激光诱导的碰撞使得光镊中要么有一个原子,要么没有原子,以辅助单原子装载。为了探测原子装载的情况,需要对原子阵列进行一次微秒级的短暂曝光成像。短暂关闭光镊,打开共振探测光,同时触发相机曝光。原子吸收光子并发出荧光,从而被高灵敏度EMCCD或sCMOS相机捕获。根据成像的结果,就可以知道哪些光镊格点中有原子,哪些没有原子,因此下一步就需要设计出原子重排算法将单原子移动重排成无缺陷的原子阵列。涉及到FPGA读取相机数据,计算重排的轨迹,FPGA实时更新AOD驱动频率的快速循环,移动路径通常需要精心设计以避免加热。之后就可以进行实验阶段了,在完美的原子阵列上执行特定的量子模拟或计算任务,施加特定的激光脉冲(如拉曼脉冲、里德伯激发脉冲),调控原子内部态或相互作用。在实验阶段结束后,最后还需要读取量子操作后的原子状态。再次打开探测光和相机,根据原子是否还存在(或是否发生了态翻转)来读出量子信息。

\section{单原子成像与探测}
\label{ch3:sec:imaging_detection}

在量子模拟实验中,判定每个格点中原子的有无(0 或 1)是信息读出的基础。我们利用高灵敏度的 EMCCD 相机收集原子荧光。为了在曝光期间防止原子丢失,我们实施了基于 795 nm 的原位冷却成像技术。

在原子阵列的实验中,对光镊中单原子的成像非常重要。而要实现高质量的成像,就需要让一个原子能够发射出成百上千个光子,但是原子在发射出光子后也会产生反冲动能从而加热原子,因此在成像过程中就需要一边让原子发出光子一边对原子进行冷却。传统的 780 nm 偏振梯度冷却(PGC)成像往往伴随着较大的光子散射加热,限制了成像的保真度和原子留存率。利用电光调制器(EOM)产生边带,可以实施 $\Lambda$-增强型灰光学黏团($\Lambda$-enhanced gray molasses)冷却。该机制利用相干暗态(Dark State)在冷却的同时显著抑制了加热效应,是实现高信噪比单原子成像的关键。

\subsection{成像光路设计}
\label{ch3:sec:imaging_path}

成像系统需要收集尽可能多的荧光光子,同时滤除背景杂散光。我们设计了一套与光镊生成光路共用高 NA 物镜的成像光路,利用二色镜将 795 nm 的原子荧光从 1064 nm 的陷阱光中分离出来,并成像在 EMCCD 的靶面上。

% [PLACEHOLDER: 科学内容填充]
% 1. [图表] 荧光收集光路图:二色镜、滤光片、管镜(Tube lens)及相机位置。
% 2. 相机配置:EM 增益设置、ROI(感兴趣区域)划分。
% 3. 背景噪声抑制策略。

\subsection{D1 线成像机制}
\label{ch3:sec:d1_imaging_mechanism}

相比于传统的 D2 线成像,D1 线灰光学黏团成像技术在提供荧光信号的同时,能够产生更强的冷却力,从而抵消光散射引起的反冲加热。这种机制使得我们能够在更浅的势阱中,或者更长的曝光时间下,依然保持极低的原子损失率,从而显著提升成像的保真度。

$\Lambda$-增强型gray molasses的原理可以如下描述。两个基态$\ket{g_1}$和$\ket{g_2}$之间通过中间激发态$\ket{e}$进行拉曼耦合,单光子失谐为$\Delta$,双光子失谐对于速度为0的原子为0。$\ket{g_1}\to\ket{e}$和$\ket{g_2}\to\ket{e}$的拉比频率为$\Omega_1,\Omega_2$。在$\Delta\gg\Omega_1,\Omega_2$时可以等效为$\ket{g_1}$和$\ket{g_2}$之间的二能级系统:
\begin{align}
\hat{H}=
\begin{pmatrix} \frac{|\Omega_1|^2}{4\Delta} & \frac{\Omega_1^*\Omega_2}{4\Delta} \\ \frac{\Omega_1\Omega_2^*}{4\Delta} & \frac{|\Omega_2|^2}{4\Delta} \end{pmatrix}
\end{align}
该体系存在一个暗态:$\ket{D}=\frac{\Omega_2\ket{g_1}-\Omega_1\ket{g_2}}{\sqrt{|\Omega_1|^2+|\Omega_2|^2}}$,暗态不与光场有任何耦合,其能量为0。对于速度为v的原子,由于多普勒效应的存在,双光子失谐将不再是0,而是$\delta=\frac{v}{c}\nu_{12}$,$\nu_{12}$表示$\ket{g_1}$和$\ket{g_2}$之间的共振频率。此时哈密顿量变为:
\begin{align}
\hat{H}=
\begin{pmatrix} \frac{|\Omega_1|^2}{4\Delta} & \frac{\Omega_1^*\Omega_2}{4\Delta} \\ \frac{\Omega_1\Omega_2^*}{4\Delta} & \frac{|\Omega_2|^2}{4\Delta}+\delta \end{pmatrix}
\end{align}
在此情况下,暗态将不再存在。在$\delta\ll\frac{|\Omega_1|^2}{4\Delta},\frac{|\Omega_2|^2}{4\Delta}$的情况下,近似到一阶,对应暗态的那个能量分支与原子速度v的关系为:$E_{dark}=\frac{|\Omega_1|^2}{|\Omega_1|^2+|\Omega_2|^2}\delta$,而亮态的能量为$E_{bright}=\frac{|\Omega_1|^2+|\Omega_2|^2}{4\Delta}+\frac{|\Omega_2|^2}{|\Omega_1|^2+|\Omega_2|^2}\delta$。在$\Delta>0$时,亮态的能量高于暗态,原子在光场中运动过程中总是会从亮态被抽运到暗态,自发辐射的光子能量多于吸收的光子能量,相当于原子损失了内态能量。之后热运动又会导致原子从暗态逐渐爬坡转化到亮态,这个过程中原子会被减速,从而实现冷却效果。


% [PLACEHOLDER: 科学内容填充]
% 1. 实验参数:795 nm 激光的失谐量、光强、曝光时间。
% 2. [数据图] 单原子成像直方图(Histogram):双峰结构(0原子/1原子)。
% 3. 保真度计算:利用直方图重叠度估算探测误差。

\section{原子重排与移动系统}
\label{ch3:sec:atom_assembly}

本章的最后一部分介绍如何利用 AOD 移动光镊将随机装载的原子阵列重排为确定性的有序阵列。这一过程被称为“原子组装”(Atom Assembly),是实现大规模量子寄存器的前提。

\subsection{移动光镊算法}
\label{ch3:sec:assembly_algorithm}

原子重排的核心是路径规划算法。系统首先根据第一张原子图像计算出当前的原子分布,然后计算出将原子从初始位置移动到目标位置所需的移动步骤。我们需要设计防碰撞算法(Collision-free algorithms)以确保在移动过程中原子不会丢失,并且尽量减少总的移动时间。

% \subsubsection{算法逻辑}
% % 1. 算法逻辑描述:如何分配源原子到目标空位(如匈牙利算法或贪心算法)。


% \subsubsection{移动波形优化}
% % 2. 移动波形设计:平滑加减速曲线(S-curve 或 Minimum-jerk)以减少加热。

\subsubsection{算法逻辑}

在原子阵列重排技术的早期探索中,基于全局二维格点匹配的路径规划是该领域的主要方法。借鉴 Barredo 等人的先驱性工作~\cite{barredo2016atom},我们将整个随机装载的原子阵列视为一个全局原子库(Reservoir),并在二维平面内寻找源原子到目标空位的最佳映射。在该策略中,系统通常以原子初始位置与目标位置之间曼哈顿距离的平方作为代价函数,并依赖经典的全局匈牙利算法(Hungarian algorithm)来求解最优分配方案。

然而,随着实验技术的快速进步和量子寄存器规模的不断扩大,这种全局搜索策略的局限性逐渐显现。全局匈牙利算法的计算复杂度为 $O(N^3)$(其中 $N$ 为处理的原子总数)。这意味着随着原子阵列规模的增加,重排策略的计算耗时会呈多项式级急速上升。在受限于原子有限真空寿命的实时反馈系统中,这种高昂的计算开销会导致极大的时间延迟,完全无法满足大规模无缺陷阵列制备的实验需求。

为了突破这一算力瓶颈,我们引入了基于声光偏转器(AOD)的多光镊并行操控技术。王帅等人与 Tian 等人的研究表明~\cite{wang2023accelerating, tian2023parallel},利用 AOD 可以同时生成并移动一排或一列光镊,从而实现原子的高效并行重排。深入借鉴这两项工作中的多光镊并行控制思路,我们设计并开发了高度契合硬件特性的“织布机算法”(Loom Algorithm)。该算法的核心逻辑是将复杂的二维全局重排任务进行维度解耦,拆分为多个独立的一维子任务。在第一阶段的横向移动中,算法按行处理并构建过渡层;在第二阶段的纵向移动中,再采用列优先策略将过渡层的原子推入最终目标位置。通过将二维空间降维至一维数组,单次匈牙利算法或排序分配的计算复杂度被大幅降低至 $O(N)$。这不仅从根本上遏制了策略计算耗时的激增,还通过行列解耦实现了最大化的并行操控,从而在维持高效重排策略生成的同时,为向数千个量子比特的扩展奠定了算法基础。
\todo{需要补充织布机算法的演示图}


\subsubsection{移动波形优化}

除了路径规划算法,移动光镊的轨迹波形(Trajectory waveform)设计对原子重排的保真度同样起着决定性作用。在系统构建初期,我们的移动光镊采用最简单的恒定速度(Constant velocity)进行点对点平动。然而,恒速移动意味着光镊在启动和停止瞬间存在极大的加速度脉冲(理论上趋于无限大的狄拉克 $\delta$ 函数阶跃)。为了在如此剧烈的加速度突变下依然稳定地囚禁原子,实验上不得不使用极深的光镊势阱,这不仅增加了非必要的激光功率开销,还对原子造成了显著的参数加热(Parametric heating),导致原子在传输过程中的逃逸率居高不下。

为了解决传输过程中的加热与丢失问题,我们将目光转向了更为平滑的波形设计。近期的量子计算重排实验(如 Chinnarasu 等人与 Zhang 等人的工作~\cite{chinnarasu2025variational, zhang2025leveraging})详细论证了零急动度轨迹(Zero-jerk trajectory)和最小急动度轨迹(Minimum-jerk trajectory)在原子输运中的优越性。研究表明,最小急动度波形能够将原子的振动量子数激发降低接近两个数量级,并且可以极大地减小 AOD 在扫频过程中的声学透镜效应(Acoustic lensing effect)。

这种移动波形优化的策略,本质上是直接借鉴了机器人控制与生物运动学领域中成熟的工程成果。根据 Flash 和 Hogan 提出的数学模型~\cite{flash1985coordination},急动度(Jerk)被定义为位置对时间的三阶导数(即加速度的变化率)。为了使机械臂或操控对象的运动极其平滑并最小化系统受到的内部冲击,控制系统需要使整个轨迹生命周期内的急动度平方积分达到最小。通过求解该泛函积分的最优化问题,可以严格推导出最佳的运动学方程是一个以时间为变量的五次多项式(5th-order polynomial)。我们将这种具有平滑加减速特性的 S 型波形(S-curve)应用到动态光镊的底层驱动中,成功消除了起步与停止阶段的加速度突变。这不仅显著降低了系统对光镊深度的依赖,还极大程度地抑制了原子的热激发,实现了高保真度、低加热的原子跨格点移动。
\todo{需要补充速度、加速度、急动度的图}

\subsection{移动与静态光镊的切换与对齐}
\label{ch3:sec:handover}

% 为了实现高效的原子抓取和释放,AOD 生成的移动光镊必须与 SLM 生成的静态光镊在空间上完美重合。我们开发了一套自动化的配准(Registration)程序,通过扫描 AOD 频率并监测原子荧光或光镊光强,建立两个坐标系之间的精确映射关系,从而优化原子在“静态阱 $\leftrightarrow$ 移动阱”之间的交接(Handover)效率。

为了实现高效的原子抓取和释放,AOD 生成的移动光镊必须与 SLM 生成的静态光镊在空间上完美重合。这一交接(Handover)过程的保真度直接决定了最终重排的成功率。为此,我们开发了一套自动化的配准(Registration)校准程序,该程序主要分为基于相机的粗对齐与基于原子信号的精细对齐两个阶段。

\subsubsection{第一阶段:基于高分辨率相机的粗对齐}
在粗对齐阶段,我们利用光学成像系统来建立动态与静态光镊之间的初始空间映射。首先,通过高分辨率相机对工作平面的光斑进行拍摄,获取待校准的 SLM 静态光镊阵列图以及已知驱动频率的 AOD 动态光镊参考图。基于 AOD 光镊在相机像素坐标上的位置与其射频驱动频率之间的确定性关系,我们可以构建一个二维频率坐标系。随后,通过图像处理算法提取出 SLM 阵列中每个静态光斑的几何中心,并将其映射至上述 AOD 频率平面上。这一过程赋予了每个静态光镊一个初始的特征频率坐标,从而完成了动态光镊与静态光镊在成像平面上的粗对齐。

\subsubsection{第二阶段:基于原子信号的精细对齐}
尽管相机成像可以提供直观的空间参考,但受限于实际光路中的像差、色散,以及相机焦平面与原子实际被囚禁平面之间存在的微小位置偏差,成像系统上的几何对齐并不等同于原子所在平面光学势阱的绝对重合。为了消除这一系统性误差,必须引入原子自身作为局部探针,进行第二步的精细对齐。

精细对齐的核心逻辑依赖于单原子对光学势阱变化的动力学响应。对于每一个囚禁在静态光镊中的单原子,我们施加特定的探测时序:缓慢且绝热地开启 AOD 动态光镊,随后快速(非绝热地)将其关闭。在此操作下,原子的响应可分为三种典型情况:
\begin{itemize}
    \item 严格对齐(Strict Alignment):当 AOD 光镊与 SLM 光镊在空间上完美重合时,原子始终处于复合光学势阱的几何中心。在动态光镊骤然关闭时,静态光镊在原子的当前位置仍具有最深的势阱和最强的束缚势。此时原子经历的加热最小,存活概率达到最高。
    \item 部分交叠(Partial Overlap):当两者存在一定的空间失配但光学势仍有交叠时,绝热开启的动态光镊会改变复合势阱的中心位置,导致原子随之被绝热转移至偏离静态光镊中心(即静态阱深较浅的边缘)的位置。此时若快速关闭动态光镊,原子所处位置的静态势垒高度大幅下降,其动能极易超过截断势垒,从而以极大概率从光镊中逃逸丢失。
    \item 完全无交叠(No Overlap):当两束光镊在空间上相距较远、光学势完全解耦时,AOD 光镊的开启与关闭对 SLM 光镊内的原子不产生任何实质性扰动。因此,原子的存活概率不受影响,丢失率为零。
\end{itemize}


\subsubsection{数学模型与特征拟合}
基于上述物理过程进行二维频率扫描,我们可以获得原子丢失概率 $P$ 随 AOD 驱动频率平面坐标 $(x, y)$ 变化的函数关系 $P(x, y)$。在 AOD 与 SLM 光束均近似为高斯光束的条件下,由于仅在存在轻微空间失配的环形区域内才会引起高概率的原子逃逸,该函数在二维频率平面上呈现出典型的“甜甜圈”(Donut)状拓扑结构。

从数学物理角度来看,这种环形分布特征可以用径向高斯函数的变体或 Ricker Wavelet(墨西哥帽小波)的径向形式来描述。为了对其进行高效的参数化拟合,我们采用了一个最简洁且高度契合物理直觉的唯象公式:
$$p(r) = A \cdot \left( \frac{r}{\sigma} \right)^2 \cdot \exp\left( 1 - \frac{r^2}{\sigma^2} \right)$$
在该公式中,$r$ 表示 AOD 光镊与 SLM 光镊几何中心在频率平面上的径向偏差距离,$A$ 代表最大原子丢失概率的振幅参数,而 $\sigma$ 则是表征光镊束腰尺寸的特征宽度参数。

通过利用该函数模型对二维扫描获取的原子丢失数据进行拟合,我们可以极其精准地提取出该“甜甜圈”中心的极小值坐标 $(x_0, y_0)$,即为动态光镊与静态光镊在原子平面上完美重合的精确射频频率。这种基于单原子探针反馈的闭环校准机制,彻底消除了复杂光路中的对齐误差,为系统实现高保真度、低加热的原子交接与无缺陷阵列重排奠定了关键的硬件控制基础。

% [PLACEHOLDER: 科学内容填充]
% 1. 坐标系配准流程。
% 2. 交接过程中的光强时序控制(Ramp down/up)。
% 3. [数据图] 重排前后的阵列荧光图对比,展示完美阵列的生成。


\end{document}
