% chapters/01-introduction.tex
\documentclass[../main.tex]{subfiles}
\begin{document}
% 章节内容...
% \section{原子内态操控与里德堡激发}

\chapter{原子内态操控与里德堡激发}
\label{chap:state_manipulation}

在上一章中,我们详细阐述了实验装置的搭建以及单原子阵列的制备与重排。为了实现基于里德堡原子的量子模拟与计算,仅实现原子的空间有序排列是不够的,我们还必须具备对原子内态进行高保真度初始化、相干操控以及里德堡态激发的能力。本章将重点介绍我们在 $^{87}\text{Rb}$ 原子内态操控方面的实验技术与结果。首先,我们将讨论如何通过光抽运技术将原子初始化到特定的塞曼子能级;随后,详细介绍利用微波和受激拉曼跃迁对基态超精细能级进行相干操控的原理与实现,特别是针对钟态(Clock state)制备的优化方案;最后,我们将阐述里德堡态的双光子激发系统,重点讨论激光频率的锁定、寻找里德堡信号的“三步走”策略以及光束的时空对齐技术,并通过拉比振荡(Rabi Oscillation)和受激拉曼绝热通过(STIRAP)实验验证了系统的相干性。

\section{原子态制备与初始化}
\label{sec:state_prep}

原子在经过磁光阱(MOT)冷却和偏振梯度冷却(PGC)后,通常随机分布在 $5S_{1/2}, F=1$ 和$5S_{1/2}, F=2$的各个磁量子数 $m_F$ 态上。为了进行确定性的量子操作,我们需要将原子初始化到一个纯态。本节将介绍两种目标态的制备方法:一种是磁敏感的极化态 $|F=2, m_F=2\rangle$,另一种是对磁场噪声不敏感的钟态 $|F=2, m_F=0\rangle$。

\subsection{极化态制备 (F=2, $m_F$=2)}
\label{subsec:polarized_state}

\begin{figure}[ht]
  \centering
  \includegraphics[width=0.5\textwidth]{chapters/chapter_04/figure/optical_pump.png}
  \caption{将原子制备到$|F=2, m_F=2\rangle$的方法}
  \label{ch4_new:optical_pump}
\end{figure}

将原子制备到 $|F=2, m_F=2\rangle$ 态是里德堡实验中最基础的初始化手段。如图\ref{ch4_new:optical_pump}所示,该过程主要依赖于对于塞曼子能级的光抽运机制。首先施加一个偏置磁场定义量子化轴的方向,之后入射一个沿着量子化轴方向传播的态制备光,其频率对应$\ket{5S_{1/2},F=2}\to\ket{5P_{3/2},F=2}$,偏振为$\sigma^+$。同时搭配一个重泵浦光将$\ket{5S_{1/2},F=1}$抽运到$\ket{5S_{1/2},F=2}$。在此情况下,只有$\ket{5S_{1/2},F=2, m_F=+2}$的能级不会与态制备发生作用,其他的磁子能级的原子均会被激发到激发态,然后又由激发态自发辐射落到$5S_{1/2}, F=1$ 和$5S_{1/2}, F=2$的各个磁子能级上。原子在多次散射光子后最终布居在$\ket{5S_{1/2},F=2, m_F=+2}$的能级上,从而完成了态制备。考虑到共振的态制备光会对原子有一定的加热效果,因此,实验上通常只需要入射很弱的制备光与原子作用很短的时间即可实现高效的态制备,同时还不会引起原子的显著加热。

% [PLACEHOLDER: 科学内容填充]
% 1. 详细描述光路配置:与 MOT 冷却光同轴或独立光路。
% 2. 描述 D2 线 F=2 -> F'=2 的能级跃迁图。
% 3. 描述再泵浦光 (Repumper) 防止原子落入 F=1 态的作用。
% 4. [图表占位]:Rb-87 D2 线塞曼能级及光抽运路径图。

\subsection{钟态制备 (F=2, $m_F$=0)}
\label{subsec:clock_state}

相比于极化态,钟态 $|F=2, m_F=0\rangle$ 由于其对一阶塞曼效应不敏感,具有更长的相干时间,是存储量子信息的理想基底。然而,传统的单光子 $\pi$ 偏振泵浦受限于选择定则的禁戒跃迁,难以直接获得高保真度。为此,我们在实验中探索并实现了两种制备方案:微波转移法(Microwave Transfer)和拉曼-回泵循环法(Raman-Depumper Cycle),本节将对比这两种方案的原理与实验效果。

\begin{figure}[ht]
  \centering
  \includegraphics[width=0.95\textwidth]{chapters/chapter_04/figure/clock_state.png}
  \caption{将原子制备到钟态的方法,(a)微波态制备法,(b)拉曼-回泵循环法}
  \label{ch4_new:clock_state}
\end{figure}

如图\ref{ch4_new:clock_state}(a)所示,微波态转移方案比较简单,先通过上面的态制备方法将原子首先全部制备到$\ket{5S_{1/2},F=2, m_F=+2}$能级上,之后打一个6.8GHz左右的$\pi$脉冲将原子全部转移到$\ket{5S_{1/2},F=1, m_F=+1}$能级上,最后再打一个另外一个6.8GHz左右的$\pi$脉冲将原子全部转移到$\ket{5S_{1/2},F=2, m_F=0}$能级上,即可将原子制备到钟态。该方法需要较高功率的微波,不然基态的微波拉比频率无法远大于磁场抖动引入的能级展宽,导致单个Pi脉冲保真度较低。此外,使用微波态转移只能对原子阵列的所有原子都起作用,不能实现空间选择性的态制备。

如图\ref{ch4_new:clock_state}(b)所示,另外一个制备钟态的方法是拉曼-回泵循环法。首先对射能覆盖原子阵列中所有原子的795nm的回泵光(depump laser),此时,$\ket{5S_{1/2},F=2}$的原子将会全部落回到$\ket{5S_{1/2},F=1}$能级上,并分布在$m_F=-1,0,+1$的三个能级上。之后,使用两个拉曼$\pi$脉冲,分别将$\ket{5S_{1/2},F=1,m_F=-1}$和$\ket{5S_{1/2},F=1,m_F=+1}$的原子分别转移到$\ket{5S_{1/2},F=2,m_F=-1}$和$\ket{5S_{1/2},F=2,m_F=+1}$能级上。之后又重新开启795回泵光,以此循环。在这个过程中$\ket{5S_{1/2},F=1,m_F=0}$是不与激光发生作用的暗态,在循环20次左右,即可将原子全部制备到$\ket{5S_{1/2},F=1,m_F=0}$的钟态上。该方法的好处是,拉曼光的束腰可以被制备得很细,从而可以对原子阵列中进行空间选择性地进行态制备,比如只将某一行原子制备到$\ket{5S_{1/2},F=1,m_F=0}$的钟态而对其他位置的的原子不进行态制备。

% [PLACEHOLDER: 科学内容填充]
% 1. 方法一:微波转移法。描述从 |2,2> -> |1,1> -> |2,0> 的两次 Pi 脉冲转移过程。
% 2. 方法二:拉曼-回泵循环法。描述结合 795nm Depumper 和 780nm 拉曼光的循环机制。
% 3. 讨论两种方法的优劣及最终选择。
% 4. [图表占位]:钟态制备的时序图(Pulse Sequence)。

% \subsection{偏振与磁场优化流程}
% \label{subsec:prep_optimization}

% 态制备的保真度极度依赖于泵浦光的偏振纯度以及偏置磁场方向的精确控制。为了实现最优的初始化效果,我们发展了一套从原子系综粗调到单原子精调的优化流程。该流程首先利用偶极阱中原子系综的损耗信号快速确定参数范围,随后通过测量单原子的 Depumping 速率实现参数的极致优化。

% [PLACEHOLDER: 科学内容填充]
% 1. 粗调 (Ensemble Optimization):描述如何监测 795nm Depump 诱导的原子数损耗。
% 2. 精调 (Single Atom Optimization):描述单原子 Depumping 曲线的测量。
% 3. 态制备保真度的测量方法:吹除法 (Blow-away) 的原理与结果。
% 4. [数据占位]:展示优化前后的 Depumping 曲线对比图。

\section{基态相干操控}
\label{sec:ground_control}

在完成态初始化后,我们需要在基态超精细能级之间(特别是 $|0\rangle \equiv |F=1, m_F=0\rangle$ 和 $|1\rangle \equiv |F=2, m_F=0\rangle$)实现任意的单比特旋转门。本节将介绍两种互补的操控手段:用于全局操作的高精度微波系统,以及用于快速寻址操作的远失谐受激拉曼光系统。

\subsection{全局微波操控}
\label{subsec:microwave}

微波操控利用磁偶极跃迁直接驱动基态超精细能级之间的翻转。由于微波波长远大于原子阵列尺寸,它提供了一种高均匀性的全局操控手段。我们利用微波拉比振荡标定系统参数,并通过 Ramsey 干涉实验评估了基态量子比特的相干时间 $T_2^*$,这为后续的磁场噪声分析提供了关键指标。

微波由低相噪的微波源产生,通过微波波导连接到微波开关上,微波开关上有一个TTL通道用来对微波信号进行时序控制,可以调控微波脉冲的发生时间以及脉冲宽度,精确度可以达到ns量级。之后微波信号经过一个大功率的线性微波功率放大器对信号进行等比例放大,放大后的微波信号再通过微波波导连接到一个矩体喇叭天线上,喇叭天线再将微波辐射耦合到真空室内的原子阵列。

全局微波操作首先是可以进行$\ket{0}$和$\ket{1}$之间的拉比振荡,将原子首先制备到$\ket{0}$或$\ket{1}$上,之后入射一个微波脉冲,然后测量处于$\ket{0}$或$\ket{1}$上的原子占比随着脉冲宽度的变化关系。扫描拉比振荡,可以精确确定微波的一些重要参数,比如微波共振频率、微波拉比频率、微波强度分布均匀性等等。良好的拉比振荡是对于原子阵列中所有格点都能以接近$100\%$的态转移效率在$\ket{0}$和$\ket{1}$之间振荡。

量子比特的相干时间是衡量其环境隔离度与操控保真度的核心指标。我们采用Ramsey干涉实验评估基态量子比特的退相干时间 $T_2^*$。实验序列为:$(\pi/2) - \tau - (\pi/2)$。第一个 $\pi/2$ 脉冲将原子制备到叠加态 $(|0\rangle + |1\rangle)/\sqrt{2}$;在自由演化时间 $\tau$ 内,量子比特相位会累积;第二个 $\pi/2$ 脉冲将相位差转换为布居数差异。改变 $\tau$ 并扫描第二个脉冲的相位$\phi$,得到布居的振荡信号:$P(\tau,\phi)=\frac{1}{2}[1-\exp(-(\tau/T_2^*)^{\alpha}) \cos(2\pi\delta\tau+\phi)]$,这里$\delta$是微波相对原子能级的失谐,$\alpha\approx1\;\mathrm{or}\;2$取决于退相干的类型。

(还差一张好看的拉比振荡和Ramsey振荡的图)

% [PLACEHOLDER: 科学内容填充]
% 1. 微波天线设计与系统架构简述。
% 2. 拉比振荡数据展示:Pi 脉冲时间的测定。
% 3. Ramsey 干涉实验描述及 T2* 测量结果。
% 4. [图表占位]:典型的微波拉比振荡曲线和 Ramsey 条纹。

\subsection{远失谐拉曼光系统}
\label{subsec:raman_system}

为了实现微秒量级的高速门操作以及未来的单比特寻址,我们构建了一套基于 $D1$ 线远失谐的双光子受激拉曼跃迁系统。在技术路线的选择上,我们对比了基于光纤 EOM 的直接强度调制方案和基于自由空间相位调制(PM)结合体光栅(VBG)的方案。实验结果表明,后者在高功率耐受性和相位-强度调制转换的稳定性方面具有显著优势,因此被选为本实验的最终方案。

% [PLACEHOLDER: 科学内容填充]
% 1. 物理原理:基于 D1 线远失谐的双光子过程,有效拉比频率公式。
% 2. 技术路线对比:
%    - 路线 A (弃用):光纤 EOM 调幅的局限性(温漂、损耗)。
%    - 路线 B (采用):自由空间 EOM + VBG 的优势。
% 3. 光路构建细节:6.8 GHz 边带产生与光学锁相环 (OPLL)。
% 4. [图表占位]:拉曼激光系统的光路示意图(突出 VBG 部分)。

% \subsection{拉曼光束的校准与对齐}
% \label{subsec:raman_alignment}

% 拉曼光束的聚焦半径通常仅为$\sim$50微米,且与原子阵列平面的重合度直接决定了操控的均匀性。因此,精密的光束校准至关重要。我们采用了一种基于“差分光频移(Differential Light Shift)”的高灵敏度校准方法。该方法利用强拉曼光(无调制)对 $\ket{F=1,mF=1}$ 和 $\ket{F=2,mF=2}e$ 态产生的 MHz 量级频移,通过扫描微波频率精确地描绘出光束在阵列平面内的空间分布,从而实现微米级的对准精度。

% [PLACEHOLDER: 科学内容填充]
% 1. 空间重合 (粗调):辅助光法原理。
% 2. 精密校准 (Fine Tuning):
%    - 差分光频移法的物理机制。
%    - 原位成像 (In-situ imaging):通过扫描微波频率反推光强分布。
% 3. 参数优化:EOM 调制深度与单光子失谐的匹配。
% 4. [数据占位]:展示差分光频移随位置扫描的曲线图。

\section{里德堡态的双光子激发}
\label{sec:rydberg_excitation}

本论文的核心目标是利用里德堡原子之间的强相互作用。本节将详细阐述基于 $780\,\text{nm} + 480\,\text{nm}$ 的双光子激发方案。由于里德堡态对激光频率和电场环境极其敏感,我们发展了一套系统的频率寻找与锁定流程,并实现了激发光束与微观原子阵列的精密耦合。

\subsection{激发方案与激光系统}
\label{subsec:excitation_scheme}

我们采用梯形能级方案,通过中间态 $5P_{3/2}$ 将基态原子激发至 $nS$ 或 $nD$ 里德堡态。为了抑制中间态的自发辐射噪声,激光频率保持较大的单光子失谐 $\Delta$。本小节将推导该三能级系统下的有效拉比频率 $\Omega_{\text{eff}}$ 和双光子失谐量 $\delta$。

\begin{figure}[ht]
  \centering
  \includegraphics[width=0.2\textwidth]{chapters/chapter_04/figure/Rydberg_excitation.png}
  \caption{里德堡激发能级方案}
  \label{ch4_new:Rydberg_excitation}
\end{figure}

考虑三个能级:$\ket{g}$、$\ket{e}$和$\ket{r}$,分别表示基态、$5P_{3/2}$激发态以及里德堡态,三能级下的哈密顿量可以写为:
\begin{align}
\hat{H}=
\begin{pmatrix} \frac{\delta}{2} & \frac{\Omega_1}{2} & 0 \\ 
\frac{\Omega_1^*}{2} & -\Delta &\frac{\Omega_2}{2}  \\
0 & \frac{\Omega_2^*}{2}  & -\frac{\delta}{2}
\end{pmatrix}
\end{align}
这里$\Delta$和$\delta$分别是单光子失谐和双光子失谐,$\Omega_1$和$\Omega_2$和分别是$\ket{g}\leftrightarrow\ket{e}$以及$\ket{e}\leftrightarrow\ket{r}$之间的拉比频率,且满足$\Delta\gg\Omega_1,\Omega_2\gg\delta$。这个哈密顿量可以分为三个部分:$\hat{H}=\hat{H}_0+\hat{V}+\hat{H}_2$,这里(设$\hbar=1$):
\begin{align}
\hat{H}_0=
\begin{pmatrix} 0 & 0 & 0 \\ 
0 & -\Delta & 0  \\
0 & 0  & 0
\end{pmatrix}, 
\hat{V}=
\begin{pmatrix} 0 & \frac{\Omega_1}{2} & 0 \\ 
\frac{\Omega_1^*}{2} & 0 &\frac{\Omega_2}{2}  \\
0 & \frac{\Omega_2^*}{2}  & 0
\end{pmatrix},
\hat{H}_2=
\begin{pmatrix} \frac{\delta}{2} & 0 & 0 \\ 
0 & 0 & 0  \\
0 & 0  & -\frac{\delta}{2}
\end{pmatrix}.
\end{align}
在二阶微扰下,只考虑$\ket{g}$和$\ket{r}$之间的二能级系统,其等效哈密顿量为:$\hat{H}_{\mathrm{eff}}=\hat{P}_{gr}[\hat{H}_2+\hat{V}(E_0-\hat{H}_0)^{-1}\hat{V}]\hat{P}_{gr}$,这里$\hat{P}_{gr}=\ket{g}\bra{g}+\ket{r}\bra{r}$是投影算符,$E_0=0$为$\ket{g},\ket{r}$在$\hat{H}_0$下的本征能量。简单推导一下,即可得到:

\begin{align}
\hat{H}_{\mathrm{eff}}=
\begin{pmatrix}
\frac{\delta}{2}+\frac{|\Omega_1|^2}{4\Delta} & \frac{\Omega_1\Omega_2}{4\Delta} \\
\frac{\Omega_1^*\Omega_2^*}{4\Delta}  & -\frac{\delta}{2}+\frac{|\Omega_2|^2}{4\Delta}
\end{pmatrix}.
\end{align}
从以上公式得到等效拉比频率为$\Omega_{\text{eff}}=\frac{\Omega_1\Omega_2}{2\Delta}$。同时,如果要实现$\ket{g}\to\ket{r}$的高效耦合,需要让等效的双光子失谐为0,也就是满足$\delta=\frac{|\Omega_2|^2-|\Omega_1|^2}{4\Delta}$。

此外,还需要考虑到中间激发态$\ket{e}$自发辐射带来的影响。考虑到一阶微扰后的本征态为:$\ket{\psi^{(1)}}=\ket{\psi_0}+(E_0-\hat{H}_0)^{-1}\hat{V}\ket{\psi_0}$。当$\ket{\psi_0}=\ket{g}$时,$\ket{g^{(1)}}=\ket{g}+\frac{\Omega_1^*}{2\Delta}\ket{e}$;当$\ket{\psi_0}=\ket{r}$时,$\ket{r^{(1)}}=\ket{r}+\frac{\Omega_2}{2\Delta}\ket{e}$。因此,一阶微扰后的$\ket{g^{(1)}}$和$\ket{r^{(1)}}$的自发辐射速率可以分别近似为:$\Gamma_g=\frac{|\Omega_1|^2}{4\Delta^2}\Gamma$以及$\Gamma_r=\frac{|\Omega_2|^2}{4\Delta^2}\Gamma$,这里$\Gamma=2\pi*6\mathrm{MHz}$是$5P_{3/2}$的自发辐射速率。当驱动一个从基态$\ket{g}$到里德堡态$\ket{r}$的$\pi$脉冲,可以近似得到损失率为:
\begin{align}
P_{loss}=1-\exp[-\int_0^{\frac{\pi}{\Omega_{\text{eff}}}}(\Gamma_g\cos^2(\frac{\Omega_{\text{eff}} t}{2})+\Gamma_r\sin^2(\frac{\Omega_{\text{eff}} t}{2}))]\approx \frac{\pi\Gamma}{4\Delta}(\frac{\Omega_1}{\Omega_2}+\frac{\Omega_2}{\Omega_1}),
\end{align}
选择$\Omega_1\approx \Omega_2$,可以看到原子损失率大致正比于$\frac{\Gamma}{\Delta}$。为了减小中间态损失,单光子失谐$\Delta$应该要足够大,实验中我们让失谐增大到$\Delta/2\pi=1\mathrm{GHz}$左右,这样中间态导致的损失可以小于$1\%$。

实验上,我们采用Toptica的带倍频腔的480nm激光器,型号是TA-SHG pro,它是由一个960nm的外腔半导体激光器搭配一个锥形半导体激光放大器以及一个二次谐波倍频腔构成的,480出光功率最高可以达到500mW左右。960nm的外腔半导体激光器可以通过之前所述的PDH稳频方法锁定在超稳腔的腔模上,之后可以通过电光调制器和声光调制器进行精确的激光移频,从而得到我们所需的激光频率。而780nm里德堡激发光则可以根据与MOT同源的光进行频率锁定,之后通过激光移频即可得到。780nm和480nm激光通过对射的方式入射到原子上,可以减少多普勒频移带来的双光子失谐。同时为了确定性地激发到某个特定的里德堡磁子能级,还需要将780nm和480nm激发光沿着量子化轴方向入射以确保激发光偏振是纯的$\sigma^+$或者$\sigma^-$偏振,同时增大偏置磁场将磁子能级劈开,以避免无关磁子能级的干扰。

\begin{figure}[ht]
  \centering
  \includegraphics[width=0.6\textwidth]{chapters/chapter_04/figure/480_laser.png}
  \caption{480nm激光器}
  \label{ch4_new:480_laser}
\end{figure}

% [PLACEHOLDER: 科学内容填充]
% 1. 能级选择:5S -> 5P -> nS/nD,画出能级图。
% 2. 理论推导:绝热消除中间态后的有效哈密顿量。
% 3. 频率锁定系统:
%    - 780 nm:基于 MOT 同源光的移频锁定。
%    - 480 nm:基于 ULE 腔的 PDH 锁定。

% \subsection{480 nm 激光的频率寻找流程}
% \label{subsec:freq_search}

% 里德堡态的能级极其密集且易受环境电场影响,准确找到目标量子数 $n$ 的跃迁频率是实验的一大挑战。我们总结出一套高效的“由粗到细”频率寻找流程:首先利用 MOT 荧光耗散谱确定大致范围,随后利用光镊中的展宽谱进一步定位,最后在自由空间中扫描出傅里叶极限线宽的拉比谱。这一流程极大地提高了实验调试的效率。

% [PLACEHOLDER: 科学内容填充]
% 1. 步骤 1:MOT 耗散谱 (Coarse)。利用高功率和梯度磁场下的原子高激发率。
% 2. 步骤 2:阱中耗散谱 (Intermediate)。利用偶极阱光引起的基态展宽 (~20 MHz) 寻找共振峰。
% 3. 步骤 3:自由空间拉比谱 (Fine)。降低/关闭阱深,扫描短脉冲拉比谱。
% 4. [图表占位]:展示从宽谱到窄谱的频率扫描数据对比。

% \subsection{光束校准与对齐}
% \label{subsec:rydberg_alignment}

% 与拉曼光类似,480 nm 激发光的对齐精度直接影响里德堡激发的效率和均匀性。由于 480 nm 光束不可见且会对 EMCCD 造成干扰,我们采用了基于高灵敏度相机的直接成像辅助对齐方案。此外,本小节还将讨论针对 780 nm 和 480 nm 激发光的偏振校准方法,以确保双光子跃迁的选择定则得到满足。

% [PLACEHOLDER: 科学内容填充]
% 1. CCD 辅助对齐:硬件设置与光斑位置监视。
% 2. 偏振校准:波片角度扫描优化跃迁强度。

\section{相干激发实验结果}
\label{sec:coherent_results}

作为本章的总结,我们将展示在该实验系统上观测到的里德堡态相干激发结果。我们分别测量了直接双光子驱动下的拉比振荡(Rabi Oscillation)以及利用受激拉曼绝热通过(STIRAP)技术实现的高保真度态转移。这些结果验证了我们对原子内态及里德堡激发过程的控制能力,为后续的量子模拟实验奠定了基础。

\subsection{双光子拉比振荡}
\label{subsec:rabi_results}

通过控制激发脉冲的长度,我们观测到了基态与里德堡态之间清晰的布居数振荡。通过对振荡曲线的拟合,我们提取了有效的拉比频率,并分析了主要的退相干机制,包括激光相位噪声、多普勒频移以及中间态散射等因素。

\begin{figure}[ht]
  \centering
  \includegraphics[width=0.8\textwidth]{chapters/chapter_04/figure/Rydberg_excitaion_Rabi.png}
  \caption{双光子拉比振荡}
  \label{ch4_new:Rydberg_excitaion_Rabi}
\end{figure}

直接拉比振荡就是利用上面的失谐情况下的双光子拉曼激发,当双光子拉比振荡时间达到$\pi$脉冲时,原子就会从基态全部跃迁到里德堡态上。如图\ref{ch4_new:Rydberg_excitaion_Rabi}所示,780拉比频率为$\Omega_1/2\pi=$150MHz,480拉比频率为$\Omega_2/2\pi=$42MHz,单光子失谐为$\Delta/2\pi=$1160MHz,因此双光子拉比频率为:$\Omega_{\text{eff}}/2\pi=2.7$MHz,与图\ref{ch4_new:Rydberg_excitaion_Rabi}所示的周期是一致的。由于780和480拉比频率不一致导致的双光子光位移为:$\delta/2\pi=4.5$MHz,实验中也确实观测到了这样的双光子位移。中间态损失可以计算得到大约$P_{loss}=1.6\%$。图\ref{ch4_new:Rydberg_excitaion_Rabi}所示的双光子拉比振荡可以用如下的公式进行拟和:$SR=\frac{B}{2}+\frac{A}{2}\cos(\Omega_{\text{eff}}(t-t_0))\exp(-(t/T_2)^2)$。这里$T_2^*$表示退相干时间,拟合出来$A=0.875,B=1.082,\Omega_{\text{eff}}/2\pi=2.7\mathrm{MHz},T_2=$,因此最高激发效率为:$\frac{A+B}{2}\approx97.9\%$,略小于$1-P_{loss}$,也是符合的。

引起退相干的机制有很多,包括激光器的相位噪声、原子热运动导致的多普勒频移、外部电场或者磁场涨落引起的能级抖动等等。首先,外部环境的扰动是主要的退相干源之一:里德堡原子由于其极高的极化率,对环境中尤其是低频的电场和磁场涨落极其敏感。例如,真空腔体内吸附在电极或窗片上的杂散电荷会引入不均勻的直流斯塔克位移,导致里德堡能级随机抖动;而电源纹波或邻近电子设备产生的交流电磁干扰,则会直接调制双光子跃迁的共振频率。其次,光场的相位与强度噪声同样关键:双光子跃迁通常需要两束相位相干的激光(如780 nm探测光与480 nm耦合光),这两束激光器本身的相位噪声会通过非共振布居转移引入额外的退相位;同时,激光功率的起伏会导致双光子拉比频率的涨落,使得不同实验序列下的累积相位产生随机差异。原子的热运动与碰撞效应也不可忽视:即使采用构型可控的双光子方案,残余的多普勒失谐仍会使部分原子偏离双光子共振条件,造成系综平均后的拉比振荡衰减。

% [PLACEHOLDER: 科学内容填充]
% 1. 展示 Rabi 振荡数据曲线。
% 2. 拟合衰减曲线,提取 Rabi 频率和退相干时间。
% 3. 简要分析主要的退相干来源。

\subsection{受激拉曼绝热路径 (STIRAP)}
\label{subsec:stirap}

为了克服拉比振荡对激光强度起伏敏感的缺点,我们实施了 STIRAP 方案。通过设计合适的双光子脉冲序列,利用暗态绝热演化实现了从基态到里德堡态的高效转移。我们扫描了脉冲延迟 $\Delta t$ 和脉冲宽度等参数,找到了最佳的转移效率平台,并对比了其与直接拉比激发的鲁棒性。



% [PLACEHOLDER: 科学内容填充]
% 1. 原理简述:脉冲序列图示。
% 2. 参数优化数据:效率随脉冲延迟的变化。
% 3. 结论:STIRAP 在鲁棒性方面的优势。

\end{document}
```
\end{document}
