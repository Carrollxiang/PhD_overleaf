% chapters/01-introduction.tex
\documentclass[../main.tex]{subfiles}
\begin{document}
% 章节内容...
% \section{原子内态操控与里德堡激发}

\chapter{原子内态操控与里德堡激发}
\label{chap:state_manipulation}

在上一章中,我们详细阐述了实验装置的搭建以及单原子阵列的制备与重排。为了实现基于里德堡原子的量子模拟与计算,仅实现原子的空间有序排列是不够的,我们还必须具备对原子内态进行高保真度初始化、相干操控以及里德堡态激发的能力。本章将重点介绍我们在 $^{87}\text{Rb}$ 原子内态操控方面的实验技术与结果。首先,我们将讨论如何通过光抽运技术将原子初始化到特定的塞曼子能级;随后,详细介绍利用微波和受激拉曼跃迁对基态超精细能级进行相干操控的原理与实现,特别是针对钟态(Clock state)制备的优化方案;最后,我们将阐述里德堡态的双光子激发系统,重点讨论激光频率的锁定、寻找里德堡信号的“三步走”策略以及光束的时空对齐技术,并通过拉比振荡(Rabi Oscillation)和受激拉曼绝热通过(STIRAP)实验验证了系统的相干性。

\section{原子态制备与初始化}
\label{sec:state_prep}

原子在经过磁光阱(MOT)冷却和偏振梯度冷却(PGC)后,通常随机分布在 $5S_{1/2}, F=2$ 的各个磁量子数 $m_F$ 态上。为了进行确定性的量子操作,我们需要将原子初始化到一个纯态。本节将介绍两种目标态的制备方法:一种是磁敏感的极化态 $|F=2, m_F=2\rangle$,另一种是对磁场噪声不敏感的钟态 $|F=2, m_F=0\rangle$。

\subsection{极化态制备 (F=2, $m_F$=2)}
\label{subsec:polarized_state}

将原子制备到 $|F=2, m_F=2\rangle$ 态是里德堡实验中最基础的初始化手段。该过程主要依赖于 $\sigma^+$ 偏振光的塞曼子能级光抽运机制。通过施加沿量子化轴方向的偏置磁场,利用选择定则限制跃迁路径,使得原子在多次散射光子后最终布居在暗态(Dark State)上。

% [PLACEHOLDER: 科学内容填充]
% 1. 详细描述光路配置:与 MOT 冷却光同轴或独立光路。
% 2. 描述 D2 线 F=2 -> F'=2 的能级跃迁图。
% 3. 描述再泵浦光 (Repumper) 防止原子落入 F=1 态的作用。
% 4. [图表占位]:Rb-87 D2 线塞曼能级及光抽运路径图。

\subsection{钟态制备 (F=2, $m_F$=0)}
\label{subsec:clock_state}

相比于极化态,钟态 $|F=2, m_F=0\rangle$ 由于其对一阶塞曼效应不敏感,具有更长的相干时间,是存储量子信息的理想基底。然而,传统的单光子 $\pi$ 偏振泵浦受限于选择定则的禁戒跃迁,难以直接获得高保真度。为此,我们在实验中探索并实现了两种制备方案:微波转移法(Microwave Transfer)和拉曼-回泵循环法(Raman-Depumper Cycle),本节将对比这两种方案的原理与实验效果。

% [PLACEHOLDER: 科学内容填充]
% 1. 方法一:微波转移法。描述从 |2,2> -> |1,1> -> |2,0> 的两次 Pi 脉冲转移过程。
% 2. 方法二:拉曼-回泵循环法。描述结合 795nm Depumper 和 780nm 拉曼光的循环机制。
% 3. 讨论两种方法的优劣及最终选择。
% 4. [图表占位]:钟态制备的时序图(Pulse Sequence)。

\subsection{偏振与磁场优化流程}
\label{subsec:prep_optimization}

态制备的保真度极度依赖于泵浦光的偏振纯度以及偏置磁场方向的精确控制。为了实现最优的初始化效果,我们发展了一套从原子系综粗调到单原子精调的优化流程。该流程首先利用偶极阱中原子系综的损耗信号快速确定参数范围,随后通过测量单原子的 Depumping 速率实现参数的极致优化。

% [PLACEHOLDER: 科学内容填充]
% 1. 粗调 (Ensemble Optimization):描述如何监测 795nm Depump 诱导的原子数损耗。
% 2. 精调 (Single Atom Optimization):描述单原子 Depumping 曲线的测量。
% 3. 态制备保真度的测量方法:吹除法 (Blow-away) 的原理与结果。
% 4. [数据占位]:展示优化前后的 Depumping 曲线对比图。

\section{基态相干操控}
\label{sec:ground_control}

在完成态初始化后,我们需要在基态超精细能级之间(特别是 $|0\rangle \equiv |F=1, m_F=0\rangle$ 和 $|1\rangle \equiv |F=2, m_F=0\rangle$)实现任意的单比特旋转门。本节将介绍两种互补的操控手段:用于全局操作的高精度微波系统,以及用于快速寻址操作的远失谐受激拉曼光系统。

\subsection{全局微波操控}
\label{subsec:microwave}

微波操控利用磁偶极跃迁直接驱动基态超精细能级之间的翻转。由于微波波长远大于原子阵列尺寸,它提供了一种高均匀性的全局操控手段。我们利用微波拉比振荡标定系统参数,并通过 Ramsey 干涉实验评估了基态量子比特的相干时间 $T_2^*$,这为后续的磁场噪声分析提供了关键指标。

% [PLACEHOLDER: 科学内容填充]
% 1. 微波天线设计与系统架构简述。
% 2. 拉比振荡数据展示:Pi 脉冲时间的测定。
% 3. Ramsey 干涉实验描述及 T2* 测量结果。
% 4. [图表占位]:典型的微波拉比振荡曲线和 Ramsey 条纹。

\subsection{远失谐拉曼光系统}
\label{subsec:raman_system}

为了实现微秒量级的高速门操作以及未来的单比特寻址,我们构建了一套基于 $D1$ 线远失谐的双光子受激拉曼跃迁系统。在技术路线的选择上,我们对比了基于光纤 EOM 的直接强度调制方案和基于自由空间相位调制(PM)结合体光栅(VBG)的方案。实验结果表明,后者在高功率耐受性和相位-强度调制转换的稳定性方面具有显著优势,因此被选为本实验的最终方案。

% [PLACEHOLDER: 科学内容填充]
% 1. 物理原理:基于 D1 线远失谐的双光子过程,有效拉比频率公式。
% 2. 技术路线对比:
%    - 路线 A (弃用):光纤 EOM 调幅的局限性(温漂、损耗)。
%    - 路线 B (采用):自由空间 EOM + VBG 的优势。
% 3. 光路构建细节:6.8 GHz 边带产生与光学锁相环 (OPLL)。
% 4. [图表占位]:拉曼激光系统的光路示意图(突出 VBG 部分)。

\subsection{拉曼光束的校准与对齐}
\label{subsec:raman_alignment}

拉曼光束的聚焦半径通常仅为$\sim$50微米,且与原子阵列平面的重合度直接决定了操控的均匀性。因此,精密的光束校准至关重要。我们采用了一种基于“差分光频移(Differential Light Shift)”的高灵敏度校准方法。该方法利用强拉曼光(无调制)对 $\ket{F=1,mF=1}$ 和 $\ket{F=2,mF=2}e$ 态产生的 MHz 量级频移,通过扫描微波频率精确地描绘出光束在阵列平面内的空间分布,从而实现微米级的对准精度。

% [PLACEHOLDER: 科学内容填充]
% 1. 空间重合 (粗调):辅助光法原理。
% 2. 精密校准 (Fine Tuning):
%    - 差分光频移法的物理机制。
%    - 原位成像 (In-situ imaging):通过扫描微波频率反推光强分布。
% 3. 参数优化:EOM 调制深度与单光子失谐的匹配。
% 4. [数据占位]:展示差分光频移随位置扫描的曲线图。

\section{里德堡态的双光子激发}
\label{sec:rydberg_excitation}

本论文的核心目标是利用里德堡原子之间的强相互作用。本节将详细阐述基于 $780\,\text{nm} + 480\,\text{nm}$ 的双光子激发方案。由于里德堡态对激光频率和电场环境极其敏感,我们发展了一套系统的频率寻找与锁定流程,并实现了激发光束与微观原子阵列的精密耦合。

\subsection{激发方案与激光系统}
\label{subsec:excitation_scheme}

我们采用梯形能级方案,通过中间态 $5P_{3/2}$ 将基态原子激发至 $nS$ 或 $nD$ 里德堡态。为了抑制中间态的自发辐射噪声,激光频率保持较大的单光子失谐 $\Delta$。本小节将推导该三能级系统下的有效拉比频率 $\Omega_{\text{eff}}$ 和双光子失谐量 $\delta$,并介绍用于维持激光频率稳定性的超稳腔(ULE Cavity)锁定系统。

% [PLACEHOLDER: 科学内容填充]
% 1. 能级选择:5S -> 5P -> nS/nD,画出能级图。
% 2. 理论推导:绝热消除中间态后的有效哈密顿量。
% 3. 频率锁定系统:
%    - 780 nm:基于 MOT 同源光的移频锁定。
%    - 480 nm:基于 ULE 腔的 PDH 锁定。

\subsection{480 nm 激光的频率寻找流程}
\label{subsec:freq_search}

里德堡态的能级极其密集且易受环境电场影响,准确找到目标量子数 $n$ 的跃迁频率是实验的一大挑战。我们总结出一套高效的“由粗到细”频率寻找流程:首先利用 MOT 荧光耗散谱确定大致范围,随后利用光镊中的展宽谱进一步定位,最后在自由空间中扫描出傅里叶极限线宽的拉比谱。这一流程极大地提高了实验调试的效率。

% [PLACEHOLDER: 科学内容填充]
% 1. 步骤 1:MOT 耗散谱 (Coarse)。利用高功率和梯度磁场下的原子高激发率。
% 2. 步骤 2:阱中耗散谱 (Intermediate)。利用偶极阱光引起的基态展宽 (~20 MHz) 寻找共振峰。
% 3. 步骤 3:自由空间拉比谱 (Fine)。降低/关闭阱深,扫描短脉冲拉比谱。
% 4. [图表占位]:展示从宽谱到窄谱的频率扫描数据对比。

\subsection{光束校准与对齐}
\label{subsec:rydberg_alignment}

与拉曼光类似,480 nm 激发光的对齐精度直接影响里德堡激发的效率和均匀性。由于 480 nm 光束不可见且会对 EMCCD 造成干扰,我们采用了基于高灵敏度相机的直接成像辅助对齐方案。此外,本小节还将讨论针对 780 nm 和 480 nm 激发光的偏振校准方法,以确保双光子跃迁的选择定则得到满足。

% [PLACEHOLDER: 科学内容填充]
% 1. CCD 辅助对齐:硬件设置与光斑位置监视。
% 2. 偏振校准:波片角度扫描优化跃迁强度。

\section{相干激发实验结果}
\label{sec:coherent_results}

作为本章的总结,我们将展示在该实验系统上观测到的里德堡态相干激发结果。我们分别测量了直接双光子驱动下的拉比振荡(Rabi Oscillation)以及利用受激拉曼绝热通过(STIRAP)技术实现的高保真度态转移。这些结果验证了我们对原子内态及里德堡激发过程的控制能力,为后续的量子模拟实验奠定了基础。

\subsection{双光子拉比振荡}
\label{subsec:rabi_results}

通过控制激发脉冲的长度,我们观测到了基态与里德堡态之间清晰的布居数振荡。通过对振荡曲线的拟合,我们提取了有效的拉比频率,并分析了主要的退相干机制,包括激光相位噪声、多普勒频移以及中间态散射等因素。

% [PLACEHOLDER: 科学内容填充]
% 1. 展示 Rabi 振荡数据曲线。
% 2. 拟合衰减曲线,提取 Rabi 频率和退相干时间。
% 3. 简要分析主要的退相干来源。

\subsection{受激拉曼绝热路径 (STIRAP)}
\label{subsec:stirap}

为了克服拉比振荡对激光强度起伏敏感的缺点,我们实施了 STIRAP 方案。通过设计 Counter-intuitive 的脉冲序列(即 480 nm 光先于 780 nm 光开启),利用暗态绝热演化实现了从基态到里德堡态的高效转移。我们扫描了脉冲延迟 $\Delta t$ 和脉冲宽度等参数,找到了最佳的转移效率平台,并对比了其与直接拉比激发的鲁棒性。

% [PLACEHOLDER: 科学内容填充]
% 1. 原理简述:脉冲序列图示。
% 2. 参数优化数据:效率随脉冲延迟的变化。
% 3. 结论:STIRAP 在鲁棒性方面的优势。

\end{document}
```
\end{document}
