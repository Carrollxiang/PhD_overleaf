% chapters/01-introduction.tex
\documentclass[../main.tex]{subfiles}
\begin{document}
% 章节内容...


\chapter{绪论}

本章旨在阐述基于中性原子的量子模拟研究背景,梳理该领域从单原子囚禁到大规模阵列操控的发展脉络,并引出本论文所致力于解决的关键技术挑战与物理问题。

\section{量子信息与量子模拟概述} 
\subsection{量子力学的第二次革命}
    *   从对微观粒子的被动观测(第一次量子革命)走向对单量子系统的相干操控(第二次量子革命)。
    *   量子计算与量子模拟的基本概念:费曼的构想(利用可控量子系统模拟复杂多体物理)。
\subsection{量子模拟的主流物理平台}
    *   简要对比不同物理体系(超导电路、离子阱、中性原子、光量子等)的特点。
    *   **中性原子平台的独特优势**:
        *   **可扩展性 (Scalability)**:易于在微米尺度集成大规模阵列。
        *   **长相干时间**:基态原子与环境解耦,相干性优异。
        *   **相互作用可控**:里德堡态提供的强长程相互作用(开关特性)。

\section{基于里德堡原子的中性原子量子计算}
\subsection{光镊阵列技术基础}
    *   **光镊囚禁原理**:利用远失谐激光的偶极力俘获中性原子。
    *   **单原子制备**:碰撞阻塞效应(Collisional Blockade)实现亚泊松分布的单原子装载。
    *   **阵列化技术**:从早期的静态晶格到基于 SLM 和 AOD 的可编程光镊阵列。
\subsection{里德堡原子物理基础}
    *   **里德堡态性质**:巨大的极化率、长寿命及强范德瓦尔斯相互作用($V \propto n^{11}/R^6$)。
    *   **里德堡阻塞效应 (Rydberg Blockade)**:两原子间强相互作用导致的激发抑制,是实现多体纠缠和量子逻辑门的核心机制。

\section{国内外研究现状与技术挑战}
\subsection{大规模原子阵列的制备与重排}
    *   回顾国内外课题组(如 Harvard/Lukin组, IOGS/Browaeys组, KAIST/Ahn组等)在原子重排算法和大规模阵列制备方面的里程碑工作。
    *   现状:已实现百比特量级的无缺陷阵列,正在向千比特迈进。
\subsection{高保真度量子操控与模拟}
    *   单比特与多比特门保真度的提升历程(从$\sim 80\%$到$\large 99.5\%$)。
    *   基于里德堡阵列的物理成果:量子多体疤痕(Scars)、拓扑物态、相变动力学等研究进展。
\subsection{当前面临的主要挑战}
    *   原子装载与寿命的限制。
    *   里德堡态激发的保真度与激光噪声控制。
    *   态制备与读出的效率(SPAM errors)。

\section{本论文的主要工作与结构安排}
\subsection{本文的主要研究内容}
    *   **全栈实验平台搭建**:从真空系统、激光稳频到 2D/3D MOT 的实现。
    *   **大规模光镊阵列与原子重排**:结合 SLM 与 AOD 技术,开发高效重排算法,实现确定性单原子阵列。
    *   **高保真度态操控**:实现基态微波操控、钟态制备及里德堡态的相干激发(拉比振荡、STIRAP)。
    *   **前沿物理探索**:基于该平台开展的量子多体疤痕(Quantum Scars)中的信息加扰(OTOC)及晶格规范场论(LGT)动力学研究。
\subsection{论文章节结构}
    *   简述后续各章(理论、装置、操控、实验结果)的安排。

\end{document}
