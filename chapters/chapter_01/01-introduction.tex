% chapters/01-introduction.tex
\documentclass[../main.tex]{subfiles}
\begin{document}
% 章节内容...


\chapter{绪论}

\section{研究背景与意义}

\subsection{复杂量子多体问题的挑战与量子模拟}
% * 引出理解强关联多体量子系统的困难(经典计算机由于指数墙难以模拟)。
% * 回顾费曼关于“量子模拟”的构想:利用高度可控的量子系统来模拟复杂的物理模型。

人类对自然界物质基本性质的认知,在很大程度上建立在对微观粒子相互作用规律的理解之上。从高温超导机制的揭示、新型量子材料的设计,到高能物理中规范场动力学的演化,这些前沿科学问题的核心均指向了同一个物理本质:强关联量子多体系统(Strongly Correlated Quantum Many-body Systems)。

基于里德堡原子阵列的量子多体动力学研究,强调动力学

然而,在理论与计算层面精确求解这类系统面临着一道难以逾越的鸿沟。根据量子力学基本原理,多体系统的状态由整个系统的波函数描述,其所处的希尔伯特空间(Hilbert Space)维度会随着粒子数的增加呈指数级膨胀。这种被称为“指数墙”的维数灾难,使得传统的经典计算机在处理包含数十个以上相互作用粒子的量子多体动力学演化时,即便是采用最先进的张量网络(Tensor Network)或量子蒙特卡洛(Quantum Monte Carlo)等近似数值算法,也往往会遭遇算力瓶颈或符号问题(Sign Problem)。

为了突破这一算力极限,著名物理学家理查德·费曼(Richard Feynman)于 1982 年极具前瞻性地提出了“量子模拟(Quantum Simulation)”的构想:既然经典体系难以有效计算量子现象,那么最自然的方法就是利用一个高度可控的、人工搭建的量子系统,去直接“模拟”另一个难以求解的复杂量子多体系统的演化过程。在量子模拟的框架下,研究人员不再需要进行极其庞大的矩阵对角化运算,而是通过精确调控人工量子系统的哈密顿量(Hamiltonian),使其演化规律与目标物理模型在数学形式上保持一致,进而通过直接测量人工量子系统的最终状态,来提取我们所需的物理观测量。这一构想为解决凝聚态物理、统计力学甚至高能物理中的复杂多体难题提供了一条极其优雅且现实的路径。

\subsection{可编程量子模拟器的崛起与物理平台对比}
% * 简述传统平台(如超冷原子光晶格)在实现单格点操控和任意几何排布上的局限性。
% * 引出**中性原子光镊阵列**:强调其兼具“单粒子级空间排布自由度”与“可调控长程相互作用”的独特优势,是目前最具潜力的可编程量子模拟平台。

自费曼提出量子模拟的构想以来,寻找并构建能够胜任这一任务的理想物理平台,便成为了现代量子物理学界的核心使命。一个理想的量子模拟器不仅需要具备优异的量子相干性以支撑长时间的多体动力学演化,还必须拥有足够大的系统规模以及高度灵活的哈密顿量调控能力。

在过去的几十年中,研究人员在多个物理体系上展开了卓有成效的探索。例如,基于超导电路(Superconducting Circuits)和离子阱(Trapped Ions)的体系在单格点操控精度上表现优异,但超导体系受限于芯片制造工艺的固定拓扑连线,难以在不引入大量额外误差的情况下模拟复杂的长程相互作用或阻挫晶格;而离子阱体系在将规模扩展至数百个粒子时,维持系统整体的相干性和声子模式的纯净度面临着巨大的工程挑战。

在探索大尺度晶格模型的模拟方面,传统的超冷原子光晶格(Optical Lattices)曾长期占据主导地位。研究者利用驻波激光场构建宏观的周期性势阱来囚禁超冷量子气体,成功模拟了玻色-哈伯德(Bose-Hubbard)模型及超流-莫特绝缘体相变。然而,传统光晶格受限于其整体周期性边界条件,极难实现对任意晶格几何构型(如缺失特定格点的随机晶格或存在几何阻挫的卡戈梅 Kagome 晶格)的定制,且在早期缺乏单格点分辨的精确操控与读出能力。

近年来,基于光镊阵列(Optical Tweezer Arrays)的中性原子平台异军突起,完美弥补了上述体系的短板,被公认为目前最具潜力的“可编程量子模拟器(Programmable Quantum Simulator)”。该平台巧妙地结合了光学操控的灵活性与中性原子的优异量子特性,展现出无可替代的独特优势:

\begin{itemize}
\item 单粒子级别的空间排布自由度:借助空间光调制器(SLM)与声光偏转器(AOD)技术,光镊不仅能在微米尺度上独立囚禁单个中性原子,还能在自由空间中将成百上千个原子“拼图式”地重排成一维、二维甚至三维的任意几何构型。这种极高的空间自由度,为模拟各种复杂的晶格模型提供了完美的底层定制画布。

\item 极佳的系统相干性与扩展性:中性原子(如铷、铯等碱金属或锶、镱等碱土金属)作为自然界完美的“全同粒子”,无需像人造超导比特那样进行逐一校准。同时,由于基态中性原子极难与外界环境发生电磁耦合,该系统具有极长的相干寿命,极其适合进行大规模多体系统的长时间相干动力学模拟。

\item 可调控的长程相互作用开关:在基态时,相距几微米的原子之间几乎没有相互作用,系统退化为一个无相互作用的气体。然而,当利用特定频率的激光将原子相干激发至高主量子数的高能态——里德堡态(Rydberg State)时,原子体积瞬间膨胀,诱导出极强的范德瓦尔斯(van der Waals)长程相互作用。这种被称为“里德堡阻塞(Rydberg Blockade)”的物理机制,本质上为系统提供了一个可通过激光脉冲精确控制的“相互作用开关”。
\end{itemize}

正是凭借“任意几何构型”与“可控长程相互作用”的完美结合,中性里德堡原子光镊阵列使得实验物理学家能够以极高的保真度在硬件上直接“捏造”和“刻印”出各种复杂的量子多体哈密顿量,从而为探索超越经典计算极限的新奇物理现象提供了极其强大的工具。



\section{中性里德堡原子阵列的物理图景}

要理解中性原子为何能成为卓越的可编程量子模拟器,首先需要建立对其底层物理机制的直观认识。该平台的核心魅力在于将两种截然不同的物理手段完美结合:利用光镊阵列提供静态的空间排布自由度,以及利用里德堡激发提供动态的强相互作用控制。这两者的结合,使得研究人员能够在实验桌上直接“搭建”出多体系统的哈密顿量。

\subsection{光镊阵列与定制化晶格几何}
% * 描述光镊如何像“定制化画板”一样,通过SLM和AOD技术将中性原子排布成一维、二维乃至阻挫晶格,为各种多体晶格模型提供底层几何结构。
在传统的凝聚态物理研究中,材料的晶格结构是由原子间的化学键天然决定的,研究人员极难在不改变材料化学组成的前提下,随心所欲地改变其底层的几何构型。而光镊技术的出现,彻底颠覆了这一局限。

光镊的本质是利用高度聚焦的强激光束与中性原子产生的诱导偶极矩发生相互作用,从而在激光焦点处形成一个极深的微观光学势阱。基态的中性原子一旦落入其中,就像是被无形的“光学镊子”牢牢夹住。更为关键的是,借助空间光调制器(SLM)或声光偏转器(AOD)等先进的光束整形与偏转技术,原本单一的激光束可以被衍射和调制成包含成百上千个焦点的光斑阵列。

在量子模拟的语境下,这些光镊阵列就像是一块极致灵活的“定制画布”或“微观钉板”。实验人员不仅可以实时监控每一个光镊中原子的装载情况,还能通过动态改变AOD的射频信号,将散乱分布的原子像拼图一样,极其精准地移动到预设的空间位点上。这意味着,无论是一维的原子链、二维的简单方格子,还是在凝聚态理论中极具研究价值但在自然界中罕见的“几何阻挫晶格”(如卡戈梅 Kagome 晶格),都可以被确定性地“打印”在真空中。这种单粒子级别的空间排布自由度,使得中性原子平台能够极其纯粹地模拟特定晶格几何对量子多体相变与动力学演化的影响。

\subsection{里德堡相互作用与哈密顿量工程}
% * 直观解释基态原子的独立性与里德堡态的强范德瓦尔斯相互作用。
% * 阐述“里德堡阻塞效应”如何被巧妙地映射为量子多体模型(如Ising模型)中的自旋相互作用,实现对复杂哈密顿量的精准“捏造”和控制。
拥有了可随意定制的晶格几何仅仅是第一步。要引发复杂的量子多体动力学演化,粒子之间必须具备可控的相互作用。在光镊阵列中,处于基态的相邻中性原子通常相距几微米。在这个尺度下,基态原子之间几乎感受不到彼此的存在,整个系统等效于一个极其纯净、没有任何相互作用的独立自旋集合。然而,当我们利用特定波长的激光,将原子最外层的电子相干激发到主量子数极高(通常 $n > 50$)的高激发态——即里德堡态(Rydberg State)时,原子的物理性质会发生极其戏剧性的变化。里德堡原子的电子轨道半径极大,这导致其极化率和偶极矩呈现出惊人的非线性增长。当两个相邻的原子同时处于里德堡态时,它们之间会产生极其强烈的范德瓦尔斯排斥作用。这种长程相互作用的存在,直接导致了一个极为关键的物理现象:里德堡阻塞效应(Rydberg Blockade)。从直觉上理解,里德堡阻塞效应就像是原子周围形成了一个无形的“排斥气泡”(即阻塞半径)。当一个原子被激光激发到里德堡态后,其强大的相互作用会使其周围一定半径内的其他原子的里德堡能级发生巨大的能量偏移。结果就是,原本共振的激发激光对于这些邻近原子变得严重失谐,从而在物理上“禁止”了阻塞半径内出现第二个里德堡激发。在量子模拟的图景中,这一机制具有极其深远的意义。如果我们把原子的基态视作自旋向下($|\downarrow\rangle$),把里德堡态视作自旋向上($|\uparrow\rangle$),那么激发激光就等效于驱动自旋翻转的横向磁场;而里德堡阻塞效应带来的巨大能量惩罚,则完美地等效于自旋模型(如量子 Ising 模型)中相邻自旋之间极其强烈的排斥相互作用项。因此,里德堡阻塞机制本质上为研究者提供了一个极其干净、响应极快的“相互作用开关”。我们无需复杂的电路连线,只需在微秒级的时间尺度内打开激光,就能瞬间在原本无相互作用的原子群中注入强大的多体相互作用,进而驱动系统在极其复杂的希尔伯特空间中展开相干演化。这种直接在物理层面上“雕刻”系统演化规则的技术,正是所谓的“哈密顿量工程(Hamiltonian Engineering)”。


\section{量子模拟的演进脉络与国内外研究现状}

\subsection{底层控制工具的成熟 (2001-2019)}
% * 梳理技术里程碑:单原子俘获  基于实时反馈的无缺陷阵列重排  里德堡激发保真度的提升。
基于中性原子的量子模拟研究,其发展轨迹是一场从对单个微观粒子的被动孤立俘获,走向对宏观数量级多体量子态主动相干控制的工程与物理革命。这一阶段的核心任务,是为复杂多体物理模拟打造完美的“底层积木”。

早在2001年,研究人员便成功实现了利用光镊(Optical Tweezer)对单个原子的俘获 ,为后续的空间阵列化奠定了基础。2009年,由单原子级别的里德堡阻塞(Rydberg Blockade)效应在实验上被明确观测到 ,证明了利用里德堡态作为临时媒介来诱导原子间强相互作用的可行性,从而为多体系统的哈密顿量工程提供了关键的物理开关。


然而,早期实验面临着一个严峻的阻碍:由于光镊对原子的俘获服从亚泊松分布,单束光镊中装载原子的概率仅约为50\%。这种随机加载导致的阵列缺陷,严重限制了多体模拟的规模与一致性。直到2016年,这一瓶颈被彻底打破。法国巴黎光学研究所(IOGS)的 Browaeys 团队与哈佛大学的 Lukin 团队分别取得重构技术的重大突破:他们利用空间光调制器(SLM)与声光偏转器(AOD),通过实时反馈控制实现了二维无缺陷任意原子阵列的自动组装 ,以及一维冷原子链的逐个原子确定性排布 。

从“随机俘获”向“确定性组装”的跨越,标志着该平台正式具备了可编程量子模拟器的核心特征。依托这种空间排布的高度自由度,Browaeys 团队在同年率先实现了可调控二维里德堡单原子阵列,并成功模拟了量子伊辛(Ising)模型的相互作用 。2017年,Lukin 团队进一步将一维阵列规模拓展至51个原子,在研究里德堡链的淬灭动力学时,首次观测到了违反热力学平衡假设的“量子多体疤痕”(Quantum Many-Body Scars)现象 。这一标志性成果不仅证明了平台在长时间多体相干演化上的极高保真度,更为探索非平衡态多体动力学开辟了全新的视野。随后,三维无缺陷阵列的组装 、受对称性保护的拓扑相的观测 以及对量子 Kibble-Zurek 机制的精确测量 相继涌现。至2019年底,构建大规模、高保真度中性原子量子模拟器的所有底层控制工具已基本成熟。

\subsection{复杂多体量子模拟的黄金时代 (2020-至今)}
% * *(重点梳理表)* 总结当前国际顶尖团队(哈佛Lukin组、威斯康星Saffman组、法国Browaeys组等)利用该平台在**复杂相干演化、拓扑物态、相变动力学**等模拟方向上的正刊级突破。
2020年前后,随着底层硬件与光电控制技术的齐备,中性原子平台迎来了探索深层多体物理的爆发期,正式迈入“复杂多体量子模拟的黄金时代”。研究者们的目光不再局限于证明系统的可控性,而是开始将该平台作为强大的可编程模拟器,去攻克凝聚态物理与高能物理中经典计算机难以模拟的前沿难题。

2021年是这一黄金时代的标志性节点。Lukin 团队成功构建了包含256个原子的可编程量子模拟器,通过精确扫描原子拓扑结构与激光参数,对二维量子相图进行了深度探索 。同年,Browaeys 团队利用数百个里德堡原子实现了二维反铁磁相变动力学的精确模拟 。更具突破性的是,Lukin 团队利用特殊的卡戈梅(Kagome)阻挫晶格排列,在实验中首次观测到了拓扑自旋液体(Topological Spin Liquid)的强有力证据 ,展示了该平台在模拟具有拓扑序的复杂量子态方面的非凡潜力。

随着阵列规模和相干时间的进一步提升,近两年的量子模拟研究开始向着更动态、更复杂的哈密顿量演化迈进。例如,2022年研究人员通过原子的相干搬运,展示了系统在演化过程中动态改变拓扑连通性的能力 ;2024年,通过在包含数百个物理原子的系统中实施复杂的错误抑制与纠缠制备协议,研究者成功演示了在大尺度重构阵列上的高保真度逻辑级相干演化 。2025年,C. Chen 等人进一步在偶极 XY 模型模拟器中完成了元激发能谱的精确测量 ,标志着系统对相互作用能级的解析能力达到了全新高度。

为了清晰呈现这一蓬勃发展的领域脉络,表\ref{chp1:tab:milestone} 总结了近年来基于中性里德堡原子平台在可编程量子模拟方向上的代表性里程碑工作。

\begin{table}[htb]
    \centering
    \caption{基于中性里德堡原子的可编程量子模拟里程碑工作}
    \label{chp1:tab:milestone}
    \begin{tabular}{lllll}
        \toprule
        年份 & 领衔作者/团队 & 核心成就 & 发表期刊 & DOI \\
        \midrule
        2001 & G. Reymond & 首次实现单原子光镊俘获 & Nature & 10.1038/35082512 \\
        2008 & M. Saffman & 首次观测到基态-里德堡拉比振荡 & PRL & 10.1103/PhysRevLett.100.113003 \\
        2009 & Philippe Grangier & 首次观测到单原子的里德堡阻塞 & Nature Physics & 10.1038/nphys1183 \\
        2010 & M. Saffman & 首次演示中性原子受控非门(CNOT) & PRA & 10.1103/PhysRevA.82.030306 \\
        2012 & T. Peyronel (Lukin) & 利用强相互作用原子实现单光子的量子非线性光学 & Nature & 10.1038/nature11361 \\
        2013 & W. Chen (Lukin) & 演示受单存储光子控制的全光开关与晶体管 & Science & 10.1126/science.1238169 \\
        2013 & O. Firstenberg (Lukin) & 在量子非线性介质中观测到光子的吸引相互作用 & Nature & 10.1038/nature12512 \\
        2014 & Adam Kaufman & 演示隧道耦合光镊中两粒子的量子干涉(HOM效应) & Science & 10.1126/science.345.6194.306 \\
        2015 & Adam Kaufman & 通过局部自旋交换实现两个可移动原子的纠缠 & Nature & 10.1038/nature14871 \\
        2016 & Y. Wang (Weiss) & 在3D中性原子阵列中演示基于目标相位移动的单比特门 & Science & 10.1126/science.aad9813 \\
        2016 & H. Labuhn (Browaeys) & 实现可调控二维里德堡单原子阵列并模拟量子伊辛模型 & Nature & 10.1038/nature18274 \\
        2016 & D. Barredo (Browaeys) & 重构技术突破: 首次实现无缺陷任意二维原子阵列的自动组装 & Science & 10.1126/science.aah3778 \\
        2016 & Manuel Endres (Lukin) & 重构技术突破: 实现无缺陷一维冷原子链的逐个原子组装 & Science & 10.1126/science.aah3752 \\
        2016 & Y. Wang & 3D中性原子阵列中的单比特逻辑门 & Science & 10.1126/science.aad9813 \\
        2017 & Hannes Bernien & 51原子模拟器及“量子多体伤疤”观测 & Nature & 10.1038/nature24622 \\
        2018 & D. Barredo & 原子到原子组装的三维合成原子结构 & Nature & 10.1038/s41586-018-0450-2 \\
        2018 & David Weiss & 3D光格中原子的确定性排序(麦克斯韦妖实验) & Nature & 10.1038/s41586-018-0458-7 \\
        2019 & S. de Léséleuc & 观测受对称性保护的拓扑相 & Science & 10.1126/science.aav9105 \\
        2019 & A. Keesling & 量子Kibble-Zurek机制的精确测量 & Nature & 10.1038/s41586-019-1070-1 \\
        2019 & Matthew Norcia & 秒级相干时间的光镊原子钟实现 & Science & 10.1126/science.aay0644 \\
        2020 & Aaron Young & 具备半分钟级相干时间的高稳定光镊钟 & Nature & 10.1038/s41586-020-2827-z \\
        2021 & S. Ebadi & 256原子可编程量子模拟器及量子相探索 & Nature & 10.1038/s41586-021-03582-4 \\
        2021 & Pascal Scholl & 数百个原子规模的二维反铁磁模拟 & Nature & 10.1038/s41586-021-03585-1 \\
        2021 & G. Semeghini & 在Kagome晶格观测到拓扑自旋液体证据 & Science & 10.1126/science.abi8794 \\
        2022 & T. M. Graham & 首个中性原子量子计算机多比特算法演示 & Nature & 10.1038/s41586-022-04603-6 \\
        2022 & D. Bluvstein & 基于相干搬运原子阵列的可重构量子处理器 & Nature & 10.1038/s41586-022-04592-6 \\
        2022 & Aaron Young & 2D量子行走的可编程哈伯德模型模拟 & Science & 10.1126/science.abo0608 \\
        2023 & Shuo Ma & 镱原子擦除转换实现高保真度门 & Nature & 10.1038/s41586-023-06481-y \\
        2023 & Pascal Scholl & 在里德堡量子模拟器中演示擦除纠错 & Nature & 10.1038/s41586-023-06516-4 \\
        2023 & K. Singh & 利用辅助原子阵列进行中路纠错 & Science & 10.1126/science.ade0648 \\
        2024 & D. Bluvstein & 48个逻辑比特的操作与逻辑层纠缠门演示 & Nature & 10.1038/s41586-023-06927-3 \\
        2024 & Ran Finkelstein & 结合原子钟与量子计算的通用操作 & Nature & 10.1038/s41586-024-08005-8 \\
        2025 & H. Manetsch & 6,100个高度相干比特的超大规模阵列实现 & Nature & 10.1038/s41586-025-09641-4 \\
        2025 & Aaron Holman & 基于元表面(Metasurface)的大规模原子俘获 & Nature & 10.1038/s41586-025-09848-5 \\
        2025 & N. Chiu & 3,000个相干比特系统的连续运行模式演示 & Nature & 10.1038/s41586-025-09596-6 \\
        2025 & 潘建伟团队 & 100公里光纤跨度的单原子节点DI-QKD演示 & Science & 10.1126/science.aec6243 \\
        2025 & C. Chen & 偶极XY模型模拟器中的元激发能谱测量 & Science & 10.1126/science.adn0618 \\
        2025 & Muqing Xu & 极低温深冷环境下的中性原子哈伯德模拟器 & Nature & 10.1038/s41586-025-09543-5 \\
        \bottomrule
    \end{tabular}
\end{table}


\subsection{国内研究格局与本平台的建设动机}
* 指明2020年左右(领域正处爆发期)国内在全栈式可编程阵列方向上的空白。
* 导出痛点与动机:为了在国际多体动力学模拟前沿占据一席之地,亟需从零自主搭建一套具备大规模确定性重排与高保真度演化能力的实验平台。

国内发展速度也非常快


\section{博士期间的主要工作与创新点}

\subsection{全栈式中性原子量子操控平台的搭建}
% * 概述自主构建真空系统、激光稳频、复杂光路及高灵敏控制系统的工程突破。
可编程量子模拟器的成功运行,极度依赖于底层硬件系统在空间、时间及频域上的极高稳定性和操控精度。本论文的基础工作,是从零开始,自主设计并完成了一套基于中性铷($^{87}Rb$)原子的全栈式实验平台的搭建,并在真空、激光、光学以及时序控制等多个核心子系统上取得了关键的工程突破:
\begin{itemize} 
\item \textbf{双腔差分超高真空系统的设计与构建}:为了解决冷原子实验中“高通量加载”与“长寿命囚禁”之间的内生矛盾,本文设计并搭建了基于二维与三维磁光阱(2D/3D MOT)的垂直堆叠双腔系统。通过精密的差分小孔流阻设计,系统在源腔维持较高铷蒸气压以保障原子通量的同时,成功将科学腔的背景气压维持在超高真空(UHV)量级。此外,科学腔内高度集成了用于产生磁光阱的线圈、用于电场补偿的电极以及具备大数值孔径(NA)的光学视窗,为微观光镊阵列的生成与里德堡操控提供了极其纯净的物理环境。

\item \textbf{多波段稳频激光系统与精密校准}:复杂的量子操控需求催生了庞大的多波段激光网络。本文自主搭建了涵盖冷却与态制备(\SI{780}{\nm})、成像(\SI{795}{\nm})、光镊囚禁(如 \SI{808}{\nm})以及里德堡双光子激发(\SI{780}{\nm} + \SI{480}{\nm})的外腔半导体激光器(ECDL)与级联放大系统。为了满足里德堡态苛刻的相干激发条件,构建了基于超低膨胀(ULE)腔的 PDH(Pound-Drever-Hall)极窄线宽稳频系统,以及用于远失谐拉曼跃迁的光学锁相环(OPLL)。同时,开发了一套“由粗到细”的原子光谱探针技术,实现了对激光频率、偏振及光束指向的微米级精密校准。

\item \textbf{高集成度复杂光束整形与寻址光路}:为实现大尺度原子阵列的空间复用,本文在光路设计上突破了传统单光束的限制。通过引入空间光调制器(SLM)并结合 Gerchberg-Saxton (GS) 全息迭代算法,实现了大视场、高均匀性的静态背景光镊晶格生成;同时,利用双轴声光偏转器(2D AOD)构建了具备微秒级响应速度的高动态移动光镊。这些高度复杂的光路经由精密的像差校正,通过共用的高 NA 物镜精准聚焦于真空腔内的同一原子平面,在光学层面为原子阵列的确定性重排与寻址提供了核心条件。

\item \textbf{高灵敏度探测与微秒级时序控制系统}:为了对多体量子态进行高保真度读出,在探测端构建了与高灵敏度电子倍增相机(EMCCD)耦合的共焦成像系统,并创新性地整合了 $D1$ 线(\SI{795}{\nm})的 $\Lambda$-增强型灰光学黏团(Gray Molasses)冷却光路,实现了单原子的原位冷却与极低损耗成像。在系统统筹方面,开发了基于 FPGA 与直接数字合成(DDS)技术的全局控制系统。该系统能够对全平台数十路射频信号、微波脉冲、激光快门及相机触发进行严格的同步把控,时序精度达到亚微秒量级,从而保障了复杂多体演化序列的严密执行。
\end{itemize}

综上所述,这一全栈式中性原子量子操控平台的成功搭建,跨越了光、机、电及真空等多个复杂工程领域,为本论文后续实现单原子阵列无缺陷重排、高保真度里德堡相干操控以及开展前沿多体物理模拟,奠定了坚实且不可或缺的硬件基础。

\subsection{量子多体疤痕态的观测与动力学基础}
% * 介绍在该平台上首次实现的疤痕态观测,验证系统长时间多体相干演化的能力,并为后续非热化子空间的研究建立动力学基础。
在平台底层硬件与基态微波操控、里德堡态相干激发等核心技术就绪后,我们首先将目标锚定在非平衡态动力学的经典现象——量子多体疤痕态(Quantum Many-Body Scars)。通过对一维里德堡原子链施加精确的哈密顿量淬灭(Quench),我们成功在特定的初始态下观测到了系统状态的长时间持续振荡。这一工作不仅在极高的保真度下完美验证了本平台出色的多体相干演化能力,更重要的是,疤痕态特殊的非热化动力学特征,为我们后续深入研究特殊子空间内的信息扩散规律提供了一个极其理想且纯净的物理环境。

\subsection{疤痕态子空间中的OTOC与信息加扰研究}
% * 说明如何利用疤痕态的特殊演化轨迹,进一步深入测量OTOC,揭示量子信息扩散与多体系统热化的内在机制。
在确认了系统具备维持疤痕态长相干演化的能力后,本人进一步将研究推向了量子多体物理的深水区——量子信息加扰(Information Scrambling)。我们巧妙地利用疤痕态子空间作为物理载体,设计了复杂的双向时间演化实验时序,成功对系统中的非时序关联函数(Out-of-Time-Order Correlator, OTOC)进行了精细测量。该研究清晰地揭示了局部量子信息在整个多体系统中扩散与全局化的内在微观机制,为理解相互作用系统中的热化过程及其背后的统计力学基础提供了珍贵的实验观测数据。

\subsection{复杂晶格规范场论(LGT)的动力学模拟}
% * 展示在极高哈密顿量控制精度下,突破性地将模拟领域拓展至高能物理交叉方向,实现对LGT动力学演化的精准模拟。
在对阵列几何构型与里德堡哈密顿量的控制精度达到全新高度的前提下,本人将平台的模拟边界拓宽至极具挑战性的高能物理交叉领域。我们将一种特定的晶格规范场论(Lattice Gauge Theory, LGT)模型,通过精妙的理论映射,编码到二维中性原子阵列的自旋自由度与里德堡阻塞相互作用之中。利用该可编程模拟器,我们成功演示了规范场在不同参数区间的动力学演化。这一突破性成果不仅验证了中性原子平台在处理复杂局域规范对称性时的优异表现,也为未来在实验桌上模拟更复杂的高能粒子物理模型开辟了切实可行的路径。


\subsection{论文章节安排}

% * 用一段连贯的文字,行云流水地串联起从底层平台搭建(第二、三、四章)到层层递进的物理模拟探索(第五、六、七章)的逻辑主线。

本文围绕“基于可编程中性原子光镊阵列的量子模拟”这一核心主题展开,整体结构遵循从底层硬件搭建、核心操控技术实现,再到前沿物理探索的严密递进逻辑。全论文共分为八章,具体安排如下:

\begin{itemize} 
\item \textbf{第一章 绪论}:阐述利用中性原子开展多体量子模拟的研究背景与物理图景,系统梳理国内外在可编程阵列方向的技术演进与里程碑工作,并总结本文的主要研究内容与章节安排。

\item \textbf{第二章 理论基础}:系统性地介绍本文涉及的底层理论框架,涵盖光与原子相互作用、光镊偶极囚禁原理、原子冷却机制,以及里德堡原子的特性与阻塞效应机制。

\item \textbf{第三章 实验装置与底层控制}:详细阐述冷里德堡原子量子模拟实验平台的硬件全貌。重点介绍真空系统、复杂光学系统、激光稳频与温控架构,以及大规模阵列的高效重排策略与底层电子学控制系统。

\item \textbf{第四章 高保真度量子操控技术}:聚焦于实现复杂动力学演化所需的操控手段。介绍基态原子的微波操控、钟态制备,以及针对里德堡态的高保真度相干激发(拉比振荡)与激光参数整形技术。

\item \textbf{第五章 量子多体疤痕态的观测与动力学基础}:展示基于上述平台开展的首次重大物理模拟实验。通过一维里德堡链中的淬灭动力学观测疤痕态现象,验证平台的整体相干演化能力,并为后续非热化研究建立动力学基石。

\item \textbf{第六章 量子信息加扰与 OTOC 的测量}:在第五章构筑的疤痕态物理框架下,深入探讨多体系统中的信息扩散机制,展示特殊子空间内非时序关联函数(OTOC)的实验测量方法与结果分析。

\item \textbf{第七章 晶格规范场论(LGT)的动力学模拟}:进一步将平台应用推向高能物理交叉领域,探讨 LGT 模型在里德堡原子阵列中的理论映射机制,并展示相关的规范场动力学模拟结果。

\item \textbf{第八章 总结与展望}:对博士期间的整体研究工作进行回顾与总结,并对中性原子平台在更大规模、更复杂多体系统模拟方向的未来升级及其潜在应用进行展望。
\end{itemize}
\end{document}
