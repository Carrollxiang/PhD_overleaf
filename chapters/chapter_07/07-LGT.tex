% chapters/08-LGT.tex
\documentclass[../main.tex]{subfiles}
\begin{document}
% 章节内容...
\chapter{晶格规范场论中的实时散射与冻结动力学}

理解规范场论的非平衡动力学是高能物理中的核心挑战之一,特别是涉及从真空中产生粒子及其散射的过程,经典计算往往难以处理。基于里德堡原子的量子模拟器为解决这一问题提供了新的途径。 本章介绍利用可编程里德堡原子阵列模拟 (1+1) 维 U(1) 晶格规范场论(Lattice Gauge Theory, LGT)的实验工作。通过设计具有交错失谐的里德堡哈密顿量,我们将原子阵列映射为 Schwinger 模型,并以此研究了高能物理中的基本现象。首先,我们通过淬火动力学观测了禁闭(Confinement)与解禁闭(Deconfinement)相变,以及伴随的味弦断裂(String Breaking)现象。进一步地,利用里德堡原子平台优异的时空操控能力,我们实现了对带电粒子(介子)波包的实时追踪,观测到了介子-介子散射过程中的干涉条纹。为了捕捉散射过程中的瞬态,我们开发了一种“量子冻结”(Quantum Freeze-out)协议,能够在碰撞发生的瞬间通过快速改变哈密顿量参数将系统投影到禁闭相,从而对散射中间态进行成像。这些结果展示了中性原子平台在模拟高能物理复杂动力学方面的潜力。

 
\end{document}
