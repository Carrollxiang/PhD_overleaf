\documentclass[UTF8,oneside,12pt]{ctexbook}   % 小四=12pt

%---------- 1. 页面几何 ----------%
\usepackage{geometry}
\geometry{top=3cm,bottom=2.5cm,left=3cm,right=2.5cm}

\usepackage{setspace}  % 设置行距
\onehalfspacing  % 1.5 倍行距

\usepackage{microtype}  % 两侧对齐微调
\setcounter{tocdepth}{1}  % 目录只列到 section

%================== 你的其他宏包照抄 ==================%
\usepackage{subfiles}
\usepackage{hyperref}
\usepackage{amsmath,amssymb}
\usepackage{graphicx}
\usepackage{dcolumn}
\usepackage{bm}
\usepackage{siunitx}
\usepackage[backend=biber,style=numeric]{biblatex}
\usepackage{booktabs,tabularx,makecell}
\usepackage{physics}
\usepackage{amsfonts}
\usepackage{todonotes}  % 侧边栏批注
%----------------------------------------------------%

\newcommand{\LL}[1]{\textcolor[RGB]{153,51,250}{#1}}
\newcommand{\XDS}[1]{\textcolor[RGB]{250,250,250}{#1}}

\renewcommand\theadfont{\bfseries}
\addbibresource{preamble/references.bib}

\title{中文标题}
\author{项德生}
\date{\today}

\begin{document}

\maketitle

\frontmatter

\chapter*{摘要}
中文摘要内容。
\par\textbf{关键词:} 关键词1;关键词2;关键词3
\addcontentsline{toc}{chapter}{摘要}

\chapter*{Abstract}
English abstract content.
\par\textbf{Keywords:} keyword1, keyword2, keyword3
\addcontentsline{toc}{chapter}{Abstract}

\tableofcontents

\mainmatter

% \chapter{绪论(初稿)}
\subfile{chapters/chapter_01/01-introduction}

% \chapter{理论}
\subfile{chapters/chapter_02/02-theory}

% \chapter{实验装置}
\subfile{chapters/chapter_03/03-setup}

% \chapter{原子内态操控与里德堡激发}
\subfile{chapters/chapter_04/04-Rydberg_control}

% \chapter{量子疤痕态}
\subfile{chapters/chapter_05/05-quantum_scar}

% \chapter{OTOC}
\subfile{chapters/chapter_06/06-OTOC}

% \chapter{LGT}
\subfile{chapters/chapter_07/07-LGT}

% \chapter{展望}
\subfile{chapters/chapter_08/08-Outlook}


\backmatter

\chapter*{致谢}
感谢所有老师与同学。
\addcontentsline{toc}{chapter}{致谢}

\printbibliography[heading=bibintoc, title=参考文献]

\appendix

\chapter{附录}
% 可在此处添加附录内容或通过 \subfile 引入

\end{document}
