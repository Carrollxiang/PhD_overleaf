\documentclass[UTF8,oneside]{ctexbook}
\usepackage{subfiles}
\usepackage{geometry}
\geometry{top=3cm,bottom=2.5cm,left=3cm,right=2.5cm}
\usepackage{hyperref}
\usepackage{amsmath}
\usepackage[backend=biber,style=numeric]{biblatex}
\usepackage{geometry}   % 如果页面边距太窄,可以调宽
\usepackage{booktabs}   % 专业横线
\usepackage{tabularx}   % 自动折行
\usepackage{amsmath}    % 数学符号
\usepackage{makecell}   % 表头多行
\renewcommand\theadfont{\bfseries} % 表头加粗
\addbibresource{preamble/ref.bib}

\title{中文标题}
\author{项德生}
\date{\today}

\begin{document}

\maketitle

\frontmatter

\chapter*{摘要}
中文摘要内容。
\par\textbf{关键词:} 关键词1;关键词2;关键词3
\addcontentsline{toc}{chapter}{摘要}

\chapter*{Abstract}
English abstract content.
\par\textbf{Keywords:} keyword1, keyword2, keyword3
\addcontentsline{toc}{chapter}{Abstract}

\tableofcontents

\mainmatter

% \chapter{绪论(初稿)}
\subfile{chapters/chapter_01/01-introduction}


% \chapter{理论}
\subfile{chapters/chapter_02/02-theory}

% \chapter{实验装置}
\subfile{chapters/chapter_03/03-setup}

% \chapter{原子内态操控与里德堡激发}
\subfile{chapters/chapter_04/04-Rydberg_control}

% \chapter{电场调控}
\subfile{chapters/05-electric_field_control}


% \chapter{量子疤痕态}
\subfile{chapters/06-quantum_scar}

% \chapter{OTOC}
\subfile{chapters/07-OTOC}

% \chapter{LGT}
\subfile{chapters/08-LGT}

% \chapter{展望}
\subfile{chapters/09-Outlook}


\backmatter

\chapter*{致谢}
感谢所有老师与同学。
\addcontentsline{toc}{chapter}{致谢}

\appendix

\chapter{附录}
% 可在此处添加附录内容或通过 \subfile 引入

\end{document}
