% 1. 替换为华科博士学位论文模板
\documentclass[type=doctor]{hustthesis} 

%========== 论文信息配置 (hustsetup) ==========
\hustsetup { 
  info =
  {
    title = { 基于里德堡原子阵列的量子多体动力学研究 },
    title* = { Research on Quantum Many-Body Dynamics Based on Rydberg Atom Arrays }, % 需补充英文标题
    degree = { academic }, 
    degree* = { Physics }, 
    author = { 项德生 },
    author* = { Xiang De-Sheng },
    student-id = { D202080053 },
    clc = {  }, % 中图分类号(示例)
    major = { 物理学 },
    major* = { Physics },
    supervisor = { 李霖\quad{} 教授 },
    supervisor* = { Prof.~ Li Lin },
    date = { 2025-02-28 }, % 或直接删除此行由系统根据编译日期生成
    % committee = { 张三 & 教授 & 华中科技大学 , 李四 & 教授 & 武汉大学 }, % 答辩时填写
  },
}

%========== 宏包引入 ==========
% 注意:原有的 geometry 和 setspace 已经被注释/删除,因为学校模板通常已严格规定了页边距和行距
\usepackage{subfiles}
% \usepackage{hyperref} % hustthesis 模板内部通常已经加载了 hyperref,若报错可保持注释
\usepackage{amsmath,amssymb}
\usepackage{graphicx}
\usepackage{rotating}
\usepackage{dcolumn}
\usepackage{bm}
\usepackage{siunitx}
\usepackage[backend=biber,style=numeric]{biblatex} % 模板和原文件都使用 biblatex
\usepackage{booktabs,tabularx,makecell}
\usepackage{physics}
\usepackage{amsfonts}
\usepackage{todonotes}

% 引入你的参考文献库
\addbibresource{preamble/references.bib} 

%========== 自定义命令 ==========
\newcommand{\LL}[1]{\textcolor[RGB]{153,51,250}{#1}}
\newcommand{\XDS}[1]{\textcolor[RGB]{250,250,250}{#1}}
\renewcommand\theadfont{\bfseries}

\begin{document}

% 自动生成符合学校规范的各类封面页和标题页
\maketitle 

\frontmatter

%========== 摘要部分 ==========
% 使用模板专用的 abstract 环境
\begin{abstract}
  中文摘要内容。
  \keywords 关键词1;关键词2;关键词3
\end{abstract}

\begin{abstract*}
  English abstract content.
  \keywords* keyword1, keyword2, keyword3
\end{abstract*}

% 生成目录
\tableofcontents

\mainmatter

%========== 正文章节包含 ==========
% 保留原有的 subfile 结构
\subfile{chapters/chapter_01/01-introduction}
\subfile{chapters/chapter_02/02-theory}
\subfile{chapters/chapter_03/03-setup}
\subfile{chapters/chapter_04/04-Rydberg_control}
\subfile{chapters/chapter_05/05-quantum_scar}
\subfile{chapters/chapter_06/06-OTOC}
\subfile{chapters/chapter_07/07-LGT}
\subfile{chapters/chapter_08/08-Outlook}

\backmatter

%========== 致谢部分 ==========
% 替换为模板专用的 acknowledgements 环境
\begin{acknowledgements} 
  感谢所有老师与同学。
\end{acknowledgements}

%========== 参考文献 ==========
\printbibliography

\appendix

\chapter{附录}
% 可在此处添加附录内容或通过 \subfile 引入

\end{document}